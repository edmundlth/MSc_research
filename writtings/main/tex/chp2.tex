\documentclass[12pt]{article}
\usepackage{../thesis_style}

\title{Chapter 2: Symbols and Pseudodifferential Operators}
\date{}

\begin{document}
\maketitle
%%%%%%%%%%%%%%%%%%%%%%%%%%%%%%%%%%%%%%%%%%%%%%%%
%%%%%%%%%%%%%%%%%%%%%%%%%%%%%%%%%%%%%%%%%%%%%%%%

\todo{In this chapter we recall the basic definitions and some well known properties about pseudodifferential operators.  We include proofs where appropriate but leave some to the reader.  Our main sources for this material are [Vasy, Melrose, Dyatlov-Zworski]} 

In this chapter, we will introduce pseudodifferential operators (or $\Psi$DOs) that generalises differential operators in Euclidean spaces, $\R^n$. We have shown before (\ref{}) that $\F : \sch(\R^n) \to \sch(\R^n)$ is a continuous linear isomorphism. As such, the action on $\sch(\R^n)$ of a $m^{th}$ order differential operator with rapidly decaying smooth coefficient $c_\alpha \in \sch(\R^n)$ 
\begin{align}
P(x, D) = \sum_{\abs{\alpha} \leq m } c_\alpha(x) D^\alpha_x : \sch(\R^n) \to \sch(\R^n) \label{eq: general diff op} 
\end{align}
is given by 
\begin{align}
P(x, D)u = \frac{1}{(2\pi)^n} \int e^{i (x - y) \cdot \xi} P(x, \xi) u(y) dy \d[\xi]  \label{eq: diff operator action} 
\end{align}
where $P(x, \xi) = \sum_{\abs{\alpha} \leq m} c_\alpha(x) \xi^\alpha$ is the `characteristic polynomial'. \\

Pseudodifferential operators are operators with the similar actions but with $P(x, \xi)$ generalised from polynomial in $\xi$ to \textit{symbols} $a = a(x, y, \xi) \in S^m_\infty(\R^{2n}; \R^n)$. These are smooth functions with certain decay conditions in $\xi$ similar to those of polynomials. We allow symbols to depends on an additional variable taking the role of $y$ in the integral (\ref{eq: diff operator action}) above. A $m^{th}$ order pseudodifferential operators, $A \in \Psi^{m}_{\infty}(\R^n)$ is thus an operator with action 
\begin{align*}
Au(x) = \frac{1}{(2\pi)^n} \int e^{i (x - y) \cdot \xi} a(x, y, \xi) u(y) \d[y] \d[\xi]. 
\end{align*}
The procedure of turning a symbol into a pseudodifferential operator is known as the quantisation procedure. \\

The goal of this chapter is to build a `calculus' of pseudodifferential operators and their symbols. This will include: 
\begin{itemize}
    \item rigourously define the space of symbols $S^{m}_{\infty}(\R^{p}; \R^{n})$. 
    \item show that the quantisation procedure outlined above is well-defined. 
    \item define elliptic sets of symbols and prove invertibility of elliptic symbols in the filtered algebra of all symbols. 
%    \item prove prove several topological and algebraic properties of $S^{m}_{\infty}(\R^{2n}; \R^{n})$. 
%    \item \todo{Something about principal symbol and symbol calculus.}
\end{itemize} 

Here, we will introduce pseudodifferential operators only on Euclidean spaces. However, the results we obtain will invariant under any change of variables. This allow us to define symbols and corresponding pseudodifferential operators on the cotangent bundle $T^*M$ on any smooth manifold $M$ (see for example \ref{}). 

%Furthermore, for constant coefficient elliptic differential operator, i.e. $c_\alpha \in \C$ and $P(x, \xi) = P(\xi)$, the parametrix $G(x, D)$ -- an approximate inverse  to $P(D)$ -- is given simply by 
%\begin{align*}
%G(x, D) f = \frac{1}{(2\pi)^n} \int e^{i (x - y) \cdot \xi} \frac{1}{P(\xi)} f(y) \d[y] \d[\xi]. 
%\end{align*}

\section{Symbols}
The most important characteristic of symbols is their behaviour at infinity. In analogy to differential operator (\ref{eq: general diff op}), we require $a(x, \xi)$ to be bounded in $x$ and have polynomial decay of increasing order with successive $\xi$-derivative.  

\todo{Historical note about hormander symbol class. Vasy's definition. Melrose generalisation etc... }

\begin{fdefinition}
    The space $S^m_\infty(\R^p; \R^n)$ of order $m$ is the space of smooth functions $a \in C^\infty(\R^p \times \R^n)$ such that for all multi-index $\alpha \in \N^p, \beta \in \N^n$
    \begin{align*}
        \abs{D^\alpha_x D^\beta_\xi a (x, \xi)} \leq C_{\alpha, \beta} \sym[\xi]^{m - \abs{\beta}} 
    \end{align*}
    uniformly on $\R^p \times \R^n$. Together with the family of seminorm (indexed by $N \in \N$) 
    \begin{align*}
        \norm[a]_{N, m} = \sup_{(x, \xi) \in \mathrm{Int}(\Omega) \times \R^n} \max_{\abs{\alpha} + \abs{\beta} \leq N} \frac{\abs{D^\alpha_x D^\beta_\xi a(x, \xi)}}{\sym[\xi]^{m - \abs{\beta}}} 
    \end{align*}
    gives a Frechet topology to $S^m_\infty(\Omega; \R^n)$. \\
\end{fdefinition}
\begin{rem} \hfill 
    \begin{itemize}
        \item Above, we defined symbols $a = a(x, \xi)$ as smooth functions in $\R^p_x \times \R^n_\xi$ for some $p, n \in \N$. When we define pseudodifferential operators, we will then take $p = 2n$, so that $a = a(x, y, \xi) \in S^{m}_\infty(\R^{2n}; \R^n)$. The variables $x, y, \xi \in \R^n$  are collectively known as the phase space variables. The variables $x, y$ are sometime known as the ``space variables" and $\xi$ is often known as the ``dual", ``Fourier" or ``fibre" variable.
        
        \item  We can also defined the space of symbol, $S^m_\infty(\Omega; \R^n)$ on a set with non-empty interior $\Omega \subset \R^p$ with $\Omega \subset \overline{\mathrm{Int}(\Omega)}$, so that the supremum in the seminorm above is taken over $(x, \xi) \in \mathrm{Int}(\Omega) \times \R^n$. 
        
        \item The subscript $\infty$ in $S^m_\infty$ refers the uniform boundedness condition in the space variable.
        
        \item There are various generalisation of the symbol spaces which result in similar pseudodifferential calculus. For instance, we could allow polynomial growth in the space variable, and extra-decay in the dual variable, resulting in the space $S^{m, l_1, l_2}_{\infty, \delta}(\R^{2n}_{x, y}; \R^n_{\xi})$, with element $a \in C^\infty(\R^{3n})$ satisfying
        \begin{align*}
        \abs{D^\alpha_xD^\beta_y D^\gamma_\xi a} \leq C \sym[x]^{l_1} \sym[y]^{l_2} \sym[\xi]^{m - \abs{\gamma} + \delta \abs{(\alpha, \beta, \gamma)}}. 
        \end{align*}
        
        \item \todo{polyhomogeneous subspace}
    \end{itemize}
\end{rem}

\begin{example}
    \todo{EXAMPLES!!}
    \begin{enumerate}
        \item \todo{microlocal cut-off}
        \item \todo{$P(x, \xi)$}
        \item polyhomogeneous ones
    \end{enumerate}
\end{example}



%%%%%%%%%%%%%%%%%%%%%%%%%%%%%%%%%%%%%%%%%%%%%%%%
%%%%%%%%%%%%%%%%%%%%%%%%%%%%%%%%%%%%%%%%%%%%%%%%
\section{Properties of Symbols}

\subsection{Symbols form graded commutative *-algebra} 
We shall establish the following summarising theorem. 
\begin{ftheorem}[Summary]
    Given $m \in \R$, $p, n \in \N$, then
    \begin{enumerate}
        \item $S^{\infty}_\infty(\R^{p}; \R^n)$ is a graded commutative *-algebra over $\C$ with continuous  inclusion  $$\iota : S^{m}_\infty(\R^p; \R^n) \to S^{m'}_\infty(\R^p; \R^n)$$ for all $m \leq m'$. 
        \item $S^{-\infty}_\infty(\R^{p}; \R^n)$ is dense in $S^{m}_\infty(\R^{p}; \R^n)$ in the topology of $S^{m + \epsilon}_\infty(\R^{p}; \R^n)$ for any $\epsilon \in \R_{> 0}$. 
    \end{enumerate}
    
\end{ftheorem}

\todo{Refactorise this sections so that it has better narative flow. }

\todo{Something about classical symbols to prepare for propagation of singularity}

We first prove continuous inclusion of lower order into higher order symbol space. 
\begin{fprop}
    Let $p, n \in \N$ be given. If $m, m' \in \R$ such that $m \leq m'$, then $S^m_\infty(\R^p; \R^n) \subset S^{m'}_\infty(\R^p; \R^n)$. Furthermore, the inclusion map 
    \begin{align*}
        \iota: S^m_\infty(\R^p; \R^n) \hookrightarrow S^{m'}_\infty(\R^p; \R^n)
    \end{align*}
    is continuous. 
\end{fprop}
\begin{proof}
    Let the real numbers $m \leq m'$ be given. We note that for any $\xi \in \R^n$
    \[
    \sym[\xi]^m \leq 1 \cdot \sym[\xi]^{m'}
    \]
    and thus if $a \in S^{m}_\infty(\R^p; \R^n)$, we have that $\forall \alpha \in \N^p, \forall \beta \in \N^n$
    \[
    \abs{D^\alpha_x D^\beta_\xi a(x, \xi)} \leq C \sym[\xi]^{m - \abs{\beta}} \leq C \sym[\xi]^{m' - \abs{\beta}} 
    \]
    which show that $a \in S^{m'}_\infty(\Omega; \R^n)$ as well. \\
    
    To show that $\iota$ is a continuous inclusion, it suffices to show that 
    \[
    \norm[\iota(a)]_{N, m'} \leq C \norm[a]_{N, m}
    \]
    for any $a \in S^{m}_\infty(\Omega; \R^n)$ and $N \in \N$. Indeed, this bound holds since 
    \[
    \frac{\abs{D^\alpha_x D^\beta_\xi a(x, \xi)}}{\sym[\xi]^{m' - \abs{\beta}}}  
    \leq \frac{\abs{D^\alpha_x D^\beta_\xi a(x, \xi)}}{\sym[\xi]^{m - \abs{\beta}}}. 
    \]
    for any $x, \xi \in \R^p \times \R^n$. 
    
\end{proof}

Next, we prove the filtration property of the symbol spaces. 

\begin{fprop}
    Let $p, n \in \N$ be given.  Then, for any $m, m' \in \R$, 
    \[
    S^{m}_\infty(\Omega; \R^n) \cdot S^{m'}_\infty(\Omega; \R^n) = S^{m + m'}_\infty(\Omega; \R^n)
    \]
\end{fprop}
\begin{proof}
    Let $a \in S^{m}_\infty(\R^p; \R^n)$ and $b \in S^{m'}_\infty(\R^p; \R^n)$ be given. By Leibinz formula \ref{}, we have that for all multi-index $\alpha \in \N^p$, $\beta \in \N^n$, 
    \begin{align*}
        & \sup_{(x, \xi) \in \R^p \times \R^n} \frac{\abs{D^\alpha_x D^\beta_\xi a(x, \xi)b(x, \xi)}}{\sym[\xi]^{(m + m') - \abs{\beta}}} \\
        &\leq  \sum_{\mu \leq \alpha, \gamma \leq \beta} \binom{\alpha}{\mu} \binom{\beta}{\gamma} \sup_{(x, \xi) \in \R^p \times \R^n} \frac{\abs{D^\mu_x D^\gamma_\xi a(x, \xi)} \abs{D^{\alpha - \mu}_x D^{\beta - \gamma}_\xi b(x, \xi)}}{\sym[\xi]^{(m + m') - \abs{\beta}}} \\
        &\leq \sum_{\mu \leq \alpha, \gamma \leq \beta} \binom{\alpha}{\mu} \binom{\beta}{\gamma} C \sup_{\xi \in \R^n} \frac{\sym[\xi]^{m - \abs{\gamma}} \sym[\xi]^{m' - \abs{\beta - \gamma}}}{\sym[\xi]^{(m + m') - \abs{\beta}}} \\
        &= \sum_{\mu \leq \alpha, \gamma \leq \beta} \binom{\alpha}{\mu} \binom{\beta}{\gamma} C \sup_{\xi \in \R^n} \sym[\xi]^{\abs{\beta} - (\abs{\beta - \gamma} + \abs{\gamma})} \\
        & \leq \sum_{\mu \leq \alpha, \gamma \leq \beta} \binom{\alpha}{\mu} \binom{\beta}{\gamma} C \\
        &< \infty
    \end{align*}
    where we have use the property of multi-index that $\abs{\beta} = \abs{\beta - \mu} + \abs{\mu}$.  We have thus shown that $S^{m}_\infty(\R^p; \R^n) \cdot S^{m'}_\infty(\R^p; \R^n) \subset S^{m + m'}_\infty(\R^p; \R^n)$\\
    \\
    For the reverse inclusion, let $c \in S^{m + m'}_\infty(\R^p; \R^n)$ be given. Define 
    \begin{align*}
        &a(x, \xi) = \sym[\xi]^m\\
        &b(x, \xi) = \frac{c(x, \xi)}{a(x, \xi)}
    \end{align*}
    and observe that 
    \begin{itemize}
        \item $a \in S^{m}_\infty(\R^p; \R^n)$:  It is clear that $a$ is smooth in both $x$ and $\xi$. It is independent of $x$ and thus any $x$ derivative gives 0. We need only to check that for all $\beta \in \N^n$, 
        \[
        \abs{D^\beta_\xi \sym[\xi]^m} \leq C \sym[\xi]^{m - \abs{\beta}}
        \]
        which can be proven by induction on $n$ and $\beta$. We shall only prove the base case where $n = 1$ and $\beta = 1$. We have 
        \begin{align*}
            \abs{D_\xi \sym[\xi]^m} = \abs{\p_\xi (1 + \xi^2)^{m/2}} = \abs{m\xi \sym[\xi]^{m - 2}} = \abs{m \frac{\xi}{\sym[\xi]}} \sym[\xi]^{m - 1} \leq \abs{m} \sym[\xi]^{m - 1}
        \end{align*}
        where we have used the fact that $\abs{\xi} \leq \sym[\xi]$ for all $\xi$. 
        \item $b \in S^{m'}_\infty(\R^p; \R^n)$:  We note first that $\sym[\xi]^m \neq 0$ for all $\xi \in \R^n$ and thus $b$ is well-defined. Since division by $\sym[\xi]^m$ does not affect any of the $x$ derivative, we only need to show that for any $\beta \in \N^n$, we have
        \[
        \abs{D^\beta_\xi b(x, \xi)} \leq C \sym[\xi]^{m + m' - \abs{\beta}}
        \]
        for some constant $C > 0$ uniformly in $\xi$. Indeed, observe that by the Leibinz formula
        \begin{align*}
            \abs{D^\beta_\xi b(x, \xi)} 
            & \leq \sum_{\mu \leq \beta} \binom{\beta}{\mu} \abs{D^\mu_\xi c(x, \xi)} \abs{D^{\beta - \mu} \sym[\xi]^{-m}} \\
            & \leq C \sum_{\mu \leq \beta} \binom{\beta}{\mu} \sym[\xi]^{m + m' - \abs{\mu}} \sym[\xi]^{-m - \abs{\beta - \mu}} \\
            & \leq C \sum_{\mu \leq \beta} \binom{\beta}{\mu} \sym[\xi]^{ m' - (\abs{\mu} +  \abs{\beta - \mu})} \\
            & = C \sum_{\mu \leq \beta} \binom{\beta}{\mu} \sym[\xi]^{ m' -\abs{\beta}} \\
            &= C 2^{\beta} \sym[\xi]^{ m' -\abs{\beta}} 
        \end{align*}
        where we have use the definition of $c$ and applied the result proven for $a$ with $m \mapsto -m$. Thus, $b \in S^{m'}_\infty(\R^p; \R^n)$. 
    \end{itemize}
    It is clear that $a \cdot b = c$ and we have therefore shown that $S^{m + m'}_\infty(\R^p; \R^n) \subset S^{m}_\infty(\R^p; \R^n) \cdot S^{m'}_\infty(\R^p; \R^n)$. 
\end{proof}

It is clear that 
$$ a^*(x, \xi) := \overline{a(x, \xi)} \in S^{m}_\infty(\R^{2n}; \R^n) \iff a \in S^{m}_\infty(\R^{2n}; \R^n). $$ Together with the result above, we obtain the desired algebraic structure for $S^{\infty}_\infty(\R^{2n}; \R^n)$ as claimed in \ref{}. \\



\subsection{Density of residual symbol space} 
Next, we have a rather technical density result. It states that the residual space $S^{-\infty}_\infty(\R^{p}; \R^n)$ is dense in $S^{m}_\infty(\R^p; \R^n)$, but only as a subspace of $S^{m + \epsilon}_\infty(\R^{p}; \R^n)$. The main strategy in this proof is to approximate any symbol with the very same symbol but cut off by a compactly supported function. As such, the main reason we \textbf{cannot} have density of $S^{-\infty}_\infty(\R^p; \R^n)$ in $S^m_\infty(\R^p; \R^n)$ is the same reason to the fact that Schwartz functions are not dense in the space of bounded smooth functions. For example, the first Schwartz seminorm
\begin{align*}
\sup_{x \in \R^n} \abs{f(x) - 1} \geq 1
\end{align*}
between of any rapidly decaying function $f$ and the constant function $1 \in S^{0}_\infty(\R^{p}; \R^n)$ is always bounded below by $1$.

\begin{flemma}
    Given any $m \in \R$, $n, p \in \N$ and $a \in S^{m}_\infty(\R^p; \R^n)$, there exist a sequence in $S^{-\infty}_\infty(\R^p; \R^n)$ that is bounded in $S^{m}_\infty(\R^p; \R^n)$ and converges to $a$ in the topology of $S^{m + \epsilon}_\infty(\R^p; \R^n)$ for any $\epsilon \in \R_{> 0}$. 
    %In other words, for any $m \in \R$ and $\epsilon > 0$, $S^{-\infty}_\infty(\R^p; \R^n)$ is dense in $S^{m}_\infty(\R^p; \R^n)$ with the topology of $S^{m + \epsilon}_\infty(\R^p; \R^n)$.
\end{flemma}
\begin{proof}
    Let $a \in S^{m}_\infty(\Omega; \R^n)$ and $\epsilon \in \R_{>0}$ be given. Let $\chi \in C^\infty_c(\R^n)$ be a non-negative smooth cut-off function, i.e. $0 \leq \chi \leq 1$  and satisfies
    \begin{align*}
        \chi(\xi) = 
        \begin{cases}
            1 & ,\abs{\xi} < 1 \\
            0 & ,\abs{\xi} > 2. 
        \end{cases}
    \end{align*}
    
    Then, for each $k \in \N$, we define
    \[
    a_k(x, \xi) = \chi\brac{\frac{\xi}{k}} a (x, \xi). 
    \]
    Now, given arbitrary $N, k \in \N$, observe that 
    \[
    \abs{a_k} \leq C \sym[\xi]^{-N} 
    \]
    since $a_k$ is compactly supported in $\xi$ (as $\chi$ is compactly supported). Furthermore, by Leibinz formula and the symbol estimates on $a \in S^{m}_\infty(\R^p; \R^n)$, we have
    \begin{align*}
        \abs{D^\alpha_x D^\beta_\xi a_k(x, \xi)} 
        & \leq \sum_{\mu \leq \beta} \binom{\beta}{\mu}k^{-\abs{\mu}} \brac{D^\mu_\xi \chi}\brac{\frac{\xi}{k}} \abs{D^\alpha_x D^{\beta - \mu}_\xi a(x, \xi) } \\
        & \leq C \sum_{\mu \leq \beta} \binom{\beta}{\mu}k^{-\abs{\mu}} \brac{D^\mu_\xi \chi}\brac{\frac{\xi}{k}} \sym[\xi]^{m - \abs{\beta - \mu}}. 
    \end{align*}
    Since $\chi$ and all its derivatives are compactly supported, each term in the sum above is zero outside of a compact region. Hence, given any $N \in \N$, for a big enough $C > 0$, 
    \[
    \abs{D^\alpha_x D^\beta_\xi a_k(x, \xi)} \leq C' \sym[\xi]^{-N}
    \]
    which allow us to conclude that $a_k \in S^{-\infty}_\infty(\R^p; \R^n)$ and is a bounded sequence in $S^{m}_\infty(\R^{p}; \R^n)$.\\
    \\
    It remains to show that $\Lim[k] a_k = a $ in $S^{m + \epsilon}_\infty(\Omega; \R^n)$. In the first symbol norm, we observe that, using the symbol estimate for $a$ 
    \begin{align*}
        \norm[a_k - a]_{0, m + \epsilon} 
        & = \sup_{(x, \xi) \in \R^p \times \R^n} \frac{\abs{a_k(x, \xi)}}{\sym[\xi]^{m + \epsilon}} \\
        & = \sup_{(x, \xi) \in \R^p \times \R^n} \frac{\abs{(1- \chi(\xi/k))} \abs{a(x, \xi)}}{\sym[\xi]^{m + \epsilon}} \\
        & \leq \norm[a]_{0, m} \sup_{\xi \in \R^n} \frac{\abs{(1- \chi(\xi/k))}}{\sym[\xi]^{\epsilon}} \\
        & \leq \norm[a]_{0, m} \sym[k]^{- \epsilon}\\
        & \to 0
    \end{align*}
    as $k \to \infty$, since $\abs{( 1- \chi(\xi/ k))}$ is 0 in the region $\abs{\xi} \leq k$ and bounded by $1$ otherwise.  We remark upon the necessity of the extra decay given by $\sym[\xi]^{-\epsilon}$ factor. For other symbol norms we shall again use Leibinz formula: 
    \begin{align*}
        &\sup_{(x, \xi) \in \R^p \times \R^n} \frac{\abs{D^\alpha_x D^\beta_\xi a_k(x, \xi)}}{\sym[\xi]^{m + \epsilon - \abs{\beta}}} \\
        & \leq \sup_{(x, \xi) \in \R^p \times \R^n} \frac{1}{\sym[\xi]^{m + \epsilon - \abs{\beta}}} \sum_{\mu \leq \beta} \binom{\beta}{\mu}k^{-\abs{\mu}} \brac{D^\mu (1 - \chi)}\brac{\frac{\xi}{k}} \abs{D^\alpha_x D^{\beta - \mu}_\xi a(x, \xi) } \\
        & \leq  \sup_{(x, \xi) \in \R^p \times \R^n} \frac{C}{\sym[\xi]^{m + \epsilon - \abs{\beta}}} \sum_{\mu \leq \beta} \binom{\beta}{\mu}k^{-\abs{\mu}} \brac{D^\mu (1 - \chi)}\brac{\frac{\xi}{k}} \sym[\xi]^{m - \abs{\beta - \mu}}  \\
        & = C \sup_{\xi \in \R^n} \sum_{\mu \leq \beta} \binom{\beta}{\mu}k^{-\abs{\mu}} \brac{D^\mu (1 - \chi)}\brac{\frac{\xi}{k}} \sym[\xi]^{- \epsilon - \abs{\mu}}  \\
        & \leq C' k^{- \epsilon} \\
        & \to 0
    \end{align*}
    as $k \to \infty$ by the same argument as before. Thus, we have proven that $a_k \to a$ as $k \to \infty$ in $S^{m + \epsilon}_\infty(\R^p; \R^n)$. 
    
\end{proof}


\subsection{Elliptic symbols} 
\todo{motivation for elliptic symbols} 

\begin{fdefinition}
    Given $p, n \in \N$ and $m \in \R$, an order $m$ symbol $a \in S^m_\infty(\R^p; \R^n)$ is (globally) \textbf{elliptic} if there exist $\epsilon \in \R_{>0}$ such that 
    \[
    \inf_{\abs{\xi} \geq 1/\epsilon} \abs{a(x, \xi)} \geq \epsilon \sym[\xi]^m. 
    \]
\end{fdefinition}
The importance of elliptic symbol is that they are invertible modulo $S^{-\infty}_\infty(\R^p; \R^n)$. 

\begin{flemma}
    Let $p, n \in \N$, $m \in \R$ be given and let $a \in S^m_\infty(\R^p; \R^n)$ be an elliptic symbol of order $m$. Then there exist a symbol $b \in S^{-m}_\infty(\R^p; \R^n)$ such that 
    \[
    a \cdot b - 1 \in S^{-\infty}_\infty(\R^p; \R^n). 
    \]
\end{flemma}
\begin{proof}
    We shall follow the general strategy of inverting the symbol outside of a compact set. Let $\phi \in C^\infty_c(\R^n)$ be a smooth cut off function, i.e $0 \leq \phi \leq 1$ and $ \phi(\xi) = 1$ for $\abs{\xi} < 1$ and $\phi(\xi) = 0 $ for $\abs{\xi} > 2$. \\
    
    Let $a \in S^{m}_\infty(\Omega; \R^n)$ be an elliptic symbol, that is, for any fixed $\epsilon \in \R_{> 0}$, we have 
    \[
    \abs{a(x, \xi)} \geq \epsilon \sym[\xi]^m
    \]
    for any $\abs{\xi} \geq 1/\epsilon$. Thus, we can define 
    \begin{align*}
        b(x, \xi) = 
        \begin{cases}
            \frac{1 - \phi( \epsilon \xi /2)}{a(x, \xi)} & \abs{\xi} \geq 1/ \epsilon \\
            0 & \abs{\xi} < 1 / \epsilon. 
        \end{cases}
    \end{align*}
    We check: 
    \begin{description}
        \item[$b$ is well-defined and smooth. ] \hfill \\
        We note that $\abs{a(x, \xi)} > 0$ whenever $\abs{\xi} \geq 1/\epsilon$ and therefore $b$ is well defined in that region. For smoothness, we note first that $b$ is smooth in the regions $\abs{\xi} > 1/ \epsilon$ and $\abs{\xi} < 1/\epsilon$. Set $\delta = 1/(2 \epsilon)$. In the region where $1/\epsilon - \delta < \abs{\xi} < 1/\epsilon + \delta$, we have $\abs{\epsilon \xi/ 2} < 1/\epsilon$ and therefore $b(x, \xi) \equiv 0$ in this region and is thus smooth. Since the we have covered $\Omega \times \R^n$ by the three chart domain above, $b$ is smooth by the (smooth) gluing lemma. 
        
        \item[$b$ is a symbol of order $-m$.  ] \hfill \\
        We can prove by induction that in the region $\abs{\xi} \geq 1/ \epsilon$
        \begin{align*}
            D^\alpha_x D^\beta_\xi b = a^{-1 - \abs{\alpha} - \abs{\beta}} G_{\alpha \beta}
        \end{align*}
        for all multi-index $\alpha, \beta$, where $G_{\alpha \beta}$ is a symbol of order $(\abs{\alpha} + \abs{\beta})m  - \abs{\beta}$. Therefore, using the ellipticity estimate for $a$, we get 
        \begin{align*}
            \norm[b]_{k, -m} 
            & = \sup_{(x, \xi) \in \mathrm{Int}(\Omega) \times \R^n} \frac{\abs{D^\alpha_x D^\beta_\xi b(x, \xi)}}{\sym[\xi]^{-m-k}} \\
            &= \sup_{\abs{\xi} \geq 1/\epsilon} \abs{a^{-1 - \abs{\alpha} - \abs{\beta}} G_{\alpha \beta}} \sym[\xi]^{m + k} \\
            &\leq \frac{\norm[G_{\alpha \beta}]_{0, (\abs{\alpha} + \abs{\beta})m - \abs{\beta}}}{\epsilon} \sup_{\abs{\xi} \geq 1/\epsilon^{1 + \abs{\alpha} + \abs{\beta}}} \sym[\xi]^{-m(1 + \abs{\alpha} + \abs{\beta})} \sym[\xi]^{m + k}\\
            & < \infty
        \end{align*}
        as required. 
        
        \item[$b$ is an inverse of $a$ modulo $ S^{-\infty}_\infty(\Omega; \R^n)$. ] \hfill \\
        The main observation is that the set where $b$ fails to be the multiplicative inverse of $a$ is a compact set (in $\xi$) and thus $a \cdot b - 1$ is in fact a compactly supported smooth function of $\xi$ which is a subset of $S^{-\infty}_\infty(\Omega; \R^n)$. \\
        \\
        Explicitly, for any $N \in \N$
        \begin{align*}
            \sup_{(x, \xi) \in \mathrm{Int}(\Omega) \times \R^n} \frac{\abs{D^\alpha_x D^\beta_\xi (a\cdot b - 1)} }{\sym[\xi]^{-N}}
            &\leq \sup_{\abs{\xi} \leq 1/ \epsilon} \sym[\xi]^N \abs{D^\alpha_x D^\beta_\xi (\phi(\xi \epsilon / 2))} < \infty.  
        \end{align*}
        
    \end{description}
\end{proof}



%%%%%%%%%%%%%%%%%%%%%%%%%%%%%%%%%%%
%%%%%%%%%%%%%%%%%%%%%%%%%%%%%%%%%%%
\section{Pseudodifferential operators}

As noted in (\ref{}), pseudodifferential operators are obtained via symbols via the quantisation procedure
\begin{align}
    S^{m}_\infty(\R^{2n}; \R^n) \ni a \mapsto I(a) = \frac{1}{(2\pi)^n} \int e^{i(x- y)\xi} a(x, y, \xi) \d[\xi]
\end{align}
with action on Schwartz functions $u \in \sch(\R^n)$ given by 
\begin{align} \label{eq: oscillatory integral}
    I(a)(u) =  \frac{1}{(2\pi)^n} \int e^{i(x- y)\xi} a(x, y, \xi) u(y) \d[y] \d[\xi]. 
\end{align}
Unfortunately, unless $m < -n$, we have no guarantee that the integral \ref{eq: oscillatory integral} above is convergent. However, it can be interpreted as a tempered distribution, with action on a Schwartz function$ v \in \sch(\R^n)$ given by 
\begin{align}
    I(a)(u)(v) = \frac{1}{(2\pi)^n} \int e^{i(x- y)\xi} a(x, y, \xi) u(y) v(x) \d[y] \d[\xi] \d[x] \in \C.  
\end{align}
Hence, our immediate concern is to ensure that this quantisation procedure is well-defined. Explicitly, we want to show that for each $a \in S^{m}_\infty(\R^{2n}; \R^n)$
\begin{align*}
    I(a) : \sch(\R^n) & \to \sch'(\R^n) \\
    u &\mapsto I(a)(u) 
\end{align*}
is a continuous linear map between Frechet spaces. 

\todo{Reconsider the appropriateness of this remark and the abuse of writing $I(a) = Op(a)$}
\begin{rem}
    Given $a \in S^{m}_\infty(\R^{2n}; \R^n)$, we sometimes write $A = Op(a) = I(a)$ for the operator $\sch(\R^n) \to \sch'(\R^n)$ defined by quantising the symbol $a$. Also, once the procedure above is proven to be well-defined, we will, with abuse of notation, identify the integral
    \begin{align*}
        I(a) = \frac{1}{(2\pi)^n} \int e^{i(x- y)\xi} a(x, y, \xi) \d[\xi] \in \sch'(\R^n\times \R^n)
    \end{align*}
    to be the \textit{Schwartz Kernel} of the operator $I(a): \sch(\R^n) \to \sch'(\R^n)$. 
\end{rem}
\todo{Insert discussion about the usage of Schwartz kernal theorem here. }

We will first establish the case for $m < -n$ (write $m = -n - \epsilon$ for some $\epsilon > 0$). As mentioned, the oscillatory integral \ref{eq: oscillatory integral}, is absolutely convergent and continuity comes from the bound given by the following computation: $\forall M \in \N$, $\forall a \in S^{-n -\epsilon}_\infty(\R^{2n}; \R^n)$, $\forall \varphi \in \sch(\R^n)$ 
\begin{align*}
    \abs{I(a)(\varphi)}
    & \leq \frac{1}{(2\pi)^n} \int \abs{a(x, y, \xi) \varphi(x, y)} \d[\xi] \d[x] \d[y]\\
    & \leq \frac{\norm[a]_{0, -n -\epsilon}}{(2\pi)^n} \int \sym[\xi]^{-n -\epsilon} \sym[(x, y)]^{-M} \sym[(x, y)]^{M} \abs{\varphi(x, y)} \d[\xi] \d[x] \d[y]\\
    & \leq \frac{\norm[a]_{0, -n -\epsilon} \norm[\varphi]_M}{(2\pi)^n} \int \sym[\xi]^{-n -\epsilon} \sym[(x, y)]^{-M}  \d[\xi] \d[x] \d[y]\\
\end{align*}
for any $M \in \N$, where 
\todo{Remember to make consistent choice for schwartz norm}
\begin{align} \label{eq: Schwartz seminorm} 
    \norm[\varphi]_M :=  \sum_{\abs{\alpha} \leq M}  \sup_{(x, y) \in \R^{2n}} \sym[(x, y)]^M \abs{D^{\alpha}_{x, y} \varphi(x, y)}
\end{align}
is the Schwartz  seminorm on $\sch(\R^{2n})$. If we choose $M \geq 2n +1$, the $x, y$ integrals are convergent and since $m = -n -\epsilon < -n$, the $\xi$ integral converges as well, hence we have 
\begin{align*}
    \abs{I(a)(\varphi)} \leq C \norm[a]_{0, m} \norm[\varphi]_M
\end{align*}
and hence $\norm[I(a)]_{\sch'} \leq C \abs{a}_{0, m}$ which is the continuity statement for linear maps $S^{m}_{\infty}(\R^{2n}; \R^{n}) \to \sch'(\R^n)$. 

The proposition below extend this result to general $m \in \R$. 
\begin{fprop}
    The continuous linear map 
    \begin{align*}
        I : S^{-\infty}_\infty(\R^{2n}; \R^n) \to \sch'(\R^{2n}) \\
    \end{align*}
    extends uniquely to a linear map 
    \begin{align*}
        I: S^{m}_\infty(\R^{2n}; \R^n) \to \sch'(\R^{2n})
    \end{align*}
    which is continuous as linear map from $S^{m'}_\infty(\R^{2n}; \R^n)$ to $\sch'(\R^{2n})$ for arbitrary $m \in \R$ and $m ' > m$. 
\end{fprop}
\begin{proof}
    Let $m, m' \in \R$, $n \in \N$ with $m < m' $ be given. For any $a \in S^{m}_\infty(\R^{2n}; \R^n)$, the density of $S^{-\infty}_\infty(\R^{2n}; \R^n)$ in  $S^{m}_\infty(\R^{2n}; \R^n)$ with the topology of $S^{m'}_\infty(\R^{2n}; \R^n)$ means that there exist a sequence $a_k \in S^{-\infty}_\infty(\R^{2n}; \R^n)$ so that $a_k \to a \in S^{m'}_\infty(\R^{2n}; \R^n)$. Together with the completeness of $\sch'(\R^{2n})$ (being a continuous linear map into $\C$ which is complete), we have unique continuous linear extension of $I : S^{-\infty}_\infty(\R^{2n}; \R^n) \to \sch'(\R^{2n})$ to $S^{m}_\infty(\R^{2n}; \R^n)$ given by
    \begin{align*}
        I(a) := \lim_{k \to \infty}I(a_k)
    \end{align*}
    which is continuous in the $S^{m'}_\infty(\R^{2n}; \R^n)$ topology. Therefore, it is enough to show that for any $a \in S^{-\infty}_\infty(\R^{2n}; \R^n) $ and $\varphi \in \sch(\R^{2n})$, there exist $N, M \in \N$, such that 
    \begin{align*}
        \abs{I(a)(\varphi)} \leq C \norm[a]_{N, m'} \norm[\varphi]_M. 
    \end{align*}
    
    Let $a \in S^{-\infty}_{\infty}(\R^{2n}; \R^{n})$, $\varphi \in \sch(\R^n)$ as above be given. It can be shown by induction that for any $N \in \N$,  
    \begin{align*}
    \brac{1 + \Delta_y}^N e^{i (x - y) \cdot \xi} = \sym[\xi]^{N}e^{i (x - y) \cdot \xi}
    \end{align*}
    where $\Delta_y = - \sum_{j = 1}^n \p_{y_j}^2$ is the Laplacian on $\R^n$. 

    With this, we can use integratusing integration by parts to introduce extra $\xi$-decay in the integral. Explicitly, for any $N \in \N$, 
    \begin{align*}
        I(a)(\varphi) 
        & = \frac{1}{(2\pi)^n} \int e^{i(x- y)\xi} a(x, y, \xi) \varphi(x, y) \d[\xi] \d[x] \d[y]\\
        & = \frac{1}{(2\pi)^n} \int \sym[\xi]^{-2N} (1 + \Delta_y)^N a(x, y, \xi) \varphi(x, y) \d[\xi] \d[x] \d[y]\\
        &=  \frac{1}{(2\pi)^n} \int \sym[\xi]^{-2N} e^{i (x - y) \xi } \brac{ \sum_{\abs{\mu} + \abs{\nu} \leq 2N} C_{\mu, \nu}  D^\mu_y a(x, y, \xi) D^\nu_{y} \varphi(x, y)} \d[\xi] \d[x] \d[y]\\
    \end{align*}
    where $C_{\mu, \nu}$ is a complex constant, independent of $a$, $\varphi$, involving only the binomial coefficient. Now, note that using the $2N^{th}$ symbol seminorm in $S^{m'}_{\infty}(\R^{2n}; \R^{n})$ we have the bound
    \begin{align*}
    \abs{D^\mu_y a(x, y, \xi)} 
    &= \sym[\xi]^{m' } \frac{\abs{D^\mu_y a(x, y, \xi)}}{\sym[\xi]^{m'}} \\
    & \leq \sym[\xi]^{m'} \sup_{(x, \xi) \in \R^{2n} \times \R^n} \max_{\abs{\mu} + \abs{\mu'} + \abs{\mu''} \leq 2N} \frac{\abs{D^{\mu'}_x D^\mu_y a(x, y, \xi)}}{\sym[\xi]^{m' - \abs{\mu''}}} \\
    & = \sym[\xi]^{m'} \norm[a]_{2N,m'}. 
    \end{align*}
    And using Schwartz seminorm, we have that for any $M \in \N$ greater than $N$,  
    \begin{align*}
    \abs{D^\nu_y \varphi(x, y)} 
    &= \sym[(x, y)]^{- M} \sym[(x, y)]^{M} \abs{D^{\nu}_{y} \varphi(x,y)} \\
    & \leq  \sym[(x, y)]^{- M} \max_{\nu{\leq} \leq M} \sup_{(x, y) \in \R^{2n}} \sym[(x, y)]^{M} \abs{D^{\nu}_{y} \varphi(x,y)}\\
    & \leq \sym[(x, y)]^{-M} \norm[\varphi]_{M}. 
    \end{align*}

    Bring together both bounds, we have for all positive integers $M > N$, 
    \begin{align*}
        \abs{I(a)(\varphi)} 
        & \leq  \frac{1}{(2\pi)^n} \sum_{\abs{\mu} + \abs{\nu} \leq 2N} C_{\mu, \nu}  \int \sym[\xi]^{-2N} \abs{ D^\mu_y a(x, y, \xi) D^\nu_{y} \varphi(x, y)} \d[\xi] \d[x] \d[y]\\
        & \leq  C' \norm[a]_{2N,m'} \norm[\varphi]_{M}  \int \sym[\xi]^{m ' -2N} \sym[(x, y)]^{-M}  \d[\xi] \d[x] \d[y]. \\
    \end{align*}
    Thus, as long as $m' - 2N < -n$, i.e. $N > \max\brac{\frac{m' + n}{2}, 0}$, the integral above converges and there exist $C > 0$ independent of $a$, $\varphi$ such that   
    \begin{align*}
        \abs{I(a)(\varphi)} \leq C \norm[a]_{2N, m'}\norm[\varphi]_{M} 
    \end{align*}
    which makes $I(a)$ a continuous linear functional on $\sch(\R^n)$. 
\end{proof}

By the Schwartz Kernel theorem (\cite{}), each $a \in S^{m}_\infty(\R^{2n}; \R^n)$ defines a continuous linear operator 
\begin{align*}
I(a) : \sch(\R^n) \to \sch'(\R^n)
\end{align*}
which is teh desired result. 


We can now define the space of $m$-order pseudodifferential operators to be the image of $S^{m}_{\infty}(\R^{2n}; \R^{n})$ under $I$. 

\begin{fdefinition}
    \begin{align*}
    \Psi^{m}_\infty(\R^n) := \set{A = I(a) \wh a \in S^{m}_\infty(\R^{2n}; \R^n)}
    \end{align*}
    with the total space $\Psi^{\infty}_\infty(\R^n) := \cup_{m \in \R} \Psi^m_\infty(\R^n)$ and the residual space $\Psi^{-\infty}_\infty(\R^n) := \cap_m \Psi^m_\infty(\R^n)$ are defined similarly to that of symbol spaces. 
\end{fdefinition} 

Now, we make the observation that
\begin{align*}
&D_{x_j} e^{i (x - y) \cdot \xi} = i \xi_{j} (-i) e^{i (x - y) \cdot \xi} = \xi_j e^{i (x - y) \cdot \xi}\\
& D_{\xi_j} e^{i (x - y) \cdot \xi} = i (x_j - y_j) (-i) e^{i (x - y) \cdot \xi} = (x_j - y_j) e^{i (x - y) \cdot \xi} 
\end{align*}
and thus, by induction and Lebinz formula, 
\begin{align*}
&D^\alpha_x e^{i (x - y) \cdot \xi} = \xi^\alpha e^{i (x - y) \cdot \xi} \\
& x^\beta e^{i (x - y) \cdot \xi} = (y - D_\xi)^\beta e^{i (x - y) \cdot \xi} = \sum_{\mu \leq \beta} \binom{\beta}{\mu} y^\mu D^{\beta - \mu}_\xi e^{i (x - y) \cdot \xi}. 
\end{align*}
Together with integration by parts, this shows that polynomial multiplication and derivative operations $x^\beta D^\alpha_x$ on $I(a)u, a \in S^{m}_{\infty}(\R^{2n}; \R^{n}), u \in \sch(\R^n)$ can be transform into operations involving only $y$ and $\xi$ variables, namely the integration variables in $I(a)u$. This suggest the following sharper result. 
\begin{fprop}
    Let $A \in \Psi^{m}_{\infty}(\R^n)$ with Schwartz kernel $I(a)$, $a \in S^{m}_{\infty}(\R^{2n}; \R^{n})$, then, 
    \begin{align*}
    A : \sch(\R^n) \to \sch(\R^n)
    \end{align*}
    is a continuous linear map. 
\end{fprop}
\begin{proof}
    From the  proof of (\ref{}), for sufficiently large $N \in \N$, we have that 
    \begin{align*}
     I(a)\varphi(x)    
     &=  \sum_{\abs{\mu} + \abs{\nu} \leq 2N} C_{\mu, \nu} \int \sym[\xi]^{-2N} e^{i (x - y) \xi }  D^\mu_y a(x, y, \xi) D^\nu_{y} \varphi(y) \d[\xi] \d[y]\\
    \end{align*}
    is an absolutely convergent integral for any $a \in S^{m}_{\infty}(\R^{2n}; \R^{n})$ and $\varphi \in \sch(\R^n)$. Thus, we can differentiate under the integral sign and apply (\ref{}) to get
    \begin{align*}
    D^\alpha_x I(a)\varphi(x) 
    & = \sum_{\abs{\mu} + \abs{\nu} \leq 2N} C_{\mu, \nu} \int \sym[\xi]^{-2N}D^\alpha_x \brac{e^{i (x - y) \xi }  D^\mu_y a(x, y, \xi)} D^\nu_{y} \varphi(y) \d[\xi] \d[y]\\
    & = \sum_{\substack{\gamma \leq \alpha \\ \abs{\mu} + \abs{\nu} \leq 2N}} C_{\mu, \nu, \gamma} \int \sym[\xi]^{-2N}\xi^{\alpha - \gamma} e^{i (x - y) \xi }  D^{\gamma}_x D^\mu_y a(x, y, \xi)  D^\nu_{y} \varphi(y) \d[\xi] \d[y]. 
    \end{align*}
    for any multi-index $\alpha \in \N^n$. Similarly, for multiplication by $x^\beta$, $\beta \in \N^n$, we can use  (\ref{}). That, together with integration by parts in $\xi$ gives
    \begin{align*}
    & x^\beta D^\alpha_x I(a)\varphi(x) \\
    & = \sum_{\substack{\gamma \leq \alpha, \, \lambda \leq \beta \\ \abs{\mu} + \abs{\nu} \leq 2N}} C_{\mu, \nu, \gamma, \lambda} \int \sym[\xi]^{-2N}\xi^{\alpha - \gamma} y^\lambda \brac{D^{\beta - \lambda}_\xi e^{i (x - y) \xi }}  D^{\gamma}_x D^\mu_y a(x, y, \xi)  D^\nu_{y} \varphi(y) \d[\xi] \d[y]. 
    \end{align*}
    Thus, similar to the proof of (\ref{}), 
    \begin{align*}
    & \abs{x^\beta D^\alpha_x I(a)\varphi(x)} \\
    & \leq \sum_{\substack{\gamma \leq \alpha, \, \lambda \leq \beta \\ \abs{\mu} + \abs{\nu} \leq 2N}} C_{\mu, \nu, \gamma, \lambda} \int \sym[\xi]^{-2N} \abs{\xi^{\alpha - \gamma}} \abs{y^\lambda} \abs{D^{\gamma}_x D^\mu_y a(x, y, \xi)} \abs{ D^\nu_{y} \varphi(y)} \d[\xi] \d[y] \\
    & \leq \norm[a]_{N,m} \norm[\varphi]_M \sum_{\substack{\gamma \leq \alpha, \, \lambda \leq \beta \\ \abs{\mu} + \abs{\nu} \leq 2N}} C_{\mu, \nu, \gamma, \lambda} \int \sym[\xi]^{-2N + \abs{\alpha} + m} \sym[y]^{\abs{\beta} - M} \d[\xi] \d[y] \\
    & \leq C_{\alpha, \beta} \norm[a]_{N,m} \norm[\varphi]_M 
    \end{align*}
    where $N, M \in \N$ are chosen so that 
    \begin{align*}
    &N > \frac{m + \abs{\alpha} + n}{2} \\
    &M > n + \abs{\beta}. 
    \end{align*}
    Hence, the (equivalent) Schwartz seminorm of $I(a)(\varphi)$ is bounded and hence $I(a)(\varphi) \in \sch(\R^n)$ as required. 
 
\end{proof}


\subsection{Adjoint} 

Now, we have shown that every pseudodifferential operator $A \in \Psi^{m}_{\infty}(\R^n)$ is an operator 
\begin{align*}
A: \sch(\R^n) \to \sch(\R^n). 
\end{align*}
Therefore, it has a Frechet space adjoint 
\begin{align*}
A^\dagger : \sch'(\R^n) \to \sch'(\R^n)
\end{align*}
defined by 
\begin{align*}
A^\dagger u(\varphi) = u(A\varphi)
\end{align*}
for all $u \in \sch'(\R^n)$, $\varphi \in \sch(\R^n)$. 

\begin{flemma} 
    Given the map
    \begin{align*}
    T: \R^{2n} \times \R &\to \R^{2n} \times \R \\
    (x, y, \xi) &\mapsto (y, x, -\xi)
    \end{align*}
    and a symbol $a S^{m}_{\infty}(\R^{2n}; \R^{n})$, adjoint of the operator $A$ with Schwartz kernel $I(a)$ is uniquely given by the operator whose symbol correspond to the pullback of $a$ under $T$, i.e. 
    \begin{align*}
    I(a)^\dagger = I(T^*a) : \sch'(\R^n) \to \sch'(\R^n)
    \end{align*}    
\end{flemma}
\begin{proof}
    Let $\iota : \sch(\R^n) \to \sch'(\R^n)$ be the continuous inclusion of Schwartz function into the space of tempered distribution given by $\iota(u) \varphi = \int u(x) \varphi(x) dx$ for all $u, \varphi \in \sch(\R^n)$. \\
    
    Suppose first $a \in S^{-\infty}_{\infty}(\R^{2n}; \R^{n})$. We know from (\ref{}) that $I(a) : \sch(\R^n) \to \sch(\R^n)$, thus $\iota \circ I(a) : \sch(\R^n) \to \sch'(\R^n)$ is a continuous function given by \begin{align*}
    \iota(I(a)u) \varphi
    & = \int I(a)(u)(x) \varphi(x) dx \\
    & = \frac{1}{(2\pi)^n}  \int\int u(y) e^{i (x - y) \cdot \xi} a(x, y, \xi) \varphi(x) \d[\xi] \d[x] \d[y] \\
    & =  \frac{1}{(2\pi)^n} \int \int u(y) e^{i (y - x) \cdot \xi} a(x, y, - \xi) \varphi(x) \d[\xi] \d[x] \d[y] \\
    &= \int u(y) I(T^*a) \varphi(y) \d[y]\\
    & = \iota(u)(I(T^*a)\varphi)
    \end{align*}
    forall $u, \varphi \in \sch(\R^n)$. Using the density of the residual space in $S^{m}_{\infty}(\R^{2n}; \R^{n})$ with the topology of $S^{m'}_{\infty}(\R^{2n}; \R^{n})$, $m' > m$, the identity above holds for any $a \in S^{m}_{\infty}(\R^{2n}; \R^{n})$. Finally, by the weak-* density of $\sch(\R^n)$ in $\sch'(\R^n)$, $\iota \circ I(a)$ has a unique continuous linear extension to a map $S: \sch'(\R^n) \to \sch'(\R^n)$ satisfying
    \begin{align*}
    S(u)(\varphi) = u(I(a)\varphi) 
    \end{align*}
    for any tempered distribution $u \in \sch'(\R^n)$. The last expression is also given by the adjoint $I(a)^\dagger (u) \varphi$, i.e. $\iota \circ I(a)$ extends continuously and uniquely to its adjoint. Therefore, together with the result above, we have 
    \begin{align*}
    I(a)^\dagger = I(T^* a)
    \end{align*}
    as required. 
\end{proof}

\begin{rem}
    Since $I(T^*a)$ is a composition of continuous map and that $T^* T^* a = a$, we can conclude that any symbol $a \in S^{m}_{\infty}(\R^{2n}; \R^{n})$ defines a continuous function 
    \begin{align*}
    I(a) = I(T^* a)^\dagger : \sch'(\R^n) \to \sch'(\R^n). 
    \end{align*}
\end{rem}


A similar conclusion can be made if we instead use the $L^2$-based pairing, i.e. the inner product on the Hilbert space $L^2(\R^n)$, 
\begin{align*}
\inprod[f, g] := \int f(x) \overline{g(x)} \d[x]. 
\end{align*}
The corresponding Hilbert space adjoint, $T^*$ of an operator $T$ is then defined by 
\begin{align*}
\int Tf(x) \overline{g(x)} \d[x] = \int f(x) \overline{T^* g(x)} \d[x]
\end{align*}


\begin{flemma}
    Given the transposition map 
    \begin{align*}
    F : \R^{2n} \times \R &\to \R^{2n} \times \R \\
    (x, y, \xi) &\mapsto (y, x, \xi)
    \end{align*}
    and a symbol $a S^{m}_{\infty}(\R^{2n}; \R^{n})$, $L^2$-adjoint of the operator $A$ with Schwartz kernel $I(a)$ is uniquely given by the operator whose symbol correspond to the complex conjugate of the pullback of $a$ under $F$, i.e. 
    \begin{align*}
    I(a)^* = I(\overline{F^*a})
    \end{align*}    
\end{flemma}
\begin{proof}
    The proof is similar to that of (\ref{}) with the computation replaced by 
    \begin{align*}
    \int I(a)u(x) \overline{\varphi(x)} \d[x] 
    & = \frac{1}{(2\pi)^n} \int u(y) \overline{\int e^{i (y - x) \cdot \xi} \overline{a(x, y, \xi)} \varphi(x) \d[x] \d[\xi] } \d[y]\\
    & = \int u(y) \overline{I(\overline{F^*a})\varphi(y)} \d[y]
    \end{align*}
\end{proof}


\subsection{Composition theorem}

In this section we shall prove that, just like symbol spaces, $\Psi^\infty_\infty(\R^n)$ forms a graded *-algebra. The difference being, this time, the algebra is \emph{non-commutative}. 

\begin{ftheorem}[Summary] 
    Given $n \in \N$, $\Psi^\infty_\infty(\R^n)$ is a graded *-algebra over $\C$ with continuous inclusion 
    \begin{align*}
        \iota : \Psi^m_\infty(\R^n) \to \Psi^{m'}_\infty(\R^n)
    \end{align*}
    for any $m \leq m'$. 
\end{ftheorem}

We shall prove this theorem by first accumulate several technical lemmas. Among them, the most important and useful result is the reduction lemma \ref{}, which arise from the observation that for any symbol $a = a(x, y, \xi) \in S^{m}_\infty(\R^{2n}; \R^n)$, there exist a unique symbol $a_L = a_L(x, \xi) \in S^{m}_\infty(\R^{n}; \R^n)$ without $y$ dependence, that quantise to the same operator, i.e.  $I(a) = I(a_L)$. In fact, for any $t \in [0, 1]$, there is a unique $a_t = a_t((1 - t)x + ty, \xi)$ that quantise to the same operator. This shows that  $I : S^{m}_\infty(\R^{2n}; \R^n) \to \Psi^{m}_{\infty}(\R^n)$ is highly non-injective. The reduction lemma allow us to construct an injective quantisation procedure
\begin{align*}
q_L : S^{m}_\infty(\R^{n}; \R^n) \to \Psi^{m}_{\infty}(\R^n). 
\end{align*}


\subsection{Asymptotic Summation}
Suppose we are given a sequence of symbols with decreasing order, $a_j \in S^{m-j}_\infty(\R^{p}; \R^n)$, $j \in \N$, we know that  $a_j(x, \xi)$ has ever higher rate of decay for large $\abs{\xi}$ with increasing $j$. However, the series $\sum_{j \in \N} a_j(x, \xi)$ need not converge. However, we have the following notion of asymptotic convergence. 
\begin{fdefinition}[Asymptotic summation] 
    A sequence of symbols with decreasing order, $a_j \in S^{m-j}_\infty(\R^{p}; \R^n)$, $j \in \N$ is \textbf{asymptotically summable} if there exist $a \in S^{m}_\infty(\R^{p}; \R^n)$ such that for all $N \in \N$, 
    \begin{align*}
        a - \sum_{j = 0}^{N -1} a_j \in S^{m-N}_\infty(\R^{p}; \R^n). 
    \end{align*}
    We write
    \begin{align*}
        a \sim \sum_{j \in \N} a_j. 
    \end{align*}
\end{fdefinition}

\begin{flemma}
    Every sequence of symbols with decreasing order is asymptotically summable.  Furthermore, the sum is unique up to an additive term in $S^{-\infty}_\infty(\R^{p}; \R^n)$. 
\end{flemma}
\begin{proof}[Sketch]
    Let $a_j \in S^{m-j}_\infty(\R^{p}; \R^n)$, $j \in \N$ be given. For uniqueness, suppose $a, a' \in S^{m}_\infty(\R^{p}; \R^n)$ are both asymptotic sums of the sequence. We need to show that $a - a' \in S^{-\infty}_\infty(\R^{p}; \R^n)$. Indeed, for any $N \in \N$, 
    \begin{align*}
        a - a' = \brac{a - \sum_{j = 0}^{N -1} a_j } - \brac{a' - \sum_{j = 0}^{N -1} a_j } \in S^{m -N}_\infty(\R^{p}; \R^n)
    \end{align*}
    since both terms on the right are elements of $S^{m -N}_\infty(\R^{p}; \R^n)$. Thus, 
    $$a - a' \in \cap_{n \in \N} S^{m -N}_\infty(\R^{p}; \R^n) = S^{-\infty}_\infty(\R^{p}; \R^n).$$ 
    \\
    For existence, we construct $a S^{m}_\infty(\R^{p}; \R^n)$ by Borel's method \cite{}. Let $\chi \in C^{\infty}_c(\R^p)$ be a bump function and define
    \begin{align*}
        a = \sum_{j \in \N} \brac{1 - \chi}(\epsilon_j \xi) a_j(x, \xi)
    \end{align*}
    where $\R_{> 0} \ni \epsilon_j \to 0 $ is a strictly monotonic decreasing sequence that converges to $0$. We note that the sequence is a finite sum for any input $(x, \xi)$ and hence define a smooth function. It remains to show that, for some choice of $\epsilon_j$ with sufficiently rapid decay, 
    \begin{align*}
        \sum_{j \geq N} \brac{1 - \chi}(\epsilon_j \xi) a_j(x, \xi) 
    \end{align*}
    converges in $S^{m -N}_\infty(\R^{p}; \R^n)$ for any $N \in \N$. 
    \\
    
    Note: This is again an exercise in using symbol seminorms and Leibniz formula.  
\end{proof}


\todo{about smoothing operators}
\begin{fprop}
    Elements of the residual operator space are exactly smoothing operators. Explicitly, a pseudodifferential operator $A : \sch(\R^n) \to \sch'(\R^n)$ is an element of $\Psi^{-\infty}_{\infty}(\R^n)$ if and only if there exist $c \in S^{-\infty}_{\infty}(\R^{n}; \R^{n})$ such that $A = I(c)$. 
\end{fprop}


\subsection{Reduction}
We will now show that $\Psi^{m}_\infty(\R^n)$ is exactly the range of $I : S^{m}_\infty(\R^{2n}; \R^n) \to \sch'(\R^{2n})$ restricted to $S^{m}_\infty(\R^{n}; \R^n) \subset S^{m}_\infty(\R^{2n}; \R^n)$. 
\begin{fdefinition}
    Let 
    \begin{align*}
        \pi_L : \R^{3n}_{x, y, \xi} \to \R^{2n}_{x, \xi}
    \end{align*}
    be the projection map $(x, y, \xi) \mapsto (x, \xi)$. We define the \textbf{left quantisation map} as 
    \begin{align*}
        q_L := I \circ \pi_L^* : S^{m}_\infty(\R^{n}; \R^n) \to \Psi^{m}_\infty(\R^{n}) 
    \end{align*}
    with elements $a_L \in S^{m}_\infty(\R^{n}; \R^n)$ known as the \textbf{left reduced symbols}.
    
\end{fdefinition}

\begin{flemma}[Reduction] 
    For any $a(x, y, \xi) \in S^{m}_\infty(\R^{2n}_{x, y}; \R^n_\xi)$ there exist unique $a_L(x, \xi) \in S^{m}_\infty(\R^{n}; \R^n)$ such that $I(a) = q_L(a_L) = I(a_L \circ \pi_L)$. Furthermore, with $\iota : \R^{2n} \ni (x, \xi) \mapsto (x, x, \xi) \in \R^{3n}$ being the diagonal inclusion map, we have 
    \begin{align}
        a_L(x, \xi) \sim \sum_{\alpha} \frac{i^{\abs{\alpha}}}{\alpha !} \iota^* D^\alpha_y D^\alpha_\xi a(x, y, \xi). 
    \end{align}
\end{flemma}
\begin{proof}[Sketch] 
    Note that 
    \begin{align*}
        D_\xi^\alpha e^{i(x- y) \xi} = (x - y)^\alpha e^{i(x -y) \cdot \xi} \implies I((x -y)^\alpha a) = I((-1)^{\abs{\alpha}D_\xi^\alpha a})
    \end{align*}
    where we have extended the identity that is true using integration by parts in $S^{-\infty}_\infty(\R^{2n}; \R^n)$ to general $S^{m}_\infty(\R^{2n}; \R^n)$ using the density result of symbol space. Now, if we Taylor expand $a$ around the diagonal $x =y$, we get 
    \begin{align*}
        a(x, y, \xi) = \sum_{\abs{\alpha} \leq N -1} \frac{(-i)^{\abs{\alpha}}}{\alpha !} (x - y)^\alpha D^\alpha_ya (x, x, \xi) + r_N(x, y, \xi)
    \end{align*}
    where 
    \begin{align*}
        r_N(x, y, \xi) = \sum_{\abs{\alpha} = N} \frac{(-i)^{\abs{\alpha}}}{\alpha !} (x - y)^\alpha \int_0^1 (1 - t)^{N -1} D^\alpha_ya(x, (1 -t)x + ty, \xi) \d[t]
    \end{align*}
    for any $N \in \N$.  Applying the identity above give us 
    \begin{align*}
        &I(a) = \sum_{j = 0}^{N -1} A_j + R_N \\
        &A_j = I\brac{\sum_{\abs{\alpha} = j} \frac{i^{\abs{\alpha}}}{\alpha !} D^\alpha_y D^\alpha_\xi a (x, x, \xi)} \in \Psi^{m - j}_\infty(\R^n)\\
        &R_N \in \Psi^{m -N}_\infty(\R^n)
    \end{align*}
    Asymptotic summation lemma give us 
    \begin{align*}
        b(x, \xi) \sim \sum_{\alpha} \frac{i^{\abs{\alpha}}}{\alpha !} D^\alpha_y D^\alpha_\xi a (x, x, \xi) \in S^{m}_\infty(\R^{n}; \R^n)
    \end{align*}
    so that $I(a) - I(b) \in \Psi^{-\infty}_\infty(\R^n)$. It remains to show that $A \in \Psi^{-\infty}_\infty(\R^n) \iff A = I(c), c \in S^{-\infty}_\infty(\R^{n}; \R^n)$. 
\end{proof}



\begin{ftheorem}[Composition]
    Let $A \in \Psi^{m}_\infty(\R^n)$, $B \in \Psi^{m'}_\infty(\R^n)$ for some $m, m' \in \R$. Then, 
    \begin{enumerate}
        \item $A^* \in \Psi^{m}_\infty(\R^n)$.
        \item $A \circ B \in \Psi^{m + m'}_\infty(\R^n)$. 
    \end{enumerate}
\end{ftheorem}
\begin{proof}[Sketch]
    Let $A \in \Psi^{m}_\infty(\R^n)$, $B \in \Psi^{m'}_\infty(\R^n)$ for some $m, m' \in \R$ be given. 
    Since $A : \sch(\R^n) \to \sch(\R^n)$ (\ref{}), we have the adjoint operator $A^* : \sch'(\R^n) \to \sch'(\R^n)$ defined by 
    \begin{align*}
        A^*u(\varphi) = u(\overline{A\varphi}), \quad u \in \sch'(\R^n), \varphi \in \sch(\R^n). 
    \end{align*}
    We check that $A^*u$ is indeed an element of $\sch'(\R^n)$ since it is the composition of the maps $u \in \sch'(\R^n)$ and $\sch(\R^n) \ni \varphi \mapsto \overline{A\varphi}$ which are both continuous.  Let $a \in S^{m}_\infty(\R^{2n}; \R^n)$ be such that $A = I(a)$. Observe that, 
    \begin{align*}
        \inprod[Au, \varphi]_{L^2} 
        & = \int Au(x) \overline{\varphi(x)}\d[x]  \\
        & = \int u(y) \overline{\int e^{i(x -y) \cdot \xi} \overline{a(x, y, \xi)} \varphi(x) \d[x] \d[\xi]} \d[y] \\
        & = \int u(y) \overline{I(b) \varphi(y)}\d[y]\\
        & = \inprod[u, A^*\varphi]_{L^2}
    \end{align*}
    where $b(x, y, \xi) = \overline{a(y, x, \xi)} \in S^{m}_\infty(\R^{2n}; \R^n)$. Thus, $A^* \in \Psi^{m}_\infty(\R^n)$. \\
    \\
    For composition, applying the reduction lemma twice to get $a_L \in S^{m}_\infty(\R^{n}; \R^n)$ and $b_L \in S^{m'}_\infty(\R^{n}; \R^n)$ so that 
    \begin{align*}
        &A \varphi(x) = \frac{1}{(2\pi)^n} \int e^{i(x -y) \cdot \xi} a(x, \xi) \varphi(y) \d[y] \d[\xi] \\
        &B^* \varphi(x) = \frac{1}{(2\pi)^n} \int e^{i(x -y) \cdot \xi} \overline{b(x, \xi) }\varphi(y) \d[y] \d[\xi]
    \end{align*}
    which shows that 
    \begin{align*}
        AB\varphi(x) = \frac{1}{(2\pi)^n} \int e^{i(x -y) \cdot \xi} a(x, \xi) b(y, \xi) \varphi(y) \d[y] \d[\xi]
    \end{align*}
    and thus $AB = I(a(x, \xi) b(y, \xi))$. Since $a(x, \xi)b(y, \xi) \in S^{m + m'}_\infty(\R^{2n}; \R^n)$, we have the result $AB \in \Psi^{m + m'}_\infty(\R^n)$ as required. 
    
\end{proof}



\section{Principal symbol}
The existence and uniqueness of the left or right reduced symbol $a_L, a_R$ of any pseudodifferential operator $A \in \Psi^{m}_{\infty}(\R^n)$ shows that the left and right quantisation maps
\begin{align*}
q_L, q_R : S^{m}_\infty(\R^{n}; \R^n) \to \Psi^{m}_{\infty}(\R^n) 
\end{align*}
are in fact topological isomorphisms. Therefore, we can define their inverse
\begin{align*}
\sigma_L, \sigma_R : \Psi^{m}_{\infty}(\R^n) \to S^{m}_\infty(\R^{n}; \R^n)
\end{align*}
which are called the left, resp. right \textit{full symbol map}. 

\begin{flemma} 
    \begin{align*}
    0 \to \Psi^{m - 1}_{\infty}(\R^n) \to \Psi^{m}_{\infty}(\R^n) \to S^{m}_{\infty}(\R^{2n}; \R^{n}) / S^{m - 1}_{\infty}(\R^{2n}; \R^{n}) \to 0 
    \end{align*}
\end{flemma}

\begin{fprop}
    \begin{align*}
    \Psi^{m}_{\infty}(\R^n) \cdot \Psi^{m'}_{\infty}(\R^n) \subset \Psi^{m + m'}_{\infty}(\R^n)
    \end{align*}
\end{fprop}

\section{$L^2$ and Sobolev boundedness} 
\begin{fprop}
    $A : L^2(\R^n) \to L^2(\R^n)$ is a bounded linear operator if $A \in \Psi^{0}_{\infty}(\R^n)$. 
\end{fprop}


\begin{flemma}[Square root construction]
    
\end{flemma}

\begin{fprop}
    $A: H^{s, r}(\R^n) \to H^{s - m, r}(\R^n)$
\end{fprop}
\end{document}