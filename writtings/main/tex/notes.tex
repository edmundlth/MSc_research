\documentclass{article}
\usepackage{../thesis_style}

\title{}
\date{}

\begin{document}

%%%%%%%%%%%%%%%%%%%%%%%%%%%%%%%%%%%%%%%%%%%%%%%%%%%
\section{Microlocal Analysis}
\begin{ftheorem}[Schwartz Kernel Theorem {\cite[Chapter 4.6, p.~345]{taylor_pde}}] Let $M, N$ be compact manifold and 
\begin{align*}
T: C^\infty(M) \to \D'(N) \\
\end{align*}
be a continuous linear map ($C^\infty(M)$ being given Frechet space topology and $D'(N)$ the weak* topology). Define a bilinear map 
\begin{align*}
B: C^\infty(M) &\times C^\infty(N) \to \C \\
(u, v) &\mapsto B(u, v) = \inprod[v, Tu]. 
\end{align*}
Then, there exist a distribution $k \in \D'(M \times N)$ such that for all $(u, v) \in C^\infty(M) \times C^\infty(N)$
\begin{align*}
B(u, v) = \inprod[u \otimes v, k]. 
\end{align*}
We call such $k$ the kernel of $T$. 
\end{ftheorem}



\subsection{Symbols}

\subsection{Pseudodifferential Operators ($\Psi$DO's)}


%%%%%%%%%%%%%%%%%%%%%%%%%%%%%%%%%%%%%%%%%%%%%%%%%%%
\section{Sobolev Spaces \cite[Chapter 4]{taylor_pde}}
\begin{fdefinition}[Sobolev Spaces] Let $p \in \R$ and $k \in \N$ be given. We define the $k^{\text{th}}$-order $L^p$-based Sobolev spaces on $\R^n$ as the Banach space
\begin{align*}
W^{p, k}(\R^n) = \set{u \in L^p(\R^n) \wh D^\alpha u \in L^p(\R^n), \, \, \abs{\alpha} \leq k }
\end{align*}
with the norm
\begin{align*}
\norm[u]_{W^{p, k}} = \norm[u]_{L^p} + \norm[D^k u]_{L^p}. 
\end{align*}
For $p = 2$, we have denote $H^k := W^{2, k}$ and note that result from Fourier analysis gives
\begin{align*}
H^k = \set{u \in L^2(\R^n) \wh \abrac{\xi}^k \hat{u} \in L^2(\R^n)}
\end{align*}
allowing us to extend the definition to each real order $s \in \R$, 
\begin{align*}
H^s = \set{u \in S'(\R^n) \wh \abrac{\xi}^s \hat{u} \in L^2(\R^n)}
\end{align*}
where $S'(\R^n)$ is the space of tempered distribution on $\R^n$. This forms a Hilbert space with inner product given by
\begin{align*}
\inprod[u, v]_{H^s} = \inprod[ \Lambda^su, \Lambda^sv]_{L^2}
\end{align*}
with $\Lambda^s: S'(\R^n) \to S'(\R^n)$ being the operator $\Lambda^su = \F^{-1}(\abrac{\xi}^s \hat{u})$. 
\end{fdefinition}



%%%%%%%%%%%%%%%%%%%%%%%%%%%%%%%%%%%%%%%%%%%%%%%%%%%
\section{Linear Elliptic equations \cite[Chapter 5]{taylor_pde}}
\begin{ftheorem}[Elliptic estimate] Let $\overline{\Omega}$ be a compact Riemanian manifold and $\Omega$ be its interior. Let $L = -\Delta + X$, where $\Delta$ is the Laplacian and $X$ and first order differential operator with smooth coefficient in $\overline{\Omega}$. Then, we have the estimate
\begin{align*}
\norm[u]_{H^{k +1}}^2 \leq C \brac{\norm[Lu]_{H^{k -1}}^2 + \norm[u]_{H^k}^2}
\end{align*}
for all $u \in H^{k + 1}(\Omega) \cap H^1_0(\Omega)$ and for some $C > 0$. 
\end{ftheorem}


%%%%%%%%%%%%%%%%%%%%%%%%%%%%%%%%%%%%%%%%%%%%%%%%%%%
\section{Compact and Fredholm Operators}






%%%%%%%%%%%%%%%%%%%%%%%%%%%%%%%%%%%%%%%%%%%%%%%%%%%
\section{Miscellaneous}
\begin{flemma}[Riez's inequality]
Let $X$ be a normed linear space. Given a non-dense subspace (or closed proper subspace) $Y \subset X$ and any $r \in (0, 1)$, then there exist $x \in X$ with $\norm[x] = 1$ such that 
\begin{align*}
\inf_{y \in Y} \norm[x - y] \geq r. 
\end{align*}


\begin{proof}\hfill \\
Let $x_0 \in \overline{Y}^c$ and $R = \inf_{y \in Y}\norm[y - x_0] > 0$. Given any $\epsilon > 0$ we can pick (by definition of $\inf$) a $y_0 \in Y$ such that 
\begin{align*}
\norm[y_0 - x_0] < R + \epsilon. 
\end{align*}
Take $\epsilon < R \frac{1 - r}{r}$ and define $x \in X$ to be
\begin{align*}
x = \frac{y_0 - x_0}{\norm[y_0 - x_0]}. 
\end{align*}
Observe that $\norm[x] = 1$ and 
\begin{align*}
\inf_{y \in Y} \norm[x - y] 
&= \frac{1}{\norm[x_0 - y_0]} \inf_{y \in Y} \norm[y_0 - x_0 - y \norm[x_0 - y_0] ] \\
&= \frac{1}{\norm[x_0 - y_0]} \inf_{y \in Y} \norm[x_0 - y] \tag*{since $\alpha y - y_0 \in Y$ for any scalar $\alpha$} \\
&\geq \frac{R}{R + \epsilon} \\
&\geq \frac{R}{R + R \frac{1 - r}{r}} \\
&= r
\end{align*}
as required. 
\end{proof}
\end{flemma}

Riez's lemma gives us a clear distinction between finite and infinite dimensional Banach spaces. 
\begin{mdframed}
\begin{cor}
The closed unit ball in a Banach Space $X$ is compact iff $X$ is finite dimensional. 
\end{cor}
\begin{proof}
Let $X$ be a Banach space and $\overline{B}$ be closed unit ball. 
\begin{case}[ $\Longleftarrow$ ] If $X$ is finite dimensional, it is isometrically isomorphic to $\R^n$ for some $n \in \N$, where, by Heine-Borel theorem, the closed unit ball is compact. 
\end{case}


\begin{case}[$\implies$] We will prove the contrapositive. Suppose, $X$ is infinite dimensional. Let $x_0 \in \p \overline{B}$ be an element in the unit sphere. For each $n \in \N$, we will use Riez Lemma to obtain a unit vector $x_n$ such that 
$$\inf_{y \in \mathrm{span}\set{x_0, \dots, x_{n - 1}}} \norm[x_n - y] \geq \frac{1}{2}. $$
It is clear that $\set{x_n \wh n \in \N}$ is a sequence of element in $\overline{B}$ that has no convergent subsequence. Therefore $\overline{B}$ is not compact. 
\end{case}
\end{proof}
\end{mdframed}




\bibliographystyle{apalike}
\bibliography{../main.bib}
\end{document}

