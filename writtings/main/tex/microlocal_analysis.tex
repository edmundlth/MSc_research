\documentclass[12pt]{article}
\usepackage{../thesis_style}

\title{Microlocal Analaysis}
\date{}

\begin{document}


%Note: 
%  - Symbols
%  - Pseudos
%  - Properties, theorems lemmas for Pseudos
%  - Sobolev spaces
%  - Singular supports, WF, Ellipticity
%  - Pseudo's on manifolds
%  
%%%%%%%%%%%%%%%%%%%%%%%%%%%%%%%%%%%%%%%%%%%%%%%%%%%

We shall follow the presentation given in  \cite{melrose intro to microlocal}. 

\section{Motivation for Pseudodifferential operators} 
\begin{itemize}
    \item Solving PDEs via Fourier transform. For example, in Euclidean space, $\R^n$, constant coefficient linear PDE
    \begin{align*}
    P(D) u = \sum_{\abs{\alpha} \leq n} c_\alpha D^{\alpha} u = f, \quad c_\alpha \in \R
    \end{align*}
    where $P \in \R[x]$, can solved by applying Fourier transform which gives a solution of the form
    \begin{align*}
    u(x) = \frac{1}{(2\pi)^n} \int e^{i(x - y)\cdot \xi} f(y) \frac{1}{P(\xi)} \d[y] \d[\xi]
    \end{align*}
    due to the observation that
    \begin{align*}
    \F P(D) u = P(\xi) \F u. 
    \end{align*}
    Moreover, for linear differential operators with smooth coefficients
    \begin{align*}
    P(x, D):  u \mapsto \sum_{\abs{\alpha} \leq n } a_\alpha D^\alpha u, \quad a_\alpha \in C^\infty(\R^n)
    \end{align*}
    we have
    \begin{align*}
    P(x, D) u = \frac{1}{(2\pi)^n} \int P(x, \xi) e^{i(x -y) \xi} u(y)\d[y] \d[\xi].
    \end{align*}
    We would like to generalise the above so that $P(x, \xi)$ are smooth functions satisfying certain uniform bounds, called \emph{symbols}, instead of just polynomials in $\xi$. This gives us a class of operators, called pseudodifferential operators, that acts as 
    \begin{align*}
    A_a u (x) =  \frac{1}{(2\pi)^n} \int a(x, \xi) e^{i(x -y) \xi} u(y)\d[y] \d[\xi]
    \end{align*}
    for each symbol $a$.
    
    \item  There isn't enough differential operators with smooth coefficient in the sense that elliptic differential operators are not, in general, invertible in this class. For example, the operator
    \begin{align*}
    u \mapsto (\Delta + 1) u 
    \end{align*}
    has inverse that acts as (using construction via Fourier transform shown above)
    \begin{align*}
    (\Delta+ 1)^{-1} f = \frac{1}{(2\pi)^n} \int \frac{1}{1 + \abs{\xi}^2} e^{i(x -y) \xi} f(y)\d[y] \d[\xi]
    \end{align*}
    which is a pseudodifferential operator with symbol $a(x, \xi) = (1 + \abs{\xi}^2)^{-1}$. 
    
    \item Motivation from quantum mechanics. The notion of ``quantisation" in quantum mechanics can be formalised as the map that sends a symbol $a$ (a smooth function that represent determistic observable in classical mechanics) to its corresponding pseudodifferential operator (i.e. the corresponding quantum observable)
    \begin{align*}
    A_a : \psi \mapsto \frac{1}{(2\pi)^n} \int \frac{1}{1 + \abs{\xi}^2} e^{i(x -y) \xi} \psi(y)\d[y] \d[\xi]
    \end{align*}
    that acts on the wavefunction $\psi$. 
    
    \item Used in the formulation and proof of Atiyah-Singer Index theorem. 
\end{itemize}

We shall define, on Euclidean space, the space of symbols, $S^m(\R^{2n}_{x, y}; \R^n_{\xi})$ and the corresponding space of pseudodifferential operators, $\Psi^m(\R^n)$ which acts on distributions via the Schwartz kernel given by the oscilliatory integral 
\begin{align*}
I(a) = \frac{1}{(2\pi)^n} \int e^{i(x- y)\xi} a(x, y, \xi) \d[\xi]. 
\end{align*}
We note that we have introduced an extra variable $y$ which will help in explicating the properties of pseudodifferential operators. However, the extra variable does not change the essence of the theory. 


\section{Symbols}
We shall here list the definition of the space of symbols of order $m \in \N$  in Euclidean space $\R^n$ that one encounter in the literature. The main motivation is again based on the property of linear differential operators of order $m \in \N$ with smooth coefficient that, after Fourier transform gives the polynomial of $\xi$ with smooth coefficient 
\begin{align*}
P(x, \xi) = \sum_{\abs{\alpha} \leq m} a_\alpha(x) \xi^\alpha. 
\end{align*}
It has the property that 
\begin{align*}
\abs{D^\alpha_x D^\beta_\xi P(x, \xi)} \leq C_{\alpha, \beta} \sym[\xi]^{m - \abs{\beta}} 
\end{align*}
i.e. $P(x, \xi)$ is smooth and decreases in order as $\xi \to \infty$ with successive $\xi$-derivative. 


\begin{fdefinition}
    The space $S^m_\infty(\R^p; \R^n)$ of order $m$ is the space of smooth functions $a \in C^\infty(\R^p \times \R^n)$ such that for all multi-index $\alpha \in \N^p, \beta \in \N^n$
    \begin{align*}
    \abs{D^\alpha_x D^\beta_\xi a (x, \xi)} \leq C_{\alpha, \beta} \sym[\xi]^{m - \abs{\beta}} 
    \end{align*}
    uniformly on $\R^p \times \R^n$. We can also defined the space of symbol, $S^m_\infty(\Omega; \R^n)$ on a set with non-empty interior $\Omega \subset \R^p$, $\Omega \subset \overline{\mathrm{Int}(\Omega)}$ such that the bound above is satisfied uniformly in $(x, \xi) \in \mathrm{Int}(\Omega) \times \R^n$. The subscript $\infty$ refers the uniform boundedness condition. Together with the family of seminorm (indexed by $N \in \N$) 
    \begin{align*}
    \norm[a]_{N, m} = \sup_{(x, \xi) \in \mathrm{Int}(\Omega) \times \R^n} \max_{\abs{\alpha} + \abs{\beta} \leq N} \frac{D^\alpha_x D^\beta_\xi a(x, \xi)}{\sym[\xi]^{m - \abs{\beta}}} 
    \end{align*}
    gives a Frechet topology to $S^m_\infty(\Omega; \R^n)$. \\
    
    Note: In defining pseudodifferential operators, we shall focus on the case where $p = 2n$.
\end{fdefinition}

\begin{fdefinition}
A \textbf{symbol} of type $S^{m, l_1, l_2}_{\delta, \delta'}$ where $m, l_1, l_2 \in \R$ and $\delta, \delta' \in [0, 1/2)$ is an element of $C^\infty(\R^n_x; \R^n_y; \R^n_\xi)$ satisfying 
\begin{align*}
\frac{\abs{D^\alpha_xD^\beta_yD^\gamma_\xi a(x, y, \xi)}}{\sym[\xi]^{m - \abs{\gamma}} \sym[x]^{l_1 - \abs{\alpha}} \sym[y]^{l_2 - \abs{\beta}} \sym[\xi]^{\delta \abs{(\alpha, \beta, \gamma)}} \sym[x, y]^{\delta' \abs{(\alpha, \beta, \gamma)}}} \leq C_{\alpha, \beta, \gamma}
\end{align*}
uniformly in $\R^{3n}$. Taking the supremum over $\R^{3n}$, we get a family of seminorms, indexed by $N \in \N$ defined by 
\begin{align*}
\norm[a]_{S^{m, l_1, l_2}_{\delta, \delta'}, N} := \sum_{\abs{(\alpha, \beta, \gamma)} \leq N} \inf C_{\alpha, \beta, \gamma}
\end{align*}
which gives $S^{m, l_1, l_2}_{\delta, \delta'}$ a Frechet topology. 
\end{fdefinition}

\begin{fdefinition}
    A (Kohn-Nirenberg) \textbf{symbol} of order $m \in \R$ on $T^*\R^n \cong \R^{2n}_{x, \xi}$ is a smooth function $a = a(x, \xi)$ satisfying
    \begin{align*}
    \forall \alpha, \beta \in \N^n, \exists C \in \R_{\geq 0} \st \abs{D_x^\alpha D_\xi^\beta a(x, \xi)} \leq C\inprod[\xi]^{m - \abs{\beta}}
    \end{align*}
    uniformly in $x$. The \textbf{space of symbol of order} $m$ on $T^*\R^n$
\end{fdefinition}

\begin{fdefinition}
Let $n \in \N$ be given. An \textbf{order function} $g \in C^\infty(\R^n; \R_{\geq 0})$ is a \textit{non-negative} function satisfying 
\begin{align*}
\forall \alpha \in N^n \exists C \in \R_{\geq 0} \st \p^\alpha g \leq C g
\end{align*}
uniformly on $\R^n$, i.e. $\p^\alpha g = O(g)$ uniformly on $\R^n$. \\

Given an order function $g$, a \textbf{symbol} of order $g$ is a smooth function $a = a(x, \xi) \in C^\infty(T^*\R^n)$ satisfying 
\begin{align*}
\forall \alpha, \beta \in \N^n \st \abs{D^\alpha_x D^\beta_\xi a(x, \xi)} \leq C g(\xi)
\end{align*}
uniformly in $x$.
\end{fdefinition}

\subsection{Properties of Symbols}

\begin{fprop}
    Let $p, n \in \N$ be given and $\Omega \subset \R^p$ such that $\Omega \subset \overline{\mathrm{Int}(\Omega)}$. If $m, m' \in \R$ such that $m \leq m'$, then $S^m_\infty(\Omega; \R^n) \subset S^{m'}_\infty(\R^)$. Furthermore, the inclusion map 
    \begin{align*}
    \iota: S^m_\infty(\Omega; \R^n) \to S^{m'}_\infty(\Omega; \R^n)
    \end{align*}
    is continuous. 
\end{fprop}
\begin{proof}
    Let the real numbers $m \leq m'$ be given. We note that for any $\xi \in \R^n$
    \[
     \sym[\xi]^m \leq 1 \cdot \sym[\xi]^{m'}
     \]
    and thus if $a \in S^{m}_\infty(\Omega; \R^n)$, we have that $\forall \alpha \in \N^p, \forall \beta \in \N^n$
    \[
    \abs{D^\alpha_x D^\beta_\xi a(x, \xi)} \leq C \sym[\xi]^{m - \abs{\beta}} \leq C \sym[\xi]^{m' - \abs{\beta}} 
    \]
    which show that $a \in S^{m'}_\infty(\Omega; \R^n)$ as well. \\
    
    To show that $\iota$ is a continuous inclusion, it suffices to show that 
    \[
    \norm[\iota(a)]_{N, m'} \leq C \norm[a]_{N, m}
    \]
    for any $a \in S^{m}_\infty(\Omega; \R^n)$ and $N \in \N$. Indeed, this bound holds since 
    \[
    \frac{D^\alpha_x D^\beta_\xi a(x, \xi)}{\sym[\xi]^{m' - \abs{\beta}}}  \leq \frac{D^\alpha_x D^\beta_\xi a(x, \xi)}{\sym[\xi]^{m - \abs{\beta}}}. 
    \]
\end{proof}

This inclusion property allow us to consider $S^{m}_\infty(\Omega; \R^n)$ as the filtration of the space 
\[
S^\infty_\infty(\Omega; \R^n) = \bigcup_{m \in \R} S^{m}_\infty(\Omega; \R^n)
\]
and we shall denote the \emph{residual} space of the filtration as 
\[
S^{-\infty}_\infty(\Omega; \R^n) = \bigcap_{m \in \R} S^{m}_\infty(\Omega; \R^n). 
\]

We have a rather technical result of the density of the residual space in $S^{m}_\infty(\Omega; \R^n)$. 
\begin{flemma}
    Given any $m \in \R$ and $a \in S^{m}_\infty(\Omega; \R^n)$, there exist a sequence in $S^{-\infty}_\infty(\Omega; \R^n)$ such that bounded in $S^{m}_\infty(\Omega; \R^n)$ and converges to $a$ in the topology of $S^{m + \epsilon}_\infty(\Omega; \R^n)$ for any $\epsilon \in \R_{> 0}$. In other words, for any $m \in \R$ and $\epsilon > 0$, $S^{-\infty}_\infty(\Omega; \R^n)$ is dense in $S^{m}_\infty(\Omega; \R^n)$ with the topology of $S^{m + \epsilon}_\infty(\Omega; \R^n)$.
\end{flemma}
\begin{proof}
    The main strategy in this proof is to approximate any symbol with the very same symbol but cut off by a compactly supported function. As such, the main reason we can't have density of $S^{-\infty}_\infty(\Omega; \R^n)$ in $S^m_\infty(\Omega; \R^n)$ is the same reason to the fact that Schwartz functions are not dense in the space of smooth bounded functions, in particular, $1 \in S^0_\infty(\Omega; \R^n)$ is not in the closure of $S^{-\infty}_\infty(\Omega; \R^n)$. \\
    \\
    Now, let $a \in S^{m}_\infty(\Omega; \R^n)$ and $\epsilon \in \R_{>0}$ be given. Take anysmooth cut off functions supported in the unit ball, i.e. take $\phi \in C^\infty_c(\R^n)$ such that $0 \leq \phi \leq 1$ and $\phi(\xi) = 1 $ if $\abs{\xi} < 1$ and $\phi(\xi) = 0$ if $\abs{\xi)} > 2$. We define for each $k \in \N$
    \[
    a_k(x, \xi) = \phi\brac{\frac{\xi}{k}} a (x, \xi)
    \]
    and we check the following 
    \begin{enumerate}
        \item $a_k \in S^{-\infty}_\infty(\Omega; \R^n)$ for all $k \in \N$; 
        \item $a_k$ are bounded in $S^{m}_\infty(\Omega; \R^n)$ for all $k \in \N$; 
        \item $a_k \to a$ as $k \to \infty$ in $S^{m + \epsilon}_\infty(\Omega; \R^n)$. 
    \end{enumerate}
     Given arbitrary $N, k \in \N$, observe that 
     \[
     \abs{a_k} \leq C \sym[\xi]^{-N} 
     \]
     since $a_k$ is compactly supported in $\xi$ (as $\phi$ is compactly supported) and by Leibinz formula and symbol estimates on $a \in S^{m}_\infty(\Omega; \R^n)$
     \[
     \abs{D^\alpha_x D^\beta_\xi a_k(x, \xi)} 
     \leq \sum_{\mu \leq \beta} \binom{\beta}{\mu}k^{-\abs{\mu}} \brac{D^\mu \phi}\brac{\frac{\xi}{k}} \abs{D^\alpha_x D^{\beta - \mu}_\xi a(x, \xi) } 
     \leq C \sum_{\mu \leq \beta} \binom{\beta}{\mu}k^{-\abs{\mu}} \brac{D^\mu \phi}\brac{\frac{\xi}{k}} \sym[\xi]^{m - \abs{\beta - \mu}}. 
     \] 
     Since $\phi$ and all its derivatives are compactly supported, each term above is bounded in $\xi$ and thus $a_k$ is bounded in $S^{m}_\infty(\Omega; \R^n)$ and that 
     \[
     \abs{D^\alpha_x D^\beta_\xi a_k(x, \xi)} \leq C' \sym[\xi]^{-N}
     \]
     which allow us to conclude that $a_k \in S^{-\infty}_\infty(\Omega; \R^n)$.\\
     \\
     It remains to show that $\Lim[k] a_k = a $ in $S^{m + \epsilon}_\infty(\Omega; \R^n)$. In the first symbol norm, we observe that, using the symbol estimate for $a$ 
     \begin{align*}
     \norm[a_k - a]_{0, m + \epsilon} 
     & = \sup_{(x, \xi) \in \mathrm{Int}(\Omega) \times \R^n} \frac{\abs{a_k(x, \xi)}}{\sym[\xi]^{m + \epsilon}} \\
     & = \sup_{(x, \xi) \in \mathrm{Int}(\Omega) \times \R^n} \frac{\abs{(1- \phi(\xi/k))} \abs{a(x, \xi)}}{\sym[\xi]^{m + \epsilon}} \\
     & \leq \norm[a]_{0, m} \sup_{\xi \in \R^n} \frac{\abs{(1- \phi(\xi/k))}}{\sym[\xi]^{\epsilon}} \\
     & \leq \norm[a]_{0, m} \sym[k]^{- \epsilon}\\
     & \to 0
     \end{align*}
     as $k \to \infty$, since $\abs{( 1- \phi(\xi/ k))}$ is 0 in the region $\abs{\xi} \leq k$ and bounded by $1$ otherwise.  We remark upon the necessity of the extra decay given by $\sym[\xi]^{-\epsilon}$ factor. For other symbol norm we shall again use Leibinz formula to obtain 
     \begin{align*}
     \sup_{(x, \xi) \in \mathrm{Int}(\Omega) \times \R^n} \frac{\abs{D^\alpha_x D^\beta_\xi a_k(x, \xi)}}{\sym[\xi]^{m + \epsilon - \abs{\beta}}} 
     & \leq \sup_{(x, \xi) \in \mathrm{Int}(\Omega) \times \R^n} \frac{1}{\sym[\xi]^{m + \epsilon - \abs{\beta}}} \sum_{\mu \leq \beta} \binom{\beta}{\mu}k^{-\abs{\mu}} \brac{D^\mu (1 - \phi)}\brac{\frac{\xi}{k}} \abs{D^\alpha_x D^{\beta - \mu}_\xi a(x, \xi) } \\
     & \leq  \sup_{(x, \xi) \in \mathrm{Int}(\Omega) \times \R^n} \frac{C}{\sym[\xi]^{m + \epsilon - \abs{\beta}}} \sum_{\mu \leq \beta} \binom{\beta}{\mu}k^{-\abs{\mu}} \brac{D^\mu (1 - \phi)}\brac{\frac{\xi}{k}} \sym[\xi]^{m - \abs{\beta - \mu}}  \\
     & = C \sup_{\xi \in \R^n} \sum_{\mu \leq \beta} \binom{\beta}{\mu}k^{-\abs{\mu}} \brac{D^\mu (1 - \phi)}\brac{\frac{\xi}{k}} \sym[\xi]^{- \epsilon - \abs{\mu}}  \\
     & \leq C' k^{- \epsilon} \\
     & \to 0
     \end{align*}
     as $k \to \infty$ by the same argument as before. Thus, we have proven that $a_k \to a$ as $k \to \infty$ in $S^{m + \epsilon}_\infty(\Omega; \R^n)$. 
    
\end{proof}



\begin{fprop}
    Let $p, n \in \N$ be given. Let $\Omega \subset \R^p$ be such that $\Omega \subset \overline{\mathrm{Int}(\Omega)}$. Then, for any $m, m' \in \R$, we have 
    \[
    S^{m}_\infty(\Omega; \R^n) \cdot S^{m'}_\infty(\Omega; \R^n) = S^{m + m'}_\infty(\Omega; \R^n)
    \]
\end{fprop}
\begin{proof}
    Let $a \in S^{m}_\infty(\Omega; \R^n)$ and $b \in S^{m'}_\infty(\Omega; \R^n)$ be given. By (general) Leibinz formula, we have that for all multi-index $\alpha, \beta$, 
    \begin{align*}
    \sup_{(x, \xi) \in \mathrm{Int}(\Omega) \times \R^n} \frac{\abs{D^\alpha_x D^\beta_\xi a(x, \xi)b(x, \xi)}}{\sym[\xi]^{(m + m') - \abs{\beta}}} 
    &\leq  \sum_{\mu \leq \alpha, \gamma \leq \beta} \binom{\alpha}{\mu} \binom{\beta}{\gamma} \sup_{(x, \xi) \in \mathrm{Int}(\Omega) \times \R^n} \frac{\abs{D^\mu_x D^\gamma_\xi a(x, \xi)} \abs{D^{\alpha - \mu}_x D^{\beta - \gamma}_\xi b(x, \xi)}}{\sym[\xi]^{(m + m') - \abs{\beta}}} \\
    &\leq \sum_{\mu \leq \alpha, \gamma \leq \beta} \binom{\alpha}{\mu} \binom{\beta}{\gamma} C \sup_{\xi \in \R^n} \frac{\sym[\xi]^{m - \abs{\gamma}} \sym[\xi]^{m' - \abs{\beta - \gamma}}}{\sym[\xi]^{(m + m') - \abs{\beta}}} \\
    &= \sum_{\mu \leq \alpha, \gamma \leq \beta} \binom{\alpha}{\mu} \binom{\beta}{\gamma} C \sup_{\xi \in \R^n} \sym[\xi]^{\abs{\beta} - (\abs{\beta - \gamma} + \abs{\gamma})} \\
    & \leq \sum_{\mu \leq \alpha, \gamma \leq \beta} \binom{\alpha}{\mu} \binom{\beta}{\gamma} C \\
    &< \infty
    \end{align*}
    where we have use the property of multi-index that $\abs{\beta} = \abs{\beta - \mu} + \abs{\mu}$.  We have thus shown that $S^{m}_\infty(\Omega; \R^n) \cdot S^{m'}_\infty(\Omega; \R^n) \subset S^{m + m'}_\infty(\Omega; \R^n)$\\
    \\
    For the reverse inclusion, let $c \in S^{m + m'}_\infty(\Omega; \R^n)$ be given. Define 
    \begin{align*}
    a : (x, \xi) &\mapsto \sym[\xi]^m\\
    b: (x, \xi) &\mapsto \frac{c(x, \xi)}{a(x, \xi))}
    \end{align*}
    and observe that 
    \begin{itemize}
        \item $a \in S^{m}_\infty(\Omega; \R^n)$. It is clear that $a$ is smooth in both $x$ and $\xi$. It is independent of $x$ and thus any $x$ derivative gives 0. We need only to check that for all $\beta \in \N^n$, 
        \[
        \abs{D^\beta_\xi \sym[\xi]^m} \leq C \sym[\xi]^{m - \abs{\beta}}
        \]
        which can be proven by induction on $n$ and $\beta$. We shall only prove the base case where $n = 1$ and $\beta = 1$. We have 
        \begin{align*}
        \abs{D_\xi \sym[\xi]^m} = \abs{\p_\xi (1 + \xi^2)^{m/2}} = \abs{m\xi \sym[\xi]^{m - 2}} = \abs{m \frac{\xi}{\sym[\xi]}} \sym[\xi]^{m - 1} \leq \abs{m} \sym[\xi]^{m - 1}
        \end{align*}
        where we have used the fact that $\abs{\xi} \leq \sym[\xi]$ for all $\xi$. 
        \item $b \in S^{m'}_\infty(\Omega; \R^n)$. We note first that $\sym[\xi]^m \neq 0$ for all $\xi \in \R^n$ and thus $b$ is well-defined. Since division by $\sym[\xi]^m$ does not affect any of the $x$ derivative, we only need to show that for any $\beta \in \N^n$, we have
        \[
        \abs{D^\beta_\xi b(x, \xi)} \leq C \sym[\xi]^{m + m' - \abs{\beta}}
        \]
        for some constant $C > 0$ uniformly in $\xi$. Indeed, observe that by the Leibinz formula
        \begin{align*}
        \abs{D^\beta_\xi b(x, \xi)} 
        & \leq \sum_{\mu \leq \beta} \binom{\beta}{\mu} \abs{D^\mu_\xi c(x, \xi)} \abs{D^{\beta - \mu} \sym[\xi]^{-m}} \\
        & \leq C \sum_{\mu \leq \beta} \binom{\beta}{\mu} \sym[\xi]^{m + m' - \abs{\mu}} \sym[\xi]^{-m - \abs{\beta - \mu}} \\
        & \leq C \sum_{\mu \leq \beta} \binom{\beta}{\mu} \sym[\xi]^{ m' - (\abs{\mu} +  \abs{\beta - \mu})} \\
        & = C \sum_{\mu \leq \beta} \binom{\beta}{\mu} \sym[\xi]^{ m' -\abs{\beta}} \\
        &= C 2^{\beta} \sym[\xi]^{ m' -\abs{\beta}} 
        \end{align*}
        where we have use the definition of $c$ and applied the result proven for $a$ with $m \mapsto -m$. Thus, $b \in S^{m'}_\infty(\Omega; \R^n)$. 
    \end{itemize}
It is clear that $a \cdot b = c$ and we have therefore shown that $S^{m + m'}_\infty(\Omega; \R^n) \subset S^{m}_\infty(\Omega; \R^n) \cdot S^{m'}_\infty(\Omega; \R^n)$. 


    
\end{proof}

A sumarising theorem: 
\begin{ftheorem}
    Given $p, n \in \N$ and $\Omega \subset \R^p$ such that $\Omega \subset \overline{\mathrm{Int}(\Omega)}$. Let 
    \[
    S^\infty_\infty(\Omega; \R^n) = \bigcup_{m \in \R} S^m_\infty(\Omega; \R^n). 
    \]
    Then $S^\infty_\infty(\Omega; \R^n)$ is a graded algebra over $\R$ with continuous  inclusion  $S^{m}_\infty(\Omega; \R^n) \to S^{m'}_\infty(\Omega; \R^n)$ for all $m \leq m'$. 
\end{ftheorem}




\subsection{Ellipticity of symbols}
\begin{fdefinition}
    Given $p, n \in \N$, $m \in \R$ and $\Omega \subset \R^p$ such that $\Omega \subset \overline{\mathrm{Int}(\Omega)}$, an order $m$ symbol $a \in S^m_\infty(\Omega; \R^n)$ is (globally) \textbf{elliptic} if there exist $\epsilon \in \R_{>0}$ such that 
    \[
        \inf_{\abs{\xi} \geq 1/\epsilon} \abs{a(x, \xi)} \geq \epsilon \sym[\xi]^m. 
    \]
\end{fdefinition}
The importance of elliptic symbol is that they are invertible modulo $S^{-\infty}_\infty(\Omega; \R^n)$. 

\begin{flemma}
    Given $p, n \in \N$, $m \in \R$ and $\Omega \subset \R^p$ such that $\Omega \subset \overline{\mathrm{Int}(\Omega)}$. Let $a \in S^m_\infty(\Omega; \R^n)$ be an elliptic symbol of order $m$. Then there exist a symbol $b \in S^{-m}_\infty(\Omega; \R^n)$ such that 
    \[
    a \cdot b - 1 \in S^{-\infty}_\infty(\Omega; \R^n). 
    \]
\end{flemma}
\begin{proof}
    We shall follow the general strategy of inverting the symbol outside of a compact set. Let $\phi \in C^\infty_c(\R^n)$ be a smooth cut off function, i.e $0 \leq \phi \leq 1$ and $ \phi(\xi) = 1$ for $\abs{\xi} < 1$ and $\phi(\xi) = 0 $ for $\abs{\xi} > 2$. \\
    
    Let $a \in S^{m}_\infty(\Omega; \R^n)$ be an elliptic symbol, that is, for any fixed $\epsilon \in \R_{> 0}$, we have 
    \[
    \abs{a(x, \xi)} \geq \epsilon \sym[\xi]^m
    \]
    for any $\abs{\xi} \geq 1/\epsilon$. Thus, we can define 
    \begin{align*}
    b(x, \xi) = 
    \begin{cases}
    \frac{1 - \phi( \epsilon \xi /2)}{a(x, \xi)} & \abs{\xi} \geq 1/ \epsilon \\
    0 & \abs{\xi} < 1 / \epsilon. 
    \end{cases}
    \end{align*}
    We check: 
    \begin{description}
        \item[$b$ is well-defined and smooth. ] \hfill \\
        We note that $\abs{a(x, \xi)} > 0$ whenever $\abs{\xi} \geq 1/\epsilon$ and therefore $b$ is well defined in that region. For smoothness, we note first that $b$ is smooth in the regions $\abs{\xi} > 1/ \epsilon$ and $\abs{\xi} < 1/\epsilon$. Set $\delta = 1/(2 \epsilon)$. In the region where $1/\epsilon - \delta < \abs{\xi} < 1/\epsilon + \delta$, we have $\abs{\epsilon \xi/ 2} < 1/\epsilon$ and therefore $b(x, \xi) \equiv 0$ in this region and is thus smooth. Since the we have covered $\Omega \times \R^n$ by the three chart domain above, $b$ is smooth by the (smooth) gluing lemma. 
        
        \item[$b$ is a symbol of order $-m$.  ] \hfill \\
        We can prove by induction that in the region $\abs{\xi} \geq 1/ \epsilon$
        \begin{align*}
        D^\alpha_x D^\beta_\xi b = a^{-1 - \abs{\alpha} - \abs{\beta}} G_{\alpha \beta}
        \end{align*}
        for all multi-index $\alpha, \beta$, where $G_{\alpha \beta}$ is a symbol of order $(\abs{\alpha} + \abs{\beta})m  - \abs{\beta}$. Therefore, using the ellipticity estimate for $a$, we get 
        \begin{align*}
        \norm[b]_{k, -m} 
        & = \sup_{(x, \xi) \in \mathrm{Int}(\Omega) \times \R^n} \frac{\abs{D^\alpha_x D^\beta_\xi b(x, \xi)}}{\sym[\xi]^{-m-k}} \\
        &= \sup_{\abs{\xi} \geq 1/\epsilon} \abs{a^{-1 - \abs{\alpha} - \abs{\beta}} G_{\alpha \beta}} \sym[\xi]^{m + k} \\
        &\leq \frac{\norm[G_{\alpha \beta}]_{0, (\abs{\alpha} + \abs{\beta})m - \abs{\beta}}}{\epsilon} \sup_{\abs{\xi} \geq 1/\epsilon^{1 + \abs{\alpha} + \abs{\beta}}} \sym[\xi]^{-m(1 + \abs{\alpha} + \abs{\beta})} \sym[\xi]^{m + k}\\
        & < \infty
        \end{align*}
        as required. 
        
        \item[$b$ is an inverse of $a$ modulo $ S^{-\infty}_\infty(\Omega; \R^n)$. ] \hfill \\
        The main observation is that the set where $b$ fails to be the multiplicative inverse of $a$ is a compact set (in $\xi$) and thus $a \cdot b - 1$ is in fact a compactly supported smooth function of $\xi$ which is a subset of $S^{-\infty}_\infty(\Omega; \R^n)$. \\
        \\
        Explicitly, for any $N \in \N$
        \begin{align*}
        \sup_{(x, \xi) \in \mathrm{Int}(\Omega) \times \R^n} \frac{\abs{D^\alpha_x D^\beta_\xi (a\cdot b - 1)} }{\sym[\xi]^{-N}}
        &\leq \sup_{\abs{\xi} \leq 1/ \epsilon} \sym[\xi]^N \abs{D^\alpha_x D^\beta_\xi (\phi(\xi \epsilon / 2))} < \infty.  
        \end{align*}
        
    \end{description}
\end{proof}


\section{Pseudodifferential Operators ($\Psi$DO's)}
As mentioned in section \ref{}, we wanted to generalise the action of differential operators 
\begin{align*}
P(x, D) u = \frac{1}{(2\pi)^n} \int P(x, \xi) e^{i (x - y)\xi} u(y) \d[y] \d[\xi]
\end{align*}
where $P$ is an $m^{th}$ order polynomial in $\xi$ with $C^\infty$ coefficient, to the actions of symbols $a \in S^{m}_\infty(\R^n; \R^n)$
\begin{align*}
A_a u = \frac{1}{(2\pi)^n} \int a(x, \xi) e^{i (x - y)\xi} u(y) \d[y] \d[\xi]
\end{align*}
or $a \in S^{m}_\infty(\R^{2n}_{(x, y)}; \R^n)$ with action 
\begin{align*}
A_a u = \frac{1}{(2\pi)^n} \int a(x, y, \xi) e^{i (x - y)\xi} u(y) \d[y] \d[\xi]. 
\end{align*}
One of the result we will prove is that action of $a(x, y, \xi)$ as in the later case can always be reduced to the action of some other $a(x, \xi)$ as in the former case. \\


Here we shall introduce a slightly more general symbol space, $\sym[x-y]^w S^{m}_\infty(\Omega; \R^n)$, to allow for polynomial growth perpendicular to the diagonal. 
\begin{fdefinition}
    Given $m, w \in \R$, a $w$-\textbf{weighted symbol space of order} $m$, $\sym[x - y]^w S^{m}_\infty(\R^{2n}_{x,y}, \R^n) $ is given by 
    \begin{align*}
    a \in \sym[x - y]^w S^{m}_\infty(\R^{2n}_{x,y}, \R^n) \iff a(x, y, \xi) = \sym[x - y]^w \tilde{a}(x, y, \xi), \, \tilde{a} \in S^{m}_\infty(\Omega; \R^n)
    \end{align*}
    or equivalently, $a \in \sym[x - y]^w S^{m}_\infty(\R^{2n}_{x,y}, \R^n) $ if and only if for all multi-index $\alpha, \beta, \gamma$, 
    \begin{align*}
    \abs{D^\alpha_x D^\beta_yD^\gamma_\xi a(x, y, \xi)} \leq C \sym[x - y]^w \sym[\xi]^{m - \abs{\gamma}}. 
    \end{align*}
\end{fdefinition}

We shall show that the elements $a \in \sym[x - y]^w S^{m}_\infty(\R^{2n}_{x,y}, \R^n) $ acts on $S(\R^n)$ via the Schwartz kernel 
\[
I(a) =  \frac{1}{(2\pi)^n} \int e^{i(x - y)\xi} a(x, y, \xi) \d[\xi]. 
\]
\begin{fprop}
    Let $n \in \N$ and $m, w \in \R$ with $m < -n$, then the map
    \begin{align*}
    I: \sym[x - y]^w S^{m}_\infty(\R^{2n}_{x,y}, \R^n) &\to (1 + \abs{x}^2 + \abs{y}^2) C^0_\infty(\R^{2n}) \\
    a &\mapsto I(a) = \frac{1}{(2\pi)^n} \int e^{i(x - y)\xi} a(x, y, \xi) \d[\xi]
    \end{align*}
    extends by continuity to 
    \begin{align*}
    I: \sym[x - y]^w S^{m}_\infty(\R^{2n}_{x,y}, \R^n) \to S'(\R^{2n})
    \end{align*}
    in the topology of $S^{m + \epsilon}_\infty(\Omega; \R^n)$ for any $\epsilon \in \R_{>0}$. 
\end{fprop}
\begin{proof}
    
\end{proof}


\section{Microlocalisation}
Roughly, the support of a distribution in $\R^n$ consist of points $x \in \R^n$ where the distribution is non-zero after any smooth cut-offs near $x$. 
\begin{fdefinition}
    The \textbf{support of a tempered distribution} $u \in S'(\R^n)$ is given by the set
    \[
    \supp(u) = \set{x \in \R^n \wh \exists \phi \in S(\R^n), \phi(x) \neq 0, \phi u = 0}^c. 
    \]
\end{fdefinition}

And the singular support of a distribution is the set of points where the distribution fails to behave like an element of $S(\R^n)$. 
\begin{fdefinition}
    The \textbf{singular support of a tempered distribution} $ u \in S'(\R^n)$ is given by the set 
    \[
    \mathrm{sing supp}(u) = \set{x \in \R^n \wh \exists \phi \in S(R^n), \phi(x) \neq 0, \phi(u) \in S(\R^n)}^c
    \]
\end{fdefinition}
Note that if the singular support of a tempered distribution is empty, it is then an element of $C^\infty(\R^n)$ 
The support of an operator is given by the support of its Schwartz kernel. 
\begin{fdefinition}
    The \textbf{support of a continuous linear operator} $A: S(R^n) \to S'(\R^n)$ is given by 
    \[
    \supp(A) = \supp(K_A) \subset \R^n \times \R^n
    \]
    where $K_A \in S'(\R^n \times \R^n)$ is the Schwartz kernel of $A$. 
\end{fdefinition}

We note from the above that supports or singular supports are complement of open sets, therefore they are closed. 
We have the following result relating the support of a smooth function after the action of a continuous linear operator. 
\begin{fprop}[Calculus of support]
    Let $A: S(\R^n) \to S(\R^n)$ be a continuous linear operator and $\phi \in C^\infty_c(\R^n)$, then
    \[
    \supp(A \phi) \subset \supp(A) \circ \supp(\phi) := \set{x \in \R^n \wh \exists y \in \supp(\phi), (x, y) \in \supp(A) }. 
    \]
\end{fprop}
\begin{proof}
    We shall show the contrapositive statement:
    \[
    x \not \in \supp(A) \circ \supp(\phi) \implies x \not \in \supp(A \phi). 
    \]
    Suppose $x \not \in \supp(A) \circ \supp(\phi)$. Observe that 
    \[
    \supp(A) \circ \supp(\phi) = \pi_x(\pi_y^{-1}(\supp(\phi)) \cap \supp(A))
    \]
    where $\pi_{x, y} : \R^2 \to \R$ are the projection map to the respective coordinates. Since $\supp(A)$ is closed and $\supp(\phi)$ is compact, we have that $\supp(A) \circ \supp(\phi)$ is closed and thus $x$ belongs to an open set. We can therefore choose a smooth cutt-off function $\chi \in C^\infty_c(\R^n)$ supported at $x$ ($\chi(x) \neq 0$) but away from $\supp(A) \circ \supp(\phi)$. Thus, 
    \[
    \supp(A) \cap (\supp(\chi) \times \supp(\phi)) = \emptyset
    \]
    and hence $\chi(x) K_A(x, y) \phi(y) = 0 \implies \chi A \phi = 0$, as required. 
\end{proof}



\subsection{Pseudolocality} 
We shall show two results on the singular support of pseudodifferential operators. The first result is that the singular support of any $\Phi$DO is contained within the diagonal, i.e. they are smooth away from $x = y$. The second result is teh pseudolocality result that says that action $\Psi$DO's do not increase singular support of distributions. 

\begin{fprop}
    Let $A \in \Psi^m_\infty(\R^n)$ for some $m \in \R$, then
    \[
    \mathrm{sing supp}(A) \subset \set{(x, y) \in \R^{2n} \wh x = y}. 
    \]
\end{fprop}
\begin{proof}
    We shall prove this theorem for elements of $\Psi^{-\infty}_\infty(\R^n)$ and then extend by continuity to all orders. 
    Let $A \in \Psi^{-\infty}_\infty(\R^n)$ with symbol $a \in S^{-\infty}_\infty(\R^{2n}; \R^n)$. Its singular support is given by the singular support of the kernel. Since all derivatives of $a$ are $O(\sym[\xi]^{-\infty})$, the oscillatory integral below representing the kernel is absolutely convergent  and, using integration by parts, we get
    \begin{align*}
    I(a) 
    & = \frac{1}{(2\pi)^n} \int e^{i(x - y) \xi} a(x, y, \xi) \d[\xi] \\
    & = \frac{1}{(2\pi)^n}  \int \frac{1}{(x - y)^\alpha} (-D^\alpha_\xi) \brac{e^{i(x - y) \xi} }a(x, y, \xi) \d[\xi] \\
    & = \frac{1}{(x - y)^\alpha} \frac{1}{(2\pi)^n} \int e^{i(x - y) \xi} (-D^\alpha_\xi) a(x, y, \xi) \d[\xi] \\
    & = \frac{1}{(x - y)^\alpha} I((-D^\alpha_\xi)a)
    \end{align*}
    which is true for all multi-index $\alpha$ of any order. Since all $x, y$-derivatives of $a$ are uniformly bounded by $\sym[\xi]^{-N}$ for any $N \in \N$, we can differentiate under the integral sign to get the equation
    \begin{align*}
    D^\beta_x D^\gamma_y (x - y)^\alpha I(a) 
    & = \frac{1}{(2\pi)^n} \int D^\beta_x D^\gamma_y e^{i(x - y) \xi} (-D^\alpha_\xi) a(x, y, \xi) \d[\xi] \\
    & = \frac{1}{(2\pi)^n} \int (-1)^{\abs{\beta}}\xi^{\beta + \gamma}e^{i(x - y) \xi} (-D^\alpha_\xi) a(x, y, \xi) \d[\xi] \\
    \end{align*}
    where the last integral gives a smooth function, thus showing that $(x - y)^\alpha I(a)$ is smooth for all $\alpha$, and hence $I(a)$ is smooth away from $x = y$. \\
    \\
    Now, for a general $A \in \Psi^m_\infty(\R^n)$, $m \in \R$, we shall use the  density of $S^{-\infty}_\infty(\R^{2n}; \R^n) \subset S^{m}_\infty(\R^{2n}; \R^n)$ and that $I$ extends by continuity to a map $I : S^{m}_\infty(\R^{2n}; \R^n) \to S'(\R^{2n})$ in the topology $S^{m + \epsilon}_\infty(\R^{2n}; \R^n)$ for any $\epsilon > 0$  \ref{}. 
    
    
\end{proof}


\begin{fprop}
    Let $A \in \Psi^m_\infty(\R^n)$ for some $m \in \R$ and $u \in C^{-\infty}(\R^n)$, then 
    \begin{align*}
    \mathrm{sing supp}(A u) \subset \mathrm{sing supp }(u). 
    \end{align*}
    We call operators that satisfies the above property \textit{pseudolocal}
\end{fprop}
\begin{proof}
     Again we shall prove the contrapositive statement that 
    \[
    x \not \in \mathrm{sing supp}(u) \implies x \not \in \mathrm{sing supp}(Au)
    \]
    Let $u \in S'(\R^n)$ be compactly supported and $x_0 \not \in \mathrm{sing supp}(u)$.  We can choose $\chi \in S(\R^n)$, (normalised) so that $\chi \equiv 1$ in a neighbourhood of $x_0$ and that $\chi u \in S(\R^n)$. Observe that 
    \begin{align*}
    Au = A(\chi u + (1 - \chi)u) = A(\chi u) + A(1 - \chi)u. 
    \end{align*}
    Since $\chi xu \in S(\R^n) \implies A\chi u \in S(\R^n)$ \cite{rbm lemma 2.3}, we have that 
    \begin{align*}
    \mathrm{singsupp}(Au) = \mathrm{singsupp}(A(1 - \chi)u). 
    \end{align*}
    Furthermore, we know that $x_0 \not \in \supp((1 - \chi)u)$. 
    Now, we shall further cut-off near $x_0$ by choosing a $\phi \in S(\R^n)$ compactly supported  away from $\supp(1 - \chi)$ and $\phi \equiv 1$ near $x_0$, i.e. 
    \[
    \supp(1 - \chi) \cap \supp \phi = \emptyset. 
    \]
    We now have an operator $\phi A(1 - \chi) $ with kernel
    \[
    \phi(x) K_A(x, y) ( 1 - \phi(y))
    \]
    that vanishes (in particular smooth) on the diagonal. Since we have shown that the singular support of a pseudodifferential operator have to be contained in the diagonal, we conclude that $\phi A(1 - \chi)$ is a smoothing operator, and thus $\phi A (1 - \chi) u \in C^\infty(\R^n)$ as required. .  
\end{proof}


\subsection{Elliptic, Characteristic, Wavefront sets}
We will now define \textit{ellipticity at a point} in phase space which allow up to define various microlocal contructions that focus on  localised (conically in phase space) behaviour $\Psi$DO's and distributions. 

\begin{fdefinition}
    A pseudodifferential operator, $A \in \Psi^m_\infty(\R^n), m \in \R$ is \textbf{elliptic at a point} $(x_0, \xi_0) \in \R^n \times \R^n \setminus \set{0}$ if there exist $\epsilon > 0$ such that its left-reduced symbol satisfies the lower bound
    \[
    \abs{\sigma_L(A)(x, \xi)} \geq \epsilon \sym[\xi]^m
    \]
    in the region
    \[
    \overline{U}_\epsilon = \set{(x, \xi) \in \R^n \times \R^n \setminus \set{0} \wh \abs{x - x_0} \leq \epsilon, \abs{\widehat{\xi} - \widehat{\xi_0}} \leq \epsilon, \abs{\xi} \geq 1/\epsilon}
    \]
    where $\widehat{\xi} = \xi / \abs{\xi}$ for any non-zero $\xi \in \R^n$. We denote the set of all elliptic points of $A$ as 
    \[
    \Ell^m(A) = \set{(x, \xi) \in \R^n \times \R^n \setminus \set{0} \wh A \text{  is elliptic of order $m$ at } (x, \xi)}
    \]
    and its complement in $\R^n \times \R^n \setminus \set{0}$ as 
    \begin{align*}
    \Char^m(A) 
    &= \Ell^m(A)^c \setminus \set{(x, 0)\wh x \in \R^n} \\
    &= \set{(x, \xi) \in \R^n \times \R^n \setminus \set{0} \wh A \text{  is \textbf{not} elliptic of order $m$ at } (x, \xi)}
    \end{align*}
\end{fdefinition}

\begin{flemma}
    Let $A \in \Psi^m_\infty(\R^n), m \in \R$. 
    \begin{enumerate}
        \item If $\sigma_m(A)(x, \xi)$ is homogeneous of degree $m$ in $\xi$, then 
        \[
        \Ell^m(A) = \set{(x_0, \xi_0) \wh  \xi_0 \neq 0, \sigma_m(A)(x_0, \xi_0) \neq 0}. 
        \]
        \item $\Ell^m(A) $ is open in $\R^n \times \R^n$. 
        \item $\Ell^m(A)$ is conic in $\R^n \times \R^n \setminus \set{0}$, in the sense that 
        \[(x_0, \xi_0) \in \Ell^m(A) \implies (x_0, t \xi_0) \in \Ell^m(A), \forall t \in \R_{>0}.\] 
        \item $\Char^m(A)$ is closed conic. 
        \item if $B \in \Psi^{m'}(\R^n)$, then 
        \[\Char^{m + m'}(A \circ B) = \Char^m(A) \cup \Char^{m'}(B).\]
    \end{enumerate}
\end{flemma}
\begin{proof}
    Let $A \in \Psi^m_\infty(\R^n)$, $m \in \R$ be given. 
    \begin{enumerate}
        \item Suppose the principal symbol $\sigma_m(A)(x, \xi)$ is homogeneous of order $m$ in $\xi$. We need to show that 
        \[
        (x_0, \xi_0) \in \Ell^m(A) \iff \xi_0 \neq 0, \, \sigma_m(A)(x_0, \xi_0) \neq 0. 
        \]
        If $\xi_0 = 0$, $(x_0, \xi_0) \not \in \Ell^m_\infty$ by definition of ellipticity. If $\sigma_m(x_0, \xi_0) = 0$, by homogeneity, we have 
        \[
        \sigma_m(x_0, t \xi_0) = t^m \sigma_m(x_0, \xi_0) = t^m \cdot 0 = 0
        \]
        for all $t \in \R_{> 0}$. By definition of principal symbol, we can write the left symbol of $A$ as 
        \[
        \sigma_L(A) = \sigma_m(A) + a
        \]
        where $a \in S^{m -1}_\infty(\R^n; \R^n)$. Now, observe that for any $\epsilon > 0$, the set 
         \[
        \overline{U}_\epsilon = \set{(x, \xi) \in \R^n \times \R^n \setminus \set{0} \wh \abs{x - x_0} \leq \epsilon, \abs{\widehat{\xi} - \widehat{\xi_0}} \leq \epsilon, \abs{\xi} \geq 1/\epsilon}
        \]
        contains the(open) half-line starting at $\widehat{\xi_0} / \epsilon$, i.e. the set $\set{(x_0, t \xi_0/(\abs{\xi_0}\epsilon)\wh t > 0}$. However, by the symbol estimate of $a$, 
        \begin{align*}
        \abs{\sigma_L(A)\brac{x_0, \frac{t \xi_0}{\abs{\xi_0} \epsilon}}}
        & \leq \brac{\frac{t}{\epsilon \abs{\xi_0}}}^m \abs{\sigma_m(x_0, \xi_0)} + \abs{a\brac{x_0, \frac{t \xi_0}{\abs{\xi_0} \epsilon}}} \\
        & = 0 + \abs{a\brac{x_0, \frac{t \xi_0}{\abs{\xi_0} \epsilon}}} \\
        & \leq C \sym[\frac{t \xi_0}{\abs{\xi_0} \epsilon}]^{m - 1} \\
        & = C\sym[t/\epsilon]^{m - 1}
        \end{align*}
        and therefore 
        \begin{align*}
        \inf_{(x, \xi) \in \overline{U}_\epsilon} \frac{\abs{\sigma_L(A)(x, \xi)}}{\sym[\xi]^m} 
        & \leq \inf_{t > 0}  \frac{\abs{\sigma_L(A)\brac{x_0, \frac{t \xi_0}{\abs{\xi_0} \epsilon}}} }{\sym[t /\epsilon]^m} \\
        & \leq \inf_{t > 0}  \frac{C\sym[t/ \epsilon]^{m - 1} }{\sym[t /\epsilon]^m} \\
        & = C\inf_{t > 0} \sym[t/\epsilon]^{-1}\\
        & = 0
        \end{align*}
        which means that $(x_0, \xi_0) \not \in \Ell^m(A)$. \\
        \\
        Conversely, if $\sigma_m(A)(x_0, \xi_0) \neq 0$, by continuity and homogeneity,  $\sigma_m(A)$, is non-zero in a (closed) conic neighbourhood, i.e. there exist $\epsilon > 0$ such that $\sigma_m(A) \neq 0$ in 
        \begin{align*}
        \overline{U}_\epsilon = \set{(x, \xi) \wh \abs{x - x_0} \leq \epsilon, \abs{\widehat{\xi} - \widehat{\xi_0}}\leq \epsilon, \abs{\xi} \geq 1/ \epsilon}. 
        \end{align*}
        Again, writing the left symbol as a sum of the principal symbol an a lower order term, we observe that in $\overline{U}_\epsilon$, 
        \begin{align*}
        \frac{\abs{\sigma_L(A)(x, \xi)} }{\sym[\xi]^m}
        & \geq \frac{\abs{\abs{\sigma_m(A)(x, \xi)} - \abs{a(x, \xi)}}}{\sym[\xi]^m} \\
        & = \abs{\frac{\abs{\xi}^m}{\sym[\xi]^m} \abs{\sigma_m(A)(x, \widehat{\xi})} - \frac{\abs{a(x, \xi)}}{\sym[\xi]^m}} \\
        \end{align*}
        By the symbol estimate of $a$, the second term is tending to $0$ which the first term is bounded below by $C = \inf_{(x, \xi) \in \overline{U}_\epsilon} \abs{\sigma_m(A)(x, \xi)} > 0$. Therefore, choosing a smaller $\epsilon$ if necessary, we have $\abs{a(x, \xi)} / \sym[\xi]^m < C$ and thus 
        \begin{align*}
        \inf_{(x, \xi) \in \overline{U}_\epsilon} \frac{\abs{\sigma_L(A)(x, \xi)} }{\sym[\xi]^m}
        \geq C' 
        \geq \epsilon. 
        \end{align*}
        and therefore $(x_0, \xi_0) \in \Ell^m(A)$. 
        
        \item We note first that if the principal symbol is homogeneous of degree $m$, the previous result applies and continuity of the principal symbol guarantee that the elliptic set is open (if $\sigma_m(A)$ is non-zero at a point, it is non-zero in an open neighbourhood of the point). \\
        \\
        For the general case, suppose $(x_0, \xi_0) \in \Ell^m(A)$. We therefore have for some $\epsilon > 0$, 
        
         \[
        \abs{\sigma_L(A)(x, \xi)} \geq \epsilon \sym[\xi]^m
        \]
        in the region
        \[
        \overline{U}_\epsilon(x_0, \xi_0) = \set{(x, \xi) \in \R^n \times \R^n \setminus \set{0} \wh \abs{x - x_0} \leq \epsilon, \abs{\widehat{\xi} - \widehat{\xi_0}} \leq \epsilon, \abs{\xi} \geq 1/\epsilon}. 
        \]
        
        It suffice to show that there is an open neighbourhood of $(x_0, \xi_0)$ where $A$ remains elliptic. We can take desired open neighbourhood to be 
        \begin{align*}
        V = \set{(x', \xi') \wh \xi' \neq 0, \, \abs{x' - x_0} < \epsilon/ 2, \, \abs{\widehat{\xi'} - \widehat{\xi_0}} < \epsilon / 2}. 
        \end{align*}
        Then, we can check that for every $(x', \xi') \in V$, $A$ satisfies the elliptic estimate in $\overline{U}_{\epsilon / 2}(x', \xi')$. Indeed, if $(x, \xi) \in \overline{U}_{\epsilon / 2}(x', \xi')$, then 
        \begin{align*}
        &\abs{x - x_0} \leq \abs{x - x'} + \abs{x' - x_0} < \frac{\epsilon}{2} + \frac{\epsilon}{2} = \epsilon \\
        &\abs{\widehat{\xi} - \widehat{\xi_0}} \leq \abs{\widehat{\xi} - \widehat{\xi'}} + \abs{\widehat{\xi'} - \widehat{\xi_0}} < \frac{\epsilon}{2} + \frac{\epsilon}{2} = \epsilon \\
        &\abs{\xi} \geq 2/\epsilon \geq 1/\epsilon
        \end{align*}
        which shows that $\overline{U}_{\epsilon / 2}(x', \xi') \subset \overline{U}_{\epsilon}(x_0, \xi_0)$. Therefore, 
        \begin{align*}
        \inf_{(x, \xi) \in \overline{U}_{\epsilon / 2}(x', \xi')} \frac{\abs{\sigma_L(A)(x, \xi)}}{\sym[\xi]^m}
        \geq  \inf_{(x, \xi) \in \overline{U}_{\epsilon}(x_0, \xi_0)} \frac{\abs{\sigma_L(A)(x, \xi)}}{\sym[\xi]^m} 
         \geq \epsilon 
         \geq \epsilon / 2
        \end{align*}
        as required. 
        
        
        
        \item Again, this result is immediate if the principal symbol is homogeneous in  $\xi$. In general, this result come from the observation that only $\widehat{\xi} = \xi / \abs{\xi}$ appears in $\overline{U}_\epsilon$ in the definition of $\Ell^m(A)$, i.e. only the \emph{direction} in the dual variable is important. \\
        \\
        Explicitly, let $(x_0, \xi_0) \in \Ell^m(A)$ and $t \in \R_{> 0}$. Clearly $t \xi_0 \neq 0$. And note that 
        \begin{align*}
        \overline{U}_\epsilon(x_0, \xi_0) = \overline{U}_\epsilon(x_0, t\xi_0) 
        \end{align*}
        since $\widehat{\xi} = \widehat{t \xi}$. 
        
        \item $\Char^m(A) = \Ell^m(A)^c$ where $\Ell^m(A)$ is open and conic. Since complement of conic set is conic, and complement of open is closed, we conclude that $\Char^m(A)$ is closed conic. 
        
        
        \item If both principal symbols are homoegenous of degree $m, m'$ respectively, we can applied the result above and by symbol calculus, we have
        \begin{align*}
        \Ell^{m + m'}(A \circ B) 
        & = \set{(x, \xi) \wh \xi \neq 0, \sigma_{m + m'}(A \circ B) = \sigma_m(A) \sigma_{m'}(B) \neq 0}\\
        & = \set{(x, \xi) \wh \xi \neq 0, \sigma_m(A)  \neq 0} \cap  \set{(x, \xi) \wh \xi \neq 0, \sigma_{m'}(B)  \neq 0} \\
        & = \Ell^m(A) \cap \Ell^{m'}(B). 
        \end{align*}
        Taking complement give the desired result. \\
        \\
        In general, 
    \end{enumerate}

\end{proof}



\begin{fdefinition}
    The \textbf{wavefront set} of a compactly supported tempered distribution 
    \[
    u \in C^{-\infty}_c(\R^n) = \set{u \in S'(\R^n) \wh \supp(u) \Subset \R^n} 
    \]
    is given by 
    \begin{align*}
    \WF(u) = \bigcap \set{\Char^0(A) \wh A \in \Psi^0_\infty(\R^n), Au \in C^\infty(\R^n)}. 
    \end{align*}
    For general tempered distribution $u \in S'(\R^n)$, its wavefront set is given by 
    \begin{align*}
    \WF(u) = \bigcup_{\chi \in C^\infty_c(\R^n)} \WF(\chi u). 
    \end{align*}
\end{fdefinition} 


\begin{fprop}
    For compactly supported tempered distribution, $u \in C^{-\infty}_c(\R^n)$, 
    \begin{align*}
    \pi(\WF(u)) = \mathrm{singsupp}(u). 
    \end{align*}
    where $\pi(x, y) = x$ is the projection map. 
\end{fprop}
\begin{proof}
    To show $\pi(\WF(u)) \subset \mathrm{singsupp}(u)$, we observe that, by definition of singular support, 
    \begin{align*}
    x_0 \not \in \mathrm{singsupp}(u) \implies \exists \phi \in S(\R^n), \, \phi(x_0) \neq 0, \, \phi u \in S(R^n). 
    \end{align*}
    But since multiplication by $\phi$ gives an operator in $\Psi^0_\infty(\R^n)$ which is elliptic at $(x_0, \xi)$ for any $\xi \neq 0$ ($\phi$ is its own principal symbol which happens to be homogeneous and non-zero for any $(x_0, \xi), \xi \neq 0$). Therefore, $x_0 \not \in \pi(\WF(u))$. \\
    \\
    Conversely, if $x_0 \not\in \pi(\WF(u))$, then for all $\xi \neq 0$, there exist $A_\xi \in \Psi^0_\infty(\R^n)$ such that $A_\xi$ is elliptic at $(x_0, \xi)$ and $A_\xi u \in C^\infty(\R^n)$. Since elliptic set $\Ell^0(A_\xi)$ is open and conic, we know that there exist $\epsilon = \epsilon(\xi)$ such that $A_\xi$ is elliptic in the open conic set
    \begin{align*}
    V_\xi = \set{(x', \xi')\in \R^n \times (\R^n \setminus \set{0}) \wh \abs{x' - x_0} < \epsilon, \abs{\widehat{\xi'} - \widehat{\xi}} < \epsilon }. 
    \end{align*}
    Compactness of the sphere (note that $\xi' \mapsto \widehat{\xi'}$ is an embedding of $\R^n \setminus \set{0}$ into $S^n$) allow us to cover $\set{x_0} \times (\R^n \setminus \set{0})$ with finite number of $V_{\xi_j}, j = 1, \dots, N$ with corresponding operators $A_{\xi_j}$. \\
    Now, consider the operator
    \begin{align*}
    A = \sum_{j = 1}^N A^*_{\xi_j}A_{\xi_j}. 
    \end{align*}
    By the pseudolocality of pseudodifferential operators, we know that $A_{\xi_j} u \in C^\infty(\R^n) \implies A^*_{\xi_j} A_{\xi_j} u \in \C^\infty(\R^n)$. Therefore, $Au \in C^\infty(\R^n)$ and $A$ is elliptic at $(x_0, \xi), \forall \xi \neq 0$ with non-negative symbol. We can pick a smooth cut-off $\chi$, $\chi \equiv 1$ when resticted to an $\epsilon/2$-ball around $x_0$ forming an operator
    \begin{align*}
    A + (1 - \chi) \in \Psi^0_\infty(\R^n)
    \end{align*}
    that is globally elliptic. We can construct a (global) elliptic parametrix $E$ so that 
    \begin{align*}
    1 - (E \circ A + E (1 - \chi)) \in \Psi^{-\infty}_\infty(\R^n). 
    \end{align*}
    Finally, we can choose any smooth cut-off $\phi$ with support subordinate to that of $\chi$, i.e. $\supp(\phi) \subset \supp(\chi)$ and note that 
    \begin{align*}
    \phi \circ E \circ (1 - \chi) \in \Psi^{-\infty}_\infty(\R^n)
    \end{align*}
    making it a smoothing operator \cite{}. Thus, we conclude 
    \begin{align*}
    \phi u = \phi E \circ A u + \phi \circ E \circ (1 - \chi) u \in C^\infty(\R^n)
    \end{align*}
    as required. 
    
    
   
    
\end{proof}


\begin{fdefinition}
    Let $a \in S^{m}_\infty(\R^{p}; \R^n)$ for some $m \in \R$, $p, n \in \N$ be a symbol. We say $a$ is of order $-\infty$ at a point $(x_0, \xi_0) \in \R^p \times \R^n \setminus \set{0}$ (write $a = O(\sym[\xi]^{-\infty})$) if there exist $\epsilon \in \R_{> 0}$ such that for all $M \in \R$, there is a constant $C_M > 0$ such that 
    \[
    \abs{a(x, \xi)} \leq C_M \sym[\xi]^{-M}
    \]
    in the neigbourhood of $(x_0, \xi_0)$ given by
    \[
    \overline{U}_{(x_0, \xi_0)} = \set{(x, \xi) \in \R^p \times \R^n \wh \abs{x - x_0} \leq \epsilon, \abs{\widehat{\xi} - \widehat{\xi_0}} \leq \epsilon}. 
    \]
    We define the cone support of the symbol $a$ to be all the points in phase space that where it fails to be $O(\sym[\xi]^{-\infty})$. 
    \[
    \mathrm{conesupp}(a) = \set{(x, \xi) \in \R^p \times \R^n \setminus \set{0} \wh a = O(\sym[\xi]^{-\infty}) \text{ at } (x, \xi)}^c. 
    \]
\end{fdefinition}

\begin{flemma}
    Let $a \in S^{\infty}_\infty(\R^{p}; \R^n)$, then 
    \begin{enumerate}
        \item $\mathrm{conesupp}(a)$ is a closed conic set in $\R^p \times \R^n$. 
        \item If $a = O(\sym[\xi]^{-\infty})$ at $(x_0, \xi_0) \in \R^p \times \R^n \setminus \set{0}$, then so is $D^\alpha_x D^\beta_\xi a(x, \xi)$ for any multi-index $\alpha, \beta$
    \end{enumerate}
\end{flemma}

With the lemma above, we can in fact define the cone support of a symbol to be the smallest closed conic subset of the phase space (with $\xi \neq 0$) such that, in the complement, $a$ and all its derivatives are of order $-\infty$.\\
 


\begin{fdefinition}
    Let $A \in \Psi^m_\infty(\R^n)$,  $m \in \R$ be pseudodifferential operator. We define the \textbf{essential support}, $\WF'(A)$, of $A$ to be the cone support of its left symbol, i.e. 
    \begin{align*}
    \WF'(A) = \mathrm{conesupp}(\sigma_L(A)) \subset \R^n \times \R^n \setminus \set{0}
    \end{align*}
\end{fdefinition}

Using symbol calculus we can prove the following result. 
\begin{flemma}
    Let $A \in \Psi^m_\infty(\R^n)$, $B \in \Psi^{m'}_\infty(\R^n)$ be pseudifferential operators. Then 
    \begin{enumerate}
        \item $\WF'(A) = \mathrm{conesupp}(\sigma_R(A))$. 
        \item $\WF'(A\circ B) \subset \WF'(A) \cap \WF'(B)$. 
        \item $\WF'(A + B) = \WF'(A) \cup \WF'(B)$. 
    \end{enumerate}
\end{flemma}

With the concept of essentail support we can define the notion of \emph{microlocal elliptic parametrix} which can be thought of as local inverse at an elliptic point of $\Psi$DO's. 
\begin{fprop}
    Let $A \in \Psi^m_\infty(\R^n)$ and $z \not \in \Char^m(A)$. Then there exist a (two-sided) microlocal parametrix $B \in \Psi^{-m}(\R^n)$ such that 
    \begin{align*}
    z \not \in \WF'(1 - AB) \text{   and   } z \not \in \WF'(1 - BA). 
    \end{align*}
    
\end{fprop}
\begin{proof}
    Let $A \in \Psi^m_\infty(\R^n)$ is elliptic at $(x_0, \xi_0) \in \Ell^m(A)$. For each $\epsilon \in \R_{> 0}$ we define
    \begin{align*}
    \gamma_\epsilon(x, \xi) = \chi\brac{\frac{x - x_0}{\epsilon}} (1 - \chi(\epsilon \xi)) \chi\brac{\frac{\widehat{\xi} - \widehat{\xi_0}}{\epsilon}}
    \end{align*}
    where $\chi \in C^\infty(\R^n)$ is a smooth cut-off function that is identically $1/2$-ball but supported only in the unit ball. Notice that $\gamma_\epsilon \in S^{0}_\infty(\R^{2n}; \R^n)$ with support given by 
    \[
    \supp(\gamma_\epsilon) \subset \set{(x, \xi) \wh \abs{x - x_0} \leq \epsilon,\, \abs{\xi} \geq\frac{1}{2 \epsilon},\, \abs{\widehat{\xi} - \widehat{\xi_0}} \leq \epsilon }. 
    \]
    Restricted to the conic set
    \[
    \overline{U}_{\epsilon} = \set{(x, \xi) \wh \abs{x - x_0} \leq \frac{\epsilon}{2}, \, \abs{\widehat{\xi} - \widehat{\xi_0}} \leq \frac{\epsilon}{2}, \, \abs{\xi} \geq \frac{1}{\epsilon}} \subset \supp(\gamma_\epsilon)
    \]
    it is identically $1$ and therefore $\gamma_\epsilon$ is elliptic at $(x_0, \xi_0)$. Let $L_\epsilon = \mathrm{Op}_L(\gamma_\epsilon)$ be the corresponding pseudodifferential operator. Observe that, by construction 
    \[
    (x_0, \xi_0) \not\in \WF'(1 - L_\epsilon)
    \]
    since $1 - \gamma_\epsilon$ is supported away from an $\epsilon$-neighbourhood of $x = x_0$ and the wavefront set of $L_\epsilon$ is contained in an $\epsilon$-neighbourhood of $(x_0, \xi_0)$, i.e. 
    \[
    \WF'(L_\epsilon) \subset N_\epsilon(x_0, \xi_0) := \set{(x, \xi) \wh \abs{x - x_0} \leq \epsilon, \, \abs{\widehat{\xi} - \widehat{\xi_0}} \leq \epsilon }
    \]
    since $\gamma_\epsilon$ is bounded below in some conic neighbourhood of every point in $N_\epsilon(x_0, \xi_0)$. \\
    \\
    Now, let $G_s = \mathrm{Op}_L(\sym[\xi]^s)$ for each $s \in \R$. Note that $G_s$ is globally elliptic with positive principal symbol. Consider the following operator, 
    \begin{align*}
    J = (1 - L_\epsilon) \circ G_{2m} + A^* A \in \Psi^{2m}_\infty(\R^n)
    \end{align*}
    with principal symbol 
    \[
    \sigma_{2m}(J) = (1 - \gamma_\epsilon)\sym[\xi]^{2m} + \abs{\sigma_m(A)}^2. 
    \]
    Since $\Ell^m(A)$ is open conic, we can choose $\epsilon$ is small enough so that $\Ell^{m}(A) \subset \supp(\gamma_\epsilon)$. Then, by positivity of all terms involved, we have
    \begin{align*}
    \frac{\abs{\sigma_{2m}(J)} }{\sym[\xi]^{2m}}
    &= (1 - \gamma_\epsilon) + \frac{\abs{\sigma_m(A)}^2}{\sym[\xi]^{2m}}
    \end{align*}
    where the first term is bounded below by $1$ outside of $\supp(\gamma_\epsilon)$ while in $\supp(\gamma_\epsilon)$ the second term is bounded below by $\epsilon$ since $A$ is elliptic (of order $m$) at every point in $\supp(\gamma_\epsilon)$. Therefore $J$ is globally elliptic and thus have a global elliptic parametrix $H \in \Psi^{-2m}_\infty(\R^n)$. We shall claim that 
    \[
    B = H \circ A^* \in \Psi^m_\infty(\R^n)
    \]
    is a (right) microlocal elliptic parametrix to $A$. Indeed, 
    \begin{align*}
    B \circ A - 1 
    &= H A^*A - 1 \\
    &= H\brac{J - (1 - L_\epsilon)G_{2m}} - 1\\
    &= (HJ - 1) - H(1 - L_\epsilon)G_{2m}. 
    \end{align*}
    Since $H$ is a global parametrix to $J$, the first term above is a smoothing operator (i.e. an element of $\Psi^{-\infty}_\infty(\R^n)$ and therefore have empty wavefront set. Furthermore, by \cite{wavefront of compositiion} the wavefront set of the second term is a subste of $\WF'(1 - L_\epsilon)$ which does not contain $(x_0, \xi_0)$ by construction. 
\end{proof}

\begin{fprop}
    Pseudodifferential operators are microlocal in the following sense: 
    Let $A \in \Psi^m_\infty(\R^n)$ and $u \in C^{-\infty}_c(\R^n)$, then 
    \begin{align}
    \WF(Au) \subset \WF(u). 
    \end{align}
    In fact, we have 
    \begin{align*}
    \WF(Au) \subset \WF'(A) \cap \WF(u). 
    \end{align*}
\end{fprop}
\begin{proof}
    
\end{proof}

A partial converse to the above is given by the following proposition. \\

\begin{fprop}
    Let $A \in \Psi^m_\infty(\R^n)$ and $u \in C^{- \infty}_c(\R^n)$, then 
    \begin{align*}
    \WF(u) \subset \WF(Au) \cup \Char^m(A).
    \end{align*}
\end{fprop}

\section{Appendix}
\subsection{Stationary phase lemma} 
In the study of pseudodifferential operators, we often encounter integral of highly oscillatory functions of the form
    \begin{align*}
    I(h) = \int_\R a(x) e^{i \varphi(x) /h} dx
    \end{align*}
where $a \in C^\infty_c(\R)$, $\varphi \in C^\infty(\R)$ and we are interested in the asymptotic behaviour as $h \to 0$. We note that if $\varphi$ is linear (or constant), i.e. $\varphi(x) = \alpha x+ \beta$, $\alpha, \beta \in \R$, then, 
\begin{align*}
   \abs{I(h)} = \abs{\int_\R a(x) e^{i (\alpha x + \beta) /h} dx} = \abs{e^{i \beta/h}} \abs{\int_\R a(x) e^{i \alpha x /h} dx} =  \abs{\int_\R a(x) e^{i \alpha x /h} dx} \to 0
\end{align*}
as $h \to 0$ by Riemann-Lebesgue lemma. That is to say, as the length scale of the oscillation tends to zero, the values of the integrand achieve perfect cancellation. In general, if $\varphi'(x) \neq 0$, we expect $e^{i\varphi(x)/h}$ to oscillate at length scale of order $h$ and thus as $h \to 0$, 







\bibliographystyle{apalike}
\bibliography{../main.bib}
\end{document}

