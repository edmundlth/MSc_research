\documentclass{article}
\usepackage{../thesis_style}
\usepackage{standalone} 
\usepackage{import} % use \import{directory}{filename} to include standalone texfiles

\title{Microlocal Analysis Seminar } 
\author{Edmund Lau} 
%\date{\today}


\begin{document}
\maketitle
\tableofcontents
%\listoffigures
%\listoftables
%%%%%%%%%%%%%%%%%%%%%%%%%%%%%%%%%%%%%%%%%%%%%%%%%
%%%%%%%%%%%%%%%%%%%%%%%%%%%%%%%%%%%%%%%%%%%%%%%%%
\pagebreak
\section{Reminder: definitions and notations}
\subsection{Symbols} 
We shall here list the definition of the space of symbols of order $m \in \N$  in Euclidean space $\R^n$ that one encounters in the literature. The main motivation is based on the property of linear differential operators of order $m \in \N$ with smooth coefficient that, after Fourier transform gives the polynomial of $\xi$ with smooth coefficient 
\begin{align*}
P(x, \xi) = \sum_{\abs{\alpha} \leq m} a_\alpha(x) \xi^\alpha. 
\end{align*}


\begin{fdefinition}
    The \textbf{space of symbols of order} $m$, denoted $S^m_\infty(\R^p; \R^n)$, is the space of smooth functions $a \in C^\infty(\R^p \times \R^n)$ such that for all multi-index $\alpha \in \N^p, \beta \in \N^n$
    \begin{align*}
    \abs{D^\alpha_x D^\beta_\xi a (x, \xi)} \leq C_{\alpha, \beta} \sym[\xi]^{m - \abs{\beta}} 
    \end{align*}
    uniformly on $\R^p \times \R^n$. We can also defined the space of symbol, $S^m_\infty(\Omega; \R^n)$ on a set with non-empty interior $\Omega \subset \R^p$, $\Omega \subset \overline{\mathrm{Int}(\Omega)}$ such that the bound above is satisfied uniformly in $(x, \xi) \in \mathrm{Int}(\Omega) \times \R^n$. The subscript $\infty$ refers the uniform boundedness condition in $x$. Together with the family of seminorms (indexed by $N \in \N$) 
    \begin{align} \label{eq : symbol seminorms} 
    \norm[a]_{N, m} = \sup_{(x, \xi) \in \mathrm{Int}(\Omega) \times \R^n} \max_{\abs{\alpha} + \abs{\beta} \leq N} \frac{D^\alpha_x D^\beta_\xi a(x, \xi)}{\sym[\xi]^{m - \abs{\beta}}} 
    \end{align}
    gives a Frechet topology to $S^m_\infty(\Omega; \R^n)$. \\
    
    Furthermore, we define the total symbol space as
    \[
    S^\infty_\infty(\Omega; \R^n) = \bigcup_{m \in \R} S^{m}_\infty(\Omega; \R^n)
    \]
    and the residual \emph{residual} space of the filtration as 
    \[
    S^{-\infty}_\infty(\Omega; \R^n) = \bigcap_{m \in \R} S^{m}_\infty(\Omega; \R^n). 
    \]
    
    Note: In defining pseudodifferential operators, we shall focus on the case where $p = 2n$, i.e. $a(x, y, \xi) \in S^{m}_\infty(\R^{2n}_{x,y}; \R^n)$. 
\end{fdefinition}


\subsection{Quantisation}
Pseudodifferential operators are defined using symbols. The main gadget is the following oscillatory integral: 
\begin{align}
S^{m}_\infty(\R^{2n}; \R^n) \ni a \mapsto I(a) = \frac{1}{(2\pi)^n} \int e^{i(x- y)\xi} a(x, y, \xi) \d[\xi]
\end{align}
with action on Schwartz functions $u \in S(\R^n)$ given by 
\begin{align} \label{eq: oscillatory integral}
I(a)(u) =  \frac{1}{(2\pi)^n} \int e^{i(x- y)\xi} a(x, y, \xi) u(y) \d[y] \d[\xi]. 
\end{align}
The integral \ref{eq: oscillatory integral} above might be divergent unless $m < -n$, but it can be interpreted as a tempered distribution, i.e. a linear function on $S(\R^n)$, with action
\begin{align}
 S(\R^n) \ni v \mapsto I(a)(u)(v) = \frac{1}{(2\pi)^n} \int e^{i(x- y)\xi} a(x, y, \xi) u(y) v(x) \d[y] \d[\xi] \d[x] \in \C.  
\end{align}
The process of turning the symbol $a$ into an operator $S(\R^n) \to S'(\R^n)$ is known as the quantisation procedure. The goal of this talk is the following: 

\begin{mdframed}
    \textbf{\large Goal : } \\
    To establish that the procedure above is well-defined, so that for each $a \in S^{m}_\infty(\R^{2n}; \R^n)$
    \begin{align*}
    I(a) : S(\R^n) & \to S'(\R^n) & & \\
    u &\mapsto I(a)(u) &: S(\R^n) & \to \C \\
     & &v & \mapsto I(a)(u v) = \frac{1}{(2\pi)^n} \int e^{i(x- y)\xi} a(x, y, \xi) u(y) v(x) \d[y] \d[\xi] \d[x]
    \end{align*}
    is a continuous linear map between Frechet spaces. 
\end{mdframed}

\begin{rem}
    Given $a \in S^{m}_\infty(\R^{2n}; \R^n)$, we sometimes write $A = Op(a) = I(a)$ for the operator $S(\R^n) \to S'(\R^n)$ defined by quantising the symbol $a$. Also, once the procedure above is proven to be well-defined, we will, with abuse of notation, identify the integral
    \begin{align*}
    I(a) = \frac{1}{(2\pi)^n} \int e^{i(x- y)\xi} a(x, y, \xi) \d[\xi] \in S'(\R^n\times \R^n)
    \end{align*}
    to be the \textit{Schwartz Kernel} of the operator $I(a): S(\R^n) \to S'(\R^n)$. 
\end{rem}




%%%%%%%%%%%%%%%%%%%%%%%%%%%%%%%%%%%%%%%%%%%%%%%%
%%%%%%%%%%%%%%%%%%%%%%%%%%%%%%%%%%%%%%%%%%%%%%%%
\pagebreak
\section{Properties of Symbols}
In this section, we shall establish the following summarising theorem: 

\begin{ftheorem}[Summary]
    Given $m \in \R$, $p, n \in \N$, then
    \begin{enumerate}
        \item $S^{m}_\infty(\R^{p}; \R^n)$ is a Frechet space, hence completely metrisable. 
        \item $S^{\infty}_\infty(\R^{p}; \R^n)$ is a graded commutative *-algebra over $\C$ with continuous  inclusion  $$\iota : S^{m}_\infty(\R^p; \R^n) \to S^{m'}_\infty(\R^p; \R^n)$$ for all $m \leq m'$. 
        \item $S^{-\infty}_\infty(\R^{p}; \R^n)$ is dense in $S^{m}_\infty(\R^{p}; \R^n)$ in the topology of $S^{m + \epsilon}_\infty(\R^{p}; \R^n)$ for any $\epsilon \in \R_{> 0}$. 
    \end{enumerate}

\end{ftheorem}

\begin{mdframed}
    \textbf{Exercise : } Show that symbol spaces are Frechet spaces. That is, show that the family of seminorms in \ref{eq : symbol seminorms} separates points and that if a sequence is Cauchy in each seminorm, then there exist a unique symbol where the sequence converges in each seminorm. 
\end{mdframed}

\subsection{$S^{\infty}_\infty(\R^{p}; \R^n)$ is a graded commutative *-algebra with continuous inclusion} 
We first prove continuous inclusion of lower order into higher order symbol space. 
\begin{fprop}
    Let $p, n \in \N$ be given and $\Omega \subset \R^p$ such that $\Omega \subset \overline{\mathrm{Int}(\Omega)}$. If $m, m' \in \R$ such that $m \leq m'$, then $S^m_\infty(\Omega; \R^n) \subset S^{m'}_\infty(\R^)$. Furthermore, the inclusion map 
    \begin{align*}
    \iota: S^m_\infty(\Omega; \R^n) \to S^{m'}_\infty(\Omega; \R^n)
    \end{align*}
    is continuous. 
\end{fprop}
\begin{proof}
    Let the real numbers $m \leq m'$ be given. We note that for any $\xi \in \R^n$
    \[
    \sym[\xi]^m \leq 1 \cdot \sym[\xi]^{m'}
    \]
    and thus if $a \in S^{m}_\infty(\Omega; \R^n)$, we have that $\forall \alpha \in \N^p, \forall \beta \in \N^n$
    \[
    \abs{D^\alpha_x D^\beta_\xi a(x, \xi)} \leq C \sym[\xi]^{m - \abs{\beta}} \leq C \sym[\xi]^{m' - \abs{\beta}} 
    \]
    which show that $a \in S^{m'}_\infty(\Omega; \R^n)$ as well. \\
    
    To show that $\iota$ is a continuous inclusion, it suffices to show that 
    \[
    \norm[\iota(a)]_{N, m'} \leq C \norm[a]_{N, m}
    \]
    for any $a \in S^{m}_\infty(\Omega; \R^n)$ and $N \in \N$. Indeed, this bound holds since 
    \[
    \frac{D^\alpha_x D^\beta_\xi a(x, \xi)}{\sym[\xi]^{m' - \abs{\beta}}}  \leq \frac{D^\alpha_x D^\beta_\xi a(x, \xi)}{\sym[\xi]^{m - \abs{\beta}}}. 
    \]
\end{proof}
\hfill \\ [3em]


\begin{fprop}
    Let $p, n \in \N$ be given. Let $\Omega \subset \R^p$ be such that $\Omega \subset \overline{\mathrm{Int}(\Omega)}$. Then, for any $m, m' \in \R$, we have 
    \[
    S^{m}_\infty(\Omega; \R^n) \cdot S^{m'}_\infty(\Omega; \R^n) = S^{m + m'}_\infty(\Omega; \R^n)
    \]
\end{fprop}
\begin{proof}
    Let $a \in S^{m}_\infty(\Omega; \R^n)$ and $b \in S^{m'}_\infty(\Omega; \R^n)$ be given. By (general) Leibinz formula, we have that for all multi-index $\alpha, \beta$, 
    \begin{align*}
    \sup_{(x, \xi) \in \mathrm{Int}(\Omega) \times \R^n} \frac{\abs{D^\alpha_x D^\beta_\xi a(x, \xi)b(x, \xi)}}{\sym[\xi]^{(m + m') - \abs{\beta}}} 
    &\leq  \sum_{\mu \leq \alpha, \gamma \leq \beta} \binom{\alpha}{\mu} \binom{\beta}{\gamma} \sup_{(x, \xi) \in \mathrm{Int}(\Omega) \times \R^n} \frac{\abs{D^\mu_x D^\gamma_\xi a(x, \xi)} \abs{D^{\alpha - \mu}_x D^{\beta - \gamma}_\xi b(x, \xi)}}{\sym[\xi]^{(m + m') - \abs{\beta}}} \\
    &\leq \sum_{\mu \leq \alpha, \gamma \leq \beta} \binom{\alpha}{\mu} \binom{\beta}{\gamma} C \sup_{\xi \in \R^n} \frac{\sym[\xi]^{m - \abs{\gamma}} \sym[\xi]^{m' - \abs{\beta - \gamma}}}{\sym[\xi]^{(m + m') - \abs{\beta}}} \\
    &= \sum_{\mu \leq \alpha, \gamma \leq \beta} \binom{\alpha}{\mu} \binom{\beta}{\gamma} C \sup_{\xi \in \R^n} \sym[\xi]^{\abs{\beta} - (\abs{\beta - \gamma} + \abs{\gamma})} \\
    & \leq \sum_{\mu \leq \alpha, \gamma \leq \beta} \binom{\alpha}{\mu} \binom{\beta}{\gamma} C \\
    &< \infty
    \end{align*}
    where we have use the property of multi-index that $\abs{\beta} = \abs{\beta - \mu} + \abs{\mu}$.  We have thus shown that $S^{m}_\infty(\Omega; \R^n) \cdot S^{m'}_\infty(\Omega; \R^n) \subset S^{m + m'}_\infty(\Omega; \R^n)$\\
    \\
    For the reverse inclusion, let $c \in S^{m + m'}_\infty(\Omega; \R^n)$ be given. Define 
    \begin{align*}
    a : (x, \xi) &\mapsto \sym[\xi]^m\\
    b: (x, \xi) &\mapsto \frac{c(x, \xi)}{a(x, \xi))}
    \end{align*}
    and observe that 
    \begin{itemize}
        \item $a \in S^{m}_\infty(\Omega; \R^n)$. It is clear that $a$ is smooth in both $x$ and $\xi$. It is independent of $x$ and thus any $x$ derivative gives 0. We need only to check that for all $\beta \in \N^n$, 
        \[
        \abs{D^\beta_\xi \sym[\xi]^m} \leq C \sym[\xi]^{m - \abs{\beta}}
        \]
        which can be proven by induction on $n$ and $\beta$. We shall only prove the base case where $n = 1$ and $\beta = 1$. We have 
        \begin{align*}
        \abs{D_\xi \sym[\xi]^m} = \abs{\p_\xi (1 + \xi^2)^{m/2}} = \abs{m\xi \sym[\xi]^{m - 2}} = \abs{m \frac{\xi}{\sym[\xi]}} \sym[\xi]^{m - 1} \leq \abs{m} \sym[\xi]^{m - 1}
        \end{align*}
        where we have used the fact that $\abs{\xi} \leq \sym[\xi]$ for all $\xi$. 
        \item $b \in S^{m'}_\infty(\Omega; \R^n)$. We note first that $\sym[\xi]^m \neq 0$ for all $\xi \in \R^n$ and thus $b$ is well-defined. Since division by $\sym[\xi]^m$ does not affect any of the $x$ derivative, we only need to show that for any $\beta \in \N^n$, we have
        \[
        \abs{D^\beta_\xi b(x, \xi)} \leq C \sym[\xi]^{m + m' - \abs{\beta}}
        \]
        for some constant $C > 0$ uniformly in $\xi$. Indeed, observe that by the Leibinz formula
        \begin{align*}
        \abs{D^\beta_\xi b(x, \xi)} 
        & \leq \sum_{\mu \leq \beta} \binom{\beta}{\mu} \abs{D^\mu_\xi c(x, \xi)} \abs{D^{\beta - \mu} \sym[\xi]^{-m}} \\
        & \leq C \sum_{\mu \leq \beta} \binom{\beta}{\mu} \sym[\xi]^{m + m' - \abs{\mu}} \sym[\xi]^{-m - \abs{\beta - \mu}} \\
        & \leq C \sum_{\mu \leq \beta} \binom{\beta}{\mu} \sym[\xi]^{ m' - (\abs{\mu} +  \abs{\beta - \mu})} \\
        & = C \sum_{\mu \leq \beta} \binom{\beta}{\mu} \sym[\xi]^{ m' -\abs{\beta}} \\
        &= C 2^{\beta} \sym[\xi]^{ m' -\abs{\beta}} 
        \end{align*}
        where we have use the definition of $c$ and applied the result proven for $a$ with $m \mapsto -m$. Thus, $b \in S^{m'}_\infty(\Omega; \R^n)$. 
    \end{itemize}
    It is clear that $a \cdot b = c$ and we have therefore shown that $S^{m + m'}_\infty(\Omega; \R^n) \subset S^{m}_\infty(\Omega; \R^n) \cdot S^{m'}_\infty(\Omega; \R^n)$. 
\end{proof}

The results above, together with the easily proven fact $ a^*(x, \xi) := \overline{a(x, \xi)} \in S^{m}_\infty(\R^{2n}; \R^n) \iff a \in S^{m}_\infty(\R^{2n}; \R^n)$, gives the desired algebraic structure for $S^{\infty}_\infty(\R^{2n}; \R^n)$



\subsection{Density of residual space, $S^{-\infty}_\infty(\R^{p}; \R^n)$ }

Next, we have a rather technical density result : the residual space, $S^{-\infty}_\infty(\R^{p}; \R^n)$, is dense in $S^{m}_\infty(\R^p; \R^n)$, but only with the topology of $S^{m + \epsilon}_\infty(\R^{p}; \R^n)$. The main strategy in this proof is to approximate any symbol with the very same symbol but cut off by a compactly supported function. As such, the main reason we \textbf{cannot} have density of $S^{-\infty}_\infty(\R^p; \R^n)$ in $S^m_\infty(\R^p; \R^n)$ is the same reason to the fact that Schwartz functions are not dense in the space of smooth bounded functions, in particular, $1 \in S^0_\infty(\Omega; \R^n)$ is not in the closure of $S^{-\infty}_\infty(\Omega; \R^n)$. \\

\begin{flemma}
    Given any $m \in \R$, $n, p \in \N$ and $a \in S^{m}_\infty(\R^p; \R^n)$, there exist a sequence in $S^{-\infty}_\infty(\R^p; \R^n)$ that is bounded in $S^{m}_\infty(\R^p; \R^n)$ and converges to $a$ in the topology of $S^{m + \epsilon}_\infty(\R^p; \R^n)$ for any $\epsilon \in \R_{> 0}$. 
    %In other words, for any $m \in \R$ and $\epsilon > 0$, $S^{-\infty}_\infty(\R^p; \R^n)$ is dense in $S^{m}_\infty(\R^p; \R^n)$ with the topology of $S^{m + \epsilon}_\infty(\R^p; \R^n)$.
\end{flemma}
\begin{proof}
    Let $a \in S^{m}_\infty(\Omega; \R^n)$ and $\epsilon \in \R_{>0}$ be given. Let $\chi \in C^\infty_c(\R^n)$ be a non-negative smooth cut-off function, i.e. $\chi \geq 0$  and satisfies
    \begin{align*}
    \chi(\xi) = 
    \begin{cases}
    1 & ,\abs{\xi} < 1 \\
    0 & ,\abs{\xi} > 2. 
    \end{cases}
    \end{align*}
    
    Then, for each $k \in \N$, we define
    \[
    a_k(x, \xi) = \chi\brac{\frac{\xi}{k}} a (x, \xi). 
    \]
    %    
    %    \begin{enumerate}
    %        \item $a_k \in S^{-\infty}_\infty(\Omega; \R^n)$ for all $k \in \N$; 
    %        \item $a_k$ are bounded in $S^{m}_\infty(\Omega; \R^n)$ for all $k \in \N$; 
    %        \item $a_k \to a$ as $k \to \infty$ in $S^{m + \epsilon}_\infty(\Omega; \R^n)$. 
    %    \end{enumerate}
    Now, given arbitrary $N, k \in \N$, observe that 
    \[
    \abs{a_k} \leq C \sym[\xi]^{-N} 
    \]
    since $a_k$ is compactly supported in $\xi$ (as $\chi$ is compactly supported). Furthermore, by Leibinz formula and the symbol estimates on $a \in S^{m}_\infty(\R^p; \R^n)$, we have
    \begin{align*}
    \abs{D^\alpha_x D^\beta_\xi a_k(x, \xi)} 
    & \leq \sum_{\mu \leq \beta} \binom{\beta}{\mu}k^{-\abs{\mu}} \brac{D^\mu_\xi \chi}\brac{\frac{\xi}{k}} \abs{D^\alpha_x D^{\beta - \mu}_\xi a(x, \xi) } \\
    & \leq C \sum_{\mu \leq \beta} \binom{\beta}{\mu}k^{-\abs{\mu}} \brac{D^\mu_\xi \chi}\brac{\frac{\xi}{k}} \sym[\xi]^{m - \abs{\beta - \mu}}. 
    \end{align*}
    Since $\chi$ and all its derivatives are compactly supported, each term above is bounded in $\xi$ and thus $a_k$ is bounded in $S^{m}_\infty(\R^p; \R^n)$ and that 
    \[
    \abs{D^\alpha_x D^\beta_\xi a_k(x, \xi)} \leq C' \sym[\xi]^{-N}
    \]
    which allow us to conclude that $a_k \in S^{-\infty}_\infty(\R^p; \R^n)$.\\
    \\
    It remains to show that $\Lim[k] a_k = a $ in $S^{m + \epsilon}_\infty(\Omega; \R^n)$. In the first symbol norm, we observe that, using the symbol estimate for $a$ 
    \begin{align*}
    \norm[a_k - a]_{0, m + \epsilon} 
    & = \sup_{(x, \xi) \in \R^p \times \R^n} \frac{\abs{a_k(x, \xi)}}{\sym[\xi]^{m + \epsilon}} \\
    & = \sup_{(x, \xi) \in \R^p \times \R^n} \frac{\abs{(1- \chi(\xi/k))} \abs{a(x, \xi)}}{\sym[\xi]^{m + \epsilon}} \\
    & \leq \norm[a]_{0, m} \sup_{\xi \in \R^n} \frac{\abs{(1- \chi(\xi/k))}}{\sym[\xi]^{\epsilon}} \\
    & \leq \norm[a]_{0, m} \sym[k]^{- \epsilon}\\
    & \to 0
    \end{align*}
    as $k \to \infty$, since $\abs{( 1- \chi(\xi/ k))}$ is 0 in the region $\abs{\xi} \leq k$ and bounded by $1$ otherwise.  We remark upon the necessity of the extra decay given by $\sym[\xi]^{-\epsilon}$ factor. For other symbol norms we shall again use Leibinz formula: 
    \begin{align*}
    \sup_{(x, \xi) \in \R^p \times \R^n} \frac{\abs{D^\alpha_x D^\beta_\xi a_k(x, \xi)}}{\sym[\xi]^{m + \epsilon - \abs{\beta}}} 
    & \leq \sup_{(x, \xi) \in \R^p \times \R^n} \frac{1}{\sym[\xi]^{m + \epsilon - \abs{\beta}}} \sum_{\mu \leq \beta} \binom{\beta}{\mu}k^{-\abs{\mu}} \brac{D^\mu (1 - \chi)}\brac{\frac{\xi}{k}} \abs{D^\alpha_x D^{\beta - \mu}_\xi a(x, \xi) } \\
    & \leq  \sup_{(x, \xi) \in \R^p \times \R^n} \frac{C}{\sym[\xi]^{m + \epsilon - \abs{\beta}}} \sum_{\mu \leq \beta} \binom{\beta}{\mu}k^{-\abs{\mu}} \brac{D^\mu (1 - \chi)}\brac{\frac{\xi}{k}} \sym[\xi]^{m - \abs{\beta - \mu}}  \\
    & = C \sup_{\xi \in \R^n} \sum_{\mu \leq \beta} \binom{\beta}{\mu}k^{-\abs{\mu}} \brac{D^\mu (1 - \chi)}\brac{\frac{\xi}{k}} \sym[\xi]^{- \epsilon - \abs{\mu}}  \\
    & \leq C' k^{- \epsilon} \\
    & \to 0
    \end{align*}
    as $k \to \infty$ by the same argument as before. Thus, we have proven that $a_k \to a$ as $k \to \infty$ in $S^{m + \epsilon}_\infty(\R^p; \R^n)$. 
    
\end{proof}


%%%%%%%%%%%%%%%%%%%%%%%%%%%%%%%%%%%%%%%%%%%%%%%%%
%%%%%%%%%%%%%%%%%%%%%%%%%%%%%%%%%%%%%%%%%%%%%%%%%
\pagebreak
\section{Quantisation}
\subsection{Continuity of $I(a) : S(\R^n) \to S'(\R^n)$}
We first note that, if $m < -n$ (write $m = -n - \epsilon$ for some $\epsilon > 0$), the oscillatory integral \ref{eq: oscillatory integral}, is absolutely convergent and defines a continuous linear operator 
\begin{align*}
I: S^{-n -\epsilon}_\infty(\R^{2n}; \R^n) &\to S'(\R^{2n}) \\
 a & \mapsto I(a) : S(\R^{2n}) \ni \varphi \mapsto I(a)(\varphi) = \frac{1}{(2\pi)^n} \int e^{i(x- y)\xi} a(x, y, \xi) \varphi(x, y) \d[\xi] \d[x] \d[y]. 
\end{align*} 
The map above is clearly linear. Continuity comes from the bound given by the following computation: $\forall M \in \N$, $\forall a \in S^{-n -\epsilon}_\infty(\R^{2n}; \R^n)$, $\forall \varphi \in S(\R^n)$ 
\begin{align*}
\abs{I(a)(\varphi)}
& \leq \frac{1}{(2\pi)^n} \int \abs{a(x, y, \xi) \varphi(x, y)} \d[\xi] \d[x] \d[y]\\
& \leq \frac{\norm[a]_{0, -n -\epsilon}}{(2\pi)^n} \int \sym[\xi]^{-n -\epsilon} \sym[(x, y)]^{-M} \sym[(x, y)]^{M} \abs{\varphi(x, y)} \d[\xi] \d[x] \d[y]\\
& \leq \frac{\norm[a]_{0, -n -\epsilon} \norm[\varphi]_M}{(2\pi)^n} \int \sym[\xi]^{-n -\epsilon} \sym[(x, y)]^{-M}  \d[\xi] \d[x] \d[y]\\
\end{align*}
for any $M \in \N$, where 
\begin{align} \label{eq: Schwartz seminorm} 
\norm[\varphi]_M :=  \sum_{\abs{\alpha} \leq M}  \sup_{(x, y) \in \R^{2n}} \sym[(x, y)]^M \abs{D^{\alpha}_{x, y} \varphi(x, y)}
\end{align}
is the Schwartz  seminorm on $S(\R^{2n})$. If we choose $M \geq 2n +1$, the $x, y$ integrals are convergent and since $m = -n -\epsilon < -n$, the $\xi$ integral converges as well, hence we have 
\begin{align*}
\abs{I(a)(\varphi)} \leq C \norm[a]_{0, m} \norm[\varphi]_M
\end{align*}
which implies continuity. \\

The proposition below extend this result to general $m \in \R$. 
\begin{fprop}
    The continuous linear map 
    \begin{align*}
    I : S^{-\infty}_\infty(\R^{2n}; \R^n) \to S'(\R^{2n}) \\
    \end{align*}
    extends uniquely to a linear map 
    \begin{align*}
    I: S^{m}_\infty(\R^{2n}; \R^n) \to S'(\R^{2n})
    \end{align*}
    which is continuous as linear map from $S^{m'}_\infty(\R^{2n}; \R^n)$ to $S'(\R^{2n})$ for arbitrary $m \in \R$ and $m ' > m$. 
\end{fprop}
\begin{proof}
    Let $m, m' \in \R$, $n \in \N$ with $m < m' $ be given. For any $a \in S^{m}_\infty(\R^{2n}; \R^n)$, the density of $S^{-\infty}_\infty(\R^{2n}; \R^n)$ in  $S^{m}_\infty(\R^{2n}; \R^n)$ with the topology of $S^{m'}_\infty(\R^{2n}; \R^n)$ means that there exist a sequence $a_k \in S^{-\infty}_\infty(\R^{2n}; \R^n)$ so that $a_k \to a \in S^{m'}_\infty(\R^{2n}; \R^n)$. Together with the completeness of $S'(\R^{2n})$ (being a continuous linear map into $\C$ which is complete), we have unique continuous linear extension of $I : S^{-\infty}_\infty(\R^{2n}; \R^n) \to S'(\R^{2n})$ to $S^{m}_\infty(\R^{2n}; \R^n)$ given by
    \begin{align*}
    I(a) := \lim_{k \to \infty}I(a_k)
    \end{align*}
    which is continuous in the $S^{m'}_\infty(\R^{2n}; \R^n)$ topology. Therefore, it is enough to show that for any $a \in S^{-\infty}_\infty(\R^{2n}; \R^n) $ and $\varphi \in S(\R^{2n})$, there exist $N, M \in \N$, such that 
    \begin{align*}
    \abs{I(a)(\varphi)} \leq C \norm[a]_{N, m'} \norm[\varphi]_M. 
    \end{align*}
    
    Let $a, \varphi$ as above be given. Note that 
    \begin{align*}
    e^{i(x - y)\xi} = \sym[\xi]^{-2q} \brac{1 + \xi \cdot D_x}^q e^{i (x - y) \xi } = \sym[\xi]^{-2q} \brac{1 - \xi \cdot D_y}^q e^{i (x - y) \xi }. 
    \end{align*}
    Thus, using integration by parts, for any $q \in \N$, 
    \begin{align*}
    I(a)(\varphi) 
    & = \frac{1}{(2\pi)^n} \int e^{i(x- y)\xi} a(x, y, \xi) \varphi(x, y) \d[\xi] \d[x] \d[y]\\
    & = \frac{1}{(2\pi)^n} \int \sym[\xi]^{-4q} \brac{1 - \xi \cdot D_y}^q \brac{1 + \xi \cdot D_x}^q e^{i (x - y) \xi } a(x, y, \xi) \varphi(x, y) \d[\xi] \d[x] \d[y]\\
    &= \frac{1}{(2\pi)^n} \int \sym[\xi]^{-4q} e^{i (x - y) \xi }  \brac{1 - \xi \cdot D_y}^q \brac{1 + \xi \cdot D_x}^q [a(x, y, \xi) \varphi(x, y)] \d[\xi] \d[x] \d[y]\\
    &=  \frac{1}{(2\pi)^n} \int \sym[\xi]^{-4q} e^{i (x - y) \xi } \brac{ \sum_{\abs{\gamma} \leq 2q} a_\gamma(x, y, \xi) D^\gamma_{x, y} \varphi(x, y)} \d[\xi] \d[x] \d[y]\\
    \end{align*}
    where 
    \begin{align*}
    a_\gamma(x, y, \xi) = \sum_{\abs{\mu}, \abs{\nu} \leq q} C_{\mu\nu} \xi^{\mu + \nu} D^\mu_x D^\nu_y a(x, y, \xi) 
    \end{align*}
    for some combinatorial constants $C_{\mu\nu}$. 
    Now, using the symbol estimate for $a \in S^{-\infty}_\infty(\R^{2n}; \R^n)$, and that $\abs{\mu} + \abs{\nu} \leq 2q$
    \begin{align*}
    \abs{a_\gamma(x, y, \xi)} 
    & \leq \sum_{\abs{\mu}, \abs{\nu} \leq q} C_{\mu\nu} \abs{\xi}^{\mu + \nu} \abs{D^\mu_x D^\nu_y a(x, y, \xi)} \\
    & = \sum_{\abs{\mu}, \abs{\nu} \leq q} C_{\mu\nu} \abs{\xi}^{\mu + \nu} \sym[\xi]^{m' } \frac{\abs{D^\mu_x D^\nu_y a(x, y, \xi)}}{\sym[\xi]^{m'}} \\
    & \leq \norm[a]_{2q, m'} \sym[\xi]^{m'} \sum_{\abs{\mu}, \abs{\nu} \leq q} C_{\mu\nu} \abs{\xi}^{\mu + \nu} \\
    &\leq \norm[a]_{2q, m'} \sym[\xi]^{m'} \sum_{\abs{\mu}, \abs{\nu} \leq q} C_{\mu\nu} \sym[\xi]^{\mu + \nu} \\    &\leq \norm[a]_{2q, m'} \sym[\xi]^{m'} \sym[\xi]^{2q}  \sum_{\abs{\mu}, \abs{\nu} \leq q} C_{\mu\nu} \\
    & \leq C_q \norm[a]_{2q, m'} \sym[\xi]^{m' + 2q}
    \end{align*}
    and  since $\abs{\gamma} \leq 2q$, 
    \begin{align*}
    \abs{D^{\gamma}_{x, y} \varphi(x, y)}
    & = \sym[(x, y)]^{-2q - 2n - 1} \sym[(x, y)]^{2q + 2n + 1} \abs{D^{\gamma}_{x, y} \varphi(x,y)}\\
    & \leq \sym[(x, y)]^{-2q - 2n - 1} \sum_{\abs{\alpha} \leq 2q + 2n + 1} \sup_{(x, y) \in \R^{2n}} \sym[(x, y)]^{2q + 2n + 1} \abs{D^{\alpha}_{x, y} \varphi(x,y)}\\
    & \leq \sym[(x, y)]^{-2q - 2n - 1} \norm[\varphi]_{2q + 2n + 1}. 
    \end{align*}
    Bring together both bounds, we have 
    \begin{align*}
    \abs{I(a)(\varphi)} 
    & \leq \frac{1}{(2\pi)^n} \int \sym[\xi]^{-4q} \brac{ \sum_{\abs{\gamma} \leq 2q} \abs{a_\gamma(x, y, \xi) D^\gamma_{x, y} \varphi(x, y)}} \d[\xi] \d[x] \d[y]\\
    & \leq C' \norm[a]_{2q, m'}\norm[\varphi]_{2q + 2n + 1}  \int \sym[\xi]^{-4q}   \sym[\xi]^{m' + 2q} \sym[(x, y)]^{-2q - 2n - 1}  \d[\xi] \d[x] \d[y]\\
    & = C' \norm[a]_{2q, m'}\norm[\varphi]_{2q + 2n + 1}  \int    \sym[\xi]^{m' - 2q} \sym[(x, y)]^{-2q - 2n - 1}  \d[\xi] \d[x] \d[y]\\
    \end{align*}
    Thus, as long as $m' - 2q < -n$, i.e. $q > \max\brac{\frac{m' + n}{2}, 0}$, the integral above converges. Finally, set $N = 2q$, $M = 2q + 2n + 1$, we have 
    \begin{align*}
    \abs{I(a)(\varphi)} \leq C \norm[a]_{N, m'}\norm[\varphi]_{M} 
    \end{align*}
    as required. 
\end{proof}

\hfill \\ [3em]
By the Schwartz Kernel theorem, each $a \in S^{m}_\infty(\R^{2n}; \R^n)$ defines a continuous linear operator 
\begin{align*}
I(a) : S(\R^n) \to S'(\R^n).
\end{align*}
We can now define the space of $m$-order pseudo-differential operators as the space 
\begin{align*}
\Psi^{m}_\infty(\R^n) := \set{A = I(a) \wh a \in S^{m}_\infty(\R^{2n}; \R^n)}
\end{align*}
with the total space $\Psi^{\infty}_\infty(\R^n) := \cup_{m \in \R} \Psi^m_\infty(\R^n)$ and the residual space $\Psi^{-\infty}_\infty(\R^n) := \cap_m \Psi^m_\infty(\R^n)$ defined similarly. 
\subsection{Composition theorem}

In this section we shall prove that, just like symbol spaces, $\Psi^\infty_\infty(\R^n)$ forms a graded *-algebra. The difference being, this time, the algebra is \emph{non-commutative}. That is, we shall show that following theorem holds. 
\begin{ftheorem}[Summary] 
    Given $n \in \N$, $\Psi^\infty_\infty(\R^n)$ is a graded *-algebra over $\C$ with continuous inclusion 
    \begin{align*}
    \iota : \Psi^m_\infty(\R^n) \to \Psi^{m'}_\infty(\R^n)
    \end{align*}
    for any $m \leq m'$. 
\end{ftheorem}

We shall prove this theorem by first accumulate several technical lemmas, of which the most important is the reduction lemma that allow us remove the dependence of either $x$ or $y$ in the symbol $a(x, y, \xi) \in S^{m}_\infty(\R^{2n}; \R^n)$. 

\subsubsection{Asymptotic Summation}
Suppose we are given a sequence of symbols with decreasing order, $a_j \in S^{m-j}_\infty(\R^{p}; \R^n)$, $j \in \N$, we know that  $a_j(x, \xi)$ has ever higher rate of decay for large $\abs{\xi}$ with increasing $j$. However, the series $\sum_{j \in \N} a_j(x, \xi)$ need not converge. However, we have the following notion of asymptotic convergence. 
\begin{fdefinition}[Asymptotic summation] 
    A sequence of symbols with decreasing order, $a_j \in S^{m-j}_\infty(\R^{p}; \R^n)$, $j \in \N$ is \textbf{asymptotically summable} if there exist $a \in S^{m}_\infty(\R^{p}; \R^n)$ such that for all $N \in \N$, 
    \begin{align*}
    a - \sum_{j = 0}^{N -1} a_j \in S^{m-N}_\infty(\R^{p}; \R^n). 
    \end{align*}
    We write
    \begin{align*}
    a \sim \sum_{j \in \N} a_j. 
    \end{align*}
\end{fdefinition}
 
\begin{flemma}
    Every sequence of symbols with decreasing order is asymptotically summable.  Furthermore, the sum is unique up to an additive term in $S^{-\infty}_\infty(\R^{p}; \R^n)$. 
\end{flemma}
\begin{proof}[Sketch]
    Let $a_j \in S^{m-j}_\infty(\R^{p}; \R^n)$, $j \in \N$ be given. For uniqueness, suppose $a, a' \in S^{m}_\infty(\R^{p}; \R^n)$ are both asymptotic sums of the sequence. We need to show that $a - a' \in S^{-\infty}_\infty(\R^{p}; \R^n)$. Indeed, for any $N \in \N$, 
    \begin{align*}
    a - a' = \brac{a - \sum_{j = 0}^{N -1} a_j } - \brac{a' - \sum_{j = 0}^{N -1} a_j } \in S^{m -N}_\infty(\R^{p}; \R^n)
    \end{align*}
    since both terms on the right are elements of $S^{m -N}_\infty(\R^{p}; \R^n)$. Thus, 
    $$a - a' \in \cap_{n \in \N} S^{m -N}_\infty(\R^{p}; \R^n) = S^{-\infty}_\infty(\R^{p}; \R^n).$$ 
    \\
    For existence, we construct $a S^{m}_\infty(\R^{p}; \R^n)$ by Borel's method \cite{}. Let $\chi \in C^{\infty}_c(\R^p)$ be a bump function and define
    \begin{align*}
    a = \sum_{j \in \N} \brac{1 - \chi}(\epsilon_j \xi) a_j(x, \xi)
    \end{align*}
    where $\R_{> 0} \ni \epsilon_j \to 0 $ is a strictly monotonic decreasing sequence that converges to $0$. We note that the sequence is a finite sum for any input $(x, \xi)$ and hence define a smooth function. It remains to show that, for some choice of $\epsilon_j$ with sufficiently rapid decay, 
    \begin{align*}
    \sum_{j \geq N} \brac{1 - \chi}(\epsilon_j \xi) a_j(x, \xi) 
    \end{align*}
    converges in $S^{m -N}_\infty(\R^{p}; \R^n)$ for any $N \in \N$. 
    \\
    
    Note: This is again an exercise in using symbol seminorms and Leibniz formula.  
\end{proof}

\subsubsection{Reduction}
We will now show that $\Psi^{m}_\infty(\R^n)$ is exactly the range of $I : S^{m}_\infty(\R^{2n}; \R^n) \to S'(\R^{2n})$ restricted to $S^{m}_\infty(\R^{n}; \R^n) \subset S^{m}_\infty(\R^{2n}; \R^n)$. 
\begin{fdefinition}
    Let 
    \begin{align*}
    \pi_L : \R^{3n}_{x, y, \xi} \to \R^{2n}_{x, \xi}
    \end{align*}
    be the projection map $(x, y, \xi) \mapsto (x, \xi)$. We define the \textbf{left quantisation map} as 
    \begin{align*}
    q_L := I \circ \pi_L^* : S^{m}_\infty(\R^{n}; \R^n) \to \Psi^{m}_\infty(\R^{n}) 
    \end{align*}
    with elements $a_L \in S^{m}_\infty(\R^{n}; \R^n)$ known as the \textbf{left reduced symbols}.
    
\end{fdefinition}

\begin{flemma}[Reduction] 
    For any $a(x, y, \xi) \in S^{m}_\infty(\R^{2n}_{x, y}; \R^n_\xi)$ there exist unique $a_L(x, \xi) \in S^{m}_\infty(\R^{n}; \R^n)$ such that $I(a) = q_L(a_L) = I(a_L \circ \pi_L)$. Furthermore, with $\iota : \R^{2n} \ni (x, \xi) \mapsto (x, x, \xi) \in \R^{3n}$ being the diagonal inclusion map, we have 
    \begin{align}
    a_L(x, \xi) \sim \sum_{\alpha} \frac{i^{\abs{\alpha}}}{\alpha !} \iota^* D^\alpha_y D^\alpha_\xi a(x, y, \xi). 
    \end{align}
\end{flemma}
\begin{proof}[Sketch] 
    Note that 
    \begin{align*}
    D_\xi^\alpha e^{i(x- y) \xi} = (x - y)^\alpha e^{i(x - y) \xi} \implies I((x -y)^\alpha a) = I((-1)^{\abs{\alpha}D_\xi^\alpha a})
    \end{align*}
    where we have extended the identity that is true using integration by parts in $S^{-\infty}_\infty(\R^{2n}; \R^n)$ to general $S^{m}_\infty(\R^{2n}; \R^n)$ using the density result of symbol space. Now, if we Taylor expand $a$ around the diagonal $x =y$, we get 
    \begin{align*}
    a(x, y, \xi) = \sum_{\abs{\alpha} \leq N -1} \frac{(-i)^{\abs{\alpha}}}{\alpha !} (x - y)^\alpha D^\alpha_ya (x, x, \xi) + r_N(x, y, \xi)
    \end{align*}
    where 
    \begin{align*}
    r_N(x, y, \xi) = \sum_{\abs{\alpha} = N} \frac{(-i)^{\abs{\alpha}}}{\alpha !} (x - y)^\alpha \int_0^1 (1 - t)^{N -1} D^\alpha_ya(x, (1 -t)x + ty, \xi) \d[t]
    \end{align*}
    for any $N \in \N$.  Applying the identity above give us 
    \begin{align*}
    &I(a) = \sum_{j = 0}^{N -1} A_j + R_N \\
    &A_j = I\brac{\sum_{\abs{\alpha} = j} \frac{i^{\abs{\alpha}}}{\alpha !} D^\alpha_y D^\alpha_\xi a (x, x, \xi)} \in \Psi^{m - j}_\infty(\R^n)\\
    &R_N \in \Psi^{m -N}_\infty(\R^n)
    \end{align*}
    Asymptotic summation lemma give us 
    \begin{align*}
    b(x, \xi) \sim \sum_{\alpha} \frac{i^{\abs{\alpha}}}{\alpha !} D^\alpha_y D^\alpha_\xi a (x, x, \xi) \in S^{m}_\infty(\R^{n}; \R^n)
    \end{align*}
    so that $I(a) - I(b) \in \Psi^{-\infty}_\infty(\R^n)$. It remains to show that $A \in \Psi^{-\infty}_\infty(\R^n) \iff A = I(c), c \in S^{-\infty}_\infty(\R^{n}; \R^n)$
\end{proof}

%We shall show that the elements $a \in \sym[x - y]^w S^{m}_\infty(\R^{2n}_{x,y}, \R^n) $ acts on $S(\R^n)$ via the Schwartz kernel 
%\[
%I(a) =  \frac{1}{(2\pi)^n} \int e^{i(x - y)\xi} a(x, y, \xi) \d[\xi]. 
%\]
%\begin{fprop}
%    Let $n \in \N$ and $m, w \in \R$ with $m < -n$, then the map
%    \begin{align*}
%    I: \sym[x - y]^w S^{m}_\infty(\R^{2n}_{x,y}, \R^n) &\to (1 + \abs{x}^2 + \abs{y}^2) C^0_\infty(\R^{2n}) \\
%    a &\mapsto I(a) = \frac{1}{(2\pi)^n} \int e^{i(x - y)\xi} a(x, y, \xi) \d[\xi]
%    \end{align*}
%    extends by continuity to 
%    \begin{align*}
%    I: \sym[x - y]^w S^{m}_\infty(\R^{2n}_{x,y}, \R^n) \to S'(\R^{2n})
%    \end{align*}
%    in the topology of $S^{m + \epsilon}_\infty(\Omega; \R^n)$ for any $\epsilon \in \R_{>0}$. 
%\end{fprop}
%\begin{proof}
%    
%\end{proof}



%%%%%%%%%%%%%%%%%%%%%%%%%%%%%%%%%%%%%%%%%%%%%%%%%
%%%%%%%%%%%%%%%%%%%%%%%%%%%%%%%%%%%%%%%%%%%%%%%%%
\pagebreak
\section{Appendix: Functional Analysis}

\begin{ftheorem}[Continuous Linear extension]
    Let $T \in \L(V, W)$ be a continuous linear map between normed vector spaces $V$ and $W$ with $W$ completely metrisable. Then, there exist unique extension $\widetilde{T} \in \L(\widetilde{V}, W)$ of $T$, i.e. $\widetilde{T}|_V = T$ where $\widetilde{V}$ is the completion of $V$. 
\end{ftheorem}

\begin{ftheorem}
    Let normed vector spaces $V$, $W$ be given. If $W$ is complete, then  $\L(V, W)$ is complete. 
\end{ftheorem}

\begin{ftheorem}[Schwartz Kernel Theorem {\cite[Chapter 4.6, p.~345]{taylor_pde}}] 
    Let $M, N$ be compact manifold and 
    \begin{align*}
    T: C^\infty(M) \to \D'(N) \\
    \end{align*}
    be a continuous linear map ($C^\infty(M)$ being given Frechet space topology and $D'(N)$ the weak* topology). Define a bilinear map 
    \begin{align*}
    B: C^\infty(M) &\times C^\infty(N) \to \C \\
    (u, v) &\mapsto B(u, v) = \inprod[v, Tu]. 
    \end{align*}
    
    Then, there exist a distribution $k \in \D'(M \times N)$ such that for all $(u, v) \in C^\infty(M) \times C^\infty(N)$
    \begin{align*}
    B(u, v) = \inprod[u \otimes v, k]. 
    \end{align*}
    We call such $k$ the kernel of $T$. 
\end{ftheorem}

\begin{fdefinition}[Frechet space]
    
    
\end{fdefinition}
%\nocite{}
%\pagebreak 
%\bibliographystyle{acm}
\bibliography{\jobname}

\end{document}
