\documentclass[12pt]{article}
\usepackage{../thesis_style}

\title{Chapter 1: Functional analysis background}
\date{}

\begin{document}
\maketitle

This chapter serves to introduces concepts and theorems that are integral to the theory of microlocal analysis and its application to Fredholm problems for pseudodifferential operators. 

\paragraph{Some notations} \hfill \\
We will employ the following notations throught the rest of the paper. Let $\N = \set{0, 1, \dots}$ denote the set of natural numbers. Given $n \in \N$, a \textit{multi-index} is a $n$-tuple of natural numbers $\alpha = (\alpha_1, \alpha_2, \dots, \alpha_n) \in \N^n$. For any multi-indices $\alpha, \beta \in \N^n$ and  any $n$-tuples $x, y \in \R^n$, we define
\begin{align*}
x^\beta &:= x_1^{\beta_1} x_2^{\beta_2}\dots x_n^{\beta_n} = \prod_{j = 1}^n x_j^{\beta_j}\\
 (x + y)^\alpha &:= \prod_{j = 1}^n (x_j + y_j)^{\beta_j} \\
D^\alpha_x &:= D_{x_1}^{\alpha_1} D_{x_2}^{\alpha_2} \dots D_{x_n}^{\alpha_n}\\
\end{align*}
where $D_{x_j} := - i \p_{x_j}$with $i \in \C$ being the imaginary unit and $\p_{x_j} $ the $x_j$-partial derivative operator. Furthermore, we define
\begin{align*}
&\abs{\alpha} := \alpha_1 + \alpha_2 + \dots + \alpha_n\\
& \alpha! := \alpha_1! \alpha_2! \dots \alpha_n! \\
& \binom{\alpha}{\beta} := \binom{\alpha_1}{\beta_1} \binom{\alpha_2}{ \beta_2} \dots \binom{\alpha_n}{\beta_n} = \frac{\alpha!}{(\beta - \alpha)! \beta!} \\
&\alpha \leq \beta \iff \alpha_i \leq \beta_i, \quad \forall i \in \set{1, \dots, n}. 
\end{align*}

We shall now state the Leibinz formula, not only to illustrate the multi-index notation, but it will also be a theorem that we shall use repeatedly. 
\begin{ftheorem}[Leibinz formula]
    Let $f, g \in C^\infty(\R^n)$, then 
    \begin{align*}
    D^\alpha_x (fg) = (-i)^{\abs{\alpha}} \p_x^\alpha(fg) = (-i)^{\abs{\alpha}} \sum_{\beta \leq \alpha} \binom{\alpha}{\beta}  \brac{\p_x^\beta f} \brac{\p_x^{\alpha - \beta}g}. 
    \end{align*}
\end{ftheorem}

Next, in discussing the order of growth of smooth functions, $f = f(x)$ on $\R^n$ as $\norm[x] \to \infty$, it is often convenient to compare $f$ to another smooth function. Hence, we define the bracket 
\begin{align*}
\sym[\cdot] : \R^n &\to \R_{\geq 1} \\
x &\mapsto \sym[x] := \brac{ 1 + x_1^2 + x_2^2 + \dots + x_n^2}^{1/2} = \brac{1 + \norm[x]^2}^{1/2}. 
\end{align*}
The main point of this bracket is that,  $\sym[x]$ is a smooth function asymptotically equivalent to $\norm[x]$ for large $x$. 



%%%%%%%%%%%%%%%%%%%%%
%%%%%%%%%%%%%%%%%%%%%

\section{Schwartz functions and tempered distributions}

\begin{fdefinition}[Schwartz space]
    The space of Schwartz (test) functions of rapidly decaying functions on $\R^n$, denoted $\sch(\R^n)$, is the space of smooth functions $\varphi : \R^n \to \C$ such that for any $\alpha \in \N^n$, 
    \begin{align}
    \sup_{x \in \R^n} \abs{\sym[x]^k D^\alpha_x \varphi(x)} < \infty. 
    \end{align}

    
    We can define a countable family of seminorm on $\sch(\R^n)$ by 
    \begin{align}
    \norm[\varphi]_k := \max_{\abs{\alpha} \leq k} \sup_{x \in \R^n} \abs{\sym[x]^k D^\alpha_x \varphi(x)}
    \end{align}
    for $k \in \N$, $\varphi \in \sch(\R^n)$. This makes $\sch(\R^n)$ a Frechet space with metric 
    \begin{align*}
    d(\varphi, \psi) = \sum_{k \in \N} 2^{-k} \frac{\norm[\varphi - \psi]_k}{1 + \norm[\varphi - \psi]_k} 
    \end{align*}
    for any $\varphi, \psi \in \sch(\R^n)$, which defines a complete metric topology on $\sch(\R^n)$. 
\end{fdefinition}
\begin{rem} \hfill 
    \begin{enumerate}
        \item     We note that the space $\sch(\R^n)$ is non-empty since it contains all the compactly supported smooth functions $\chi : \R^n \to \C$. In fact, 
        \begin{align*}
        C^\infty_c(\R^n) = \set{u \in C^\infty(\R^n) \wh \exists C \in \R_{> 0}, \, \abs{x} > C \implies u(x) = 0} \subset \sch(\R^n)
        \end{align*}
        is a dense inclusion. 
        
        \item $\sch(\R^n)$ with pointwise multiplication and addition is a commutative algebra over $\C$ without identity since $1 \not\in \sch(\R^n)$. It is also closed under several useful elementary operations including coordinate multiplication and partial differentiation
        \begin{align*}
        x_j : \sch(\R^n) \to \sch(\R^n) \\
        D_{x_j} : \sch(\R^n) \to \sch(\R^n). 
        \end{align*}
        
    \end{enumerate}
\end{rem}

\begin{fdefinition}[Tempered distribution]
    The space of tempered distribution is the dual space of Schwartz space. More precisely, the space of tempered distribution $\sch'(\R^n)$ on $\R^n$ is defined by 
    \begin{align*}
    \sch'(\R^n) = \brac{\sch(\R^n)}' = \L(\sch(\R^n),\C)
    \end{align*}
    where $\L(V, W)$ denotes the continuous linear maps between any topological vector spaces $V \to W$. Explicitly in terms of seminorms on $\sch(\R^n)$, the elements $u \in \sch'(\R^n)$ are linear functionals $u : \sch(\R^n) \to \C$ satisfying: for any $\varphi \in \sch(\R^n)$, there exist $k \in \N$ and $C \in \R_{> 0}$ such that 
    \begin{align*}
    \abs{u(\varphi)} \leq C \norm[\varphi]_k. 
    \end{align*}
    We usually equip $\sch'(\R^n)$ with the weak-* topology, i.e. the weakest topology for which all linear maps of the form $\inprod[\varphi, \cdot] : \sch'(\R^n) \to \C$ for any $\varphi \in \sch(\R^n)$ are continuous. Here,  $\inprod : \sch(\R^n) \times \sch'(\R^n) \to \C$ is the Frechet space pairing defined by
    \begin{align*}
    \inprod[\varphi, u] := u(\varphi). 
    \end{align*}
    A neighbourhood basis around $0 \in \sch'(\R^n)$ for the topology is given by the collection of sets of the form
    \begin{align*}
    \set{u \in \sch'(\R^n) \wh \abs{u(\varphi_j)} < \epsilon_j, \, \varphi_j \in \sch(\R^n), \epsilon_j \in \R_{> 0}, \, \, j = 1, \dots, N}
    \end{align*}
    for any $N \in \N$. 
\end{fdefinition}

The following two standard results are important in the development of pseudodifferential calculus. The first of which allows us to extend results concerning continuous linear maps on $\sch(\R^n)$ to $\sch'(\R^n)$. 
\begin{flemma}
    Let $\iota : \sch(\R^n) \to \sch'(\R^n)$ be the injection map defined by the integral pairing 
    \begin{align*}
    \iota(\varphi)(\psi) = \int \varphi(x) \psi(x) \d[x] \in \C
    \end{align*}
    for any $\varphi, \psi \in \sch(\R^n)$. Then, the image of $\iota$ is dense in $\sch'(\R^n)$ with the weak-* topology. 
\end{flemma}    

The second result is the celebrated Schwartz kernel theorem. To motivate this theorem, observe that any element $k \in \sch'(\R^{n + m})$ defines a continuous linear \textit{operator} of the form 
\begin{align*}
A_k : \sch(\R^m) &\to \sch'(\R^n)\\
\varphi(x) &\to A_k \varphi : \psi(y) \in \sch(\R^n) \mapsto A_k \varphi (\psi) := k(\varphi(x) \psi(y)). 
\end{align*}
The Schwartz kernel theorem states that the converse is also true. 
\begin{ftheorem}[Schwartz kernel theorem]
    Let $m, n \in \N$ be given. Then, $A : \sch(\R^m) \to \sch'(\R^n)$ is a continous linear operator if and only if there exist unique $k \in \sch'(\R^{n + m})$ such that 
    \begin{align*}
    A\varphi(\psi) = k(\varphi \cdot \psi)
    \end{align*}
    for any $\varphi \in \sch(\R^m)$ and $\psi \in \sch(\R^n)$. 
\end{ftheorem}

We call the unique tempered distribution $k \in \sch'(\R^{n + m})$ representing the operator $A: \sch(\R^m) \to \sch'(\R^m)$ the Schwartz kernel of $A$. 


%\begin{ftheorem}[Schwartz Kernel Theorem {\cite[Chapter 4.6, p.~345]{taylor_pde}}] Let $M, N$ be compact manifold and 
%    \begin{align*}
%        T: C^\infty(M) \to \D'(N) \\
%    \end{align*}
%    be a continuous linear map ($C^\infty(M)$ being given Frechet space topology and $D'(N)$ the weak* topology). Define a bilinear map 
%    \begin{align*}
%        B: C^\infty(M) &\times C^\infty(N) \to \C \\
%        (u, v) &\mapsto B(u, v) = \inprod[v, Tu]. 
%    \end{align*}
%    
%    Then, there exist a distribution $k \in \D'(M \times N)$ such that for all $(u, v) \in C^\infty(M) \times C^\infty(N)$
%    \begin{align*}
%        B(u, v) = \inprod[u \otimes v, k]. 
%    \end{align*}
%    We call such $k$ the kernel of $T$. 
%\end{ftheorem}


\section{Fourier transform} 
Fourier transform plays a crucial role in the theory of pseudodifferential operator.
\begin{fdefinition}
    The Fourier transform $\F f$ of a function $f \in L^1(\R^n)$ is defined by the integral
    \begin{align*}
    \widehat{f}(\xi) := \F f(\xi) = \frac{1}{(2\pi)^{n/2}} \int e^{- i x \cdot \xi} f(x)\d[x] \in L^\infty(\R^n). 
    \end{align*}
\end{fdefinition}
    
We can verify that, on Schwartz space, the same integral operation defines a continuous linear map 
\begin{align*}
\F : \sch(\R^n) \to \sch(\R^n)
\end{align*}
Its $L^2$-adjoint is given by 
\begin{align*}
\F^* f(\xi) = \frac{1}{(2\pi)^{n/2}} \int e^{i x \cdot \xi} f(x)\d[x], \quad f \in \sch(\R^n). 
\end{align*}
The Fourier inversion theorem states that $\F^* $ is the continuous inverse, i.e. $\F^* = \F^{-1} : \sch(\R^n) \to \sch(\R^n)$. Explicitly, this means that for any $f \in \sch(\R^n)$ 
\begin{align*}
f(x) = \F^* \F f (x) = \frac{1}{(2\pi)^n} \int e^{i(x - y) \cdot \xi} f(y) \d[y] \d[\xi]. 
\end{align*}
With the inversion formula and the definition of $L^2$-dual, we can easily see that 
\begin{align*}
\inprod[f, g]_{L^2} = \inprod[f, \F^* \F g]_{L^2} = \inprod[\F f, \F g]_{L^2}, \quad f, g \in \sch(\R^n). 
\end{align*}
This allows us to uniquely extend $\F$, $\F^*$ from $\sch(\R^n)$ to a unitary map 
\begin{align*}
\F : L^2(\R^n) \to L^2(\R^n)
\end{align*}
with unitary  inverse $\F^*$. This is also known as the Plancherel theorem \cite{}. 

%%%%%%%%%%%%%%%%%%%%%
%%%%%%%%%%%%%%%%%%%%%


\section{Sobolev Spaces }
\cite[Chapter 4]{taylor_pde}

\begin{fdefinition}[Sobolev Spaces] Let $p \in \R$ and $n, k \in \N$ be given. We define the $k^{\text{th}}$-order $L^p$-based Sobolev space on $\R^n$ as the Banach space
    \begin{align*}
    W^{k, p}(\R^n) = \set{u \in L^p(\R^n) \wh D^\alpha u \in L^p(\R^n), \, \, \abs{\alpha} \leq k }
    \end{align*}
    with the norm
    \begin{align*}
    \norm[u]_{W^{p, k}} = \norm[u]_{L^p} + \sum_{j = 1}^k \norm[D^j u]_{L^p}. 
    \end{align*}
    For $p = 2$, we have denote $H^k := W^{k ,2}$ and note that result from Fourier analysis gives
    \begin{align*}
    H^k(\R^n) = \set{u \in L^2(\R^n) \wh \abrac{\xi}^k \hat{u} \in L^2(\R^n)}
    \end{align*}
    allowing us to extend the definition to each real order $s \in \R$, 
    \begin{align*}
    H^s(\R^n) = \set{u \in \sch'(\R^n) \wh \abrac{\xi}^s \hat{u} \in L^2(\R^n)} = \Lambda^{-s} L^2(\R^n) 
    \end{align*}
    where $\sch'(\R^n)$ is the space of tempered distribution on $\R^n$ and 
    $$\Lambda^s: \sch'(\R^n) \to \sch'(\R^n)$$
     being the operator defined by $\Lambda^su = \F^{-1}(\abrac{\xi}^s \hat{u})$. This forms a Hilbert space with inner product given by
    \begin{align*}
    \inprod[u, v]_{H^s} = \inprod[ \Lambda^su, \Lambda^sv]_{L^2}. 
    \end{align*}
\end{fdefinition}
\begin{rem}
    It is straightforward to show that the derivative operator $D_{x_j}$ is a continuous linear operator $D_{x_j}: H^s(\R^n) \to H^{s - 1}(\R^n)$ and thus by induction, for any multi-index $\alpha \in \N^n$, $D^{\alpha} : H^s(\R^n) \to H^{s - \abs{\alpha}}(\R^n)$. 
\end{rem}

It can be proven that \ref{} the definition above can be extended to any smooth compact manifold $M$. 
\begin{fdefinition}
    Let $M$ be a smooth compact manifold and 
    \begin{align*}
    \mathcal{D}'(M) = \brac{C^\infty(M)}'
    \end{align*}
    be the space of distribution. Then, the Sobolev space $H^s(M)$ for $s \in \R$, is the set of distribution $u \in \mathcal{D}'(M)$ satisfying
    \begin{align*}
    (\chi u) \circ \Phi_U^{-1}  \in H^{s}(\R^n)
    \end{align*}
    for any coordinate domain $U \subset M$, chart $\Phi_U: U \to \R^n$ and any compactly supported smooth function $\chi \in C^\infty_c(U)$. 
\end{fdefinition}

%\begin{fprop}[Sobolev Embedding theorem]
%    Let $s \in \R$, $n \in \N$ be given. If $s > n /2$, then every $u \in H^s(\R^n)$ is bounded and continuous.     
%\end{fprop}
%




\todo{corollary in Hilbert space}
%\todo{having already assumed that $\Box - iQ$ has closed range}
%\begin{flemma}
%    For any $s \in \R$, the maps 
%    \begin{align*}
%    &\Box - iQ : \mathcal{X}^s \to H^{s - 1}(M) \\
%    &\Box + iQ^* : H^{s - 1}(M)^* = H^{1 -s}(M) \to \brac{\mathcal{X}^{s}}^*
%    \end{align*}
%    satisfies 
%    \begin{align*}
%    \coker (\Box - iQ) \subset \ker (\Box + iQ^*). 
%    \end{align*}
%\end{flemma}
%\begin{proof}
%    First note that the element of the cokernel
%    \begin{align*}
%    \coker (\Box - iQ) = H^{s -1}(M) / (\Box - iQ)(\mathcal{X}^s)
%    \end{align*}
%    are precisely the elements $u \in H^{s - 1}(M)$ orthogonal to the image of $\Box - iQ$, i.e. $\inprod[u, v]_{\mathcal{X}^s} = 0$, $\forall v \in (\Box - iQ)(\mathcal{X}^s)$. It remains to show that $(\Box + iQ^*)(u) = 0$ for any $u \in  $
%    
%    \begin{align*}
%    \inprod[v, (\Box + iQ^*) u] = \inprod[(\Box - iQ)v, u] = 0. 
%    \end{align*}
%    
%\end{proof}

%%%%%%%%%%%%%%%%%%%%%
%%%%%%%%%%%%%%%%%%%%%
\section{Compact and Fredholm operators}
    
In this section we shall restrict our attention to just maps between Banach spaces. A compact operator between Banach spaces is one where the image of all bounded sets are precompact. We shall denote the set of all compact (continuous) operators between $V$ and $W$ as $\K(V, W) \subset \L(V, W)$. Some useful results regarding linear maps between Banach spaces are given below. 
%These results will be used in chapter \ref{}. 


%\begin{flemma} Let $V$, $W$ be Banach spaces. 
%    \begin{enumerate}
%        \item $\K(V, W)$ is a closed linear subspace in $\L(V, W)$ in the (operator-)norm topology, i.e. $\K$ is closed and closed under linear combination. 
%        \item If $T \in \L(V, W)$ and $T(V)$ is finite dimensional, then $T$ is compact. 
%        \item If $T \in \K(V, W)$ then $T' \in \K(W', V')$. 
%    \end{enumerate}
%\end{flemma}

%\begin{flemma}
%    Let $X$ be a Banach space and $M \leq X$ a closed linear subspace. Define the perpendicular subspace $M^\perp \subset X'$ to be the set of linear functionals on $X$ that annihilate $M$, i.e. 
%    \begin{align*}
%    M^\perp := \set{\omega \in X' \wh \omega(M) = 0}. 
%    \end{align*}
%    Then, we have topological isomorphisms
%    \begin{align*}
%    M \cong X' / M^\perp \\
%    M^\perp \cong X / M. 
%    \end{align*}
%\end{flemma}

\begin{flemma} \ref{taylorpde}
    Let $T: X \to Y$ be a bounded linear map between Banach spaces $X, Y$ and let $T' : Y' \to X'$ be the dual linear map. Then
    \begin{enumerate}
        \item $\ker T' = T(X)^\perp = \set{\omega \in Y' \wh \omega(Tx) = 0, \forall x \in X}$. 
        \item If $T$ has closed range, then $T'(Y')  =\ker T'$.  
    \end{enumerate}
\end{flemma}
\begin{proof}
    Let $\inprod[x, \omega] = \omega(x)$ denote the pairing between a Banach space with its dual. The dual map $T'$ of $T$ is characterised by $\inprod[Tx, \omega] = \inprod[x, T'\omega]$ for any $x \in X$, $\omega \in Y'$. Observe that 
    \begin{align*}
    \omega \in T(X)^\perp 
    & \iff \forall x \in X, \, \inprod[Tx, \omega] = 0 \\
    & \iff \forall x \in X, \, \inprod[x, T' \omega] = 0 \\
    & \iff \omega \in \ker T'
    \end{align*}
    which proves the first statement. \\
    
    For the second statement, if $T(X)$ is a closed linear subspace of $Y$, then
    \begin{align*}
    \widetilde{T} : Y / \ker T \to T(X)\\
    [y] \mapsto T(y)
    \end{align*}
    defines a topological isomorphism, which in turns give rise to the topological isomorphism 
    \begin{align*}
    \widetilde{T}' : T(X)' \to (X / \ker T)'.
    \end{align*}
    We also know that $(X/ker T)' \cong (\ker T)^\perp$ are naturally isomorphic as Banach spaces. There is also a natural projection $p : Y' \to T(X)'$. We can then express $T': Y' \to X'$ as the composition
    \begin{equation*}
    \begin{tikzcd}
    Y' \ar[r, "p"] & T(X)' \ar[r, "\widetilde{T}"] & (X / \ker T)' \ar[r, "\sim"]& (\ker T)^\perp \\
    \end{tikzcd}
    \end{equation*}
    which gives the desired $T'(Y') = (\ker T)^\perp$. 
    \todo{verify proof!!} 
    
\end{proof}


\begin{ftheorem} 
    Let $V$, $W$, $Y$ be Banach spaces, $T \in \L(V, W)$ and $K \in \K(V, Y)$. If for all $u \in V$, the estimate 
    \begin{align*}
    \norm[u]_V \leq C \brac{\norm[Tu]_W + \norm[Ku]_Y}
    \end{align*}
    holds for some positive real constant $C \in \R_{> 0}$, then the image, $T(V)$ is closed, and has finite dimensional kernel. 
    \todo{Add in and prove finite dim kernel part of the statement!} 
\end{ftheorem}
\begin{proof}
    Let $\set{Tu_n \in T(V) \wh n \in \N, u_n \in V}$ be a convergent sequence in $T(V)$ with limit $w \in W$. We need to show that there exist $v \in V$ such that $T v = w$. Let $L = \ker T$. There are two cases 
    
    \begin{case}[ {$\forall n \in \N, \, d(u_n, L) \leq a < \infty$}] \hfill \\
        By definition of distance of a point to a set, for each $n$ there exist $x_n \in L$ such that $\norm[u_n - x_n] \leq 2a$. We can therefore define, for each $n$, $v_n = u_n - x_n$. Note that $\norm[v_n] \leq 2a$ and $\Lim Tv_n = \Lim T u_n + Tx_n = \Lim Tu_n + 0 = w$. Since the sequence $v_n$ is bounded and $K$ is compact, there exist a subsequence $\set{v_{n_j}}_{j \in \N}$ such that $Kv_{n_j} \to y_0 \in Y$. Then, applying the estimate on $v_{n_j} - v_{n_{j + k}}$, we get, as $j \to \infty$
        \begin{align*}
        \norm[v_{n_j} - v_{n_{j + k}}]_V 
        & \leq C \brac{\norm[Tv_{n_j} - Tv_{n_{j + k}}]_W + \norm[Kv_{n_j} - Kv_{n_{j + k}}]_Y} \\
        & \to  \brac{\norm[w - w]_W + \norm[y_0 - y_0]_Y} \\
        & = 0 
        \end{align*}
        which shows that $\set{v_{n_j}}_j$ is a Cauchy and therefore has a limit $v \in V$. Using continuity we get $w = \Lim Tv_{n} = \Lim[j] Tu_{n_j} = T \Lim[j] u_{n_j} = Tv$ as required. 
    \end{case} \hfill \\
    
    \begin{case}[ {$d(u_n, L) \to \infty $ as $n \to \infty$}] \hfill \\
        We can assume without loss of generality that $d(u_n, L) \geq 1, \forall n$. For each $n$, there exist $x_n \in L$ such that $1 \leq d(u_n, L) \leq \norm[v_n] \leq d(u_n, L) + 1$ where $v_n := u_n - x_n$. Define $w_n = v_n / \norm[v_n]$. Since $w_n$ is a bounded sequence (bounded by 1), there is a subsequence $Kw_{n_j}$ that converges, with limit $y_0 \in Y$. Furthermore, $T(w_n) = T(u_n - x_n) / \norm[u_n - x_n] \to 0$ as $n \to \infty$ since $\norm[u_n - x_n] \geq d(u_n, L) \to \infty$. Therefore, the estimate applied on $w_{n_j} - w_{n_{j + k}}$ gives
        \begin{align*}
        \norm[w_{n_j} - w_{n_{j + k}}]_V
        & \leq C \brac{\norm[Tw_{n_j} - Tw_{n_{j + k}}]_W + \norm[Kw_{n_j} - Kw_{n_{j + k}}]_Y} \\
        & \to  \brac{\norm[0 - 0]_W + \norm[y_0 - y_0]_Y} \\
        & = 0 
        \end{align*}
        as $j \to \infty$, showing that $\set{w_{n_j}}_j$ is a Cauchy sequence and therefore have a limit $w \in V$. But, $Tw = \Lim[j] Tw_{n_j} = 0 \implies w \in L$, yet 
        \begin{align*}
        d(w_n, L) 
        &= \inf_{x \in L} \norm[\frac{v_n}{\norm[v_n]} - x] \\
        &= \norm[v_n] \inf_{x \in L} \norm[v_n - x] \\
        &= \norm[v_n] \inf_{x \in L} \norm[u_n - x] \\
        &= \norm[v_n] d(u_n, L) \\
        &\geq 1
        \end{align*}
        implying that, in the limit as $n \to \infty$, $d(w, L) \geq 1$ which is a contraction. \\
        
    \end{case}


    To show that $T$ has finite dimensional kernel, it suffice to show that the closed unit ball in $\ker T$ is compact. Let $u_n \in \ker T$, $n \in \N$ be any sequence in the closed unit ball of $\ker T$, i.e. $\norm[u_n]_V \leq 1$, for all $n \in \N$. In otherwords,  $\set{u_n}_{n \in \N}$ is a bounded set (bounded by 1). Since $K$ is a compact operator,  the exist a subsequence indexed by $n_j$ for which $Ku_{n_j}$ converges in $Y$. Since convergent sequence is necessarily Cauchy, using the estimate provided we have
    \begin{align*}
    \norm[u_{n_j} - u_{n_{j + k}}]_V
     & \leq C \norm[T u_{n_j} - T u_{n_{j + k}}]_{W} + C \norm[K u_{n_j} - K u_{n_{j + k}}]\\
     & \leq C \cdot 0 + C \norm[K u_{n_j} - K u_{n_{j + k}}]\\
     & \to 0
    \end{align*}
    as $j \to \infty$ and any $k \in \N$. This shows that $\set{u_{n_j}}_j$ is a Cauchy sequence in the closed unit ball of $\ker T$ which is complete (being a closed subspace of a complete space). Hence, the original sequence $\set{u_n}_n$ has a convergent subsequence in the closed unit ball. We have just shown that every sequence in the closed unit ball of $\ker T$ has a convergent subsequence, i.e. the closed unit ball is compact as required. 
\end{proof}

%\begin{ftheorem}
%    Let $X, Y$ be Hilbert spaces and $T: X \to Y \in \L(X, Y)$ be a continuous (therefore bounded) linear operator. Suppose $T$ satisfies 6
%    \begin{align*}
%        \forall u \in X, \quad \norm[u]_{X} &\leq C \brac{\norm[Tu]_Y + \norm[u]_Z} \\
%        \forall v \in Y, \quad \norm[v]_{Y} &\leq C' \brac{\norm[T^*v]_X + \norm[v]_{Z^*}}
%    \end{align*}
%    where $Z \Subset X$ and $Z^* \Subset Y$ are compact subsets, then $T$ is Fredholm, i.e. $T(X)$ is closed in $Y$ and both $\ker T, \coker T$ are finite dimensional. 
%\end{ftheorem}
%\begin{proof}[proof sketch] \hfill \\ 
%    
%    
%\end{proof}



\begin{flemma}[Riez's lemma]
    Let $X$ be a normed linear space. Given a non-dense subspace (in particular, proper subspaces) $Y \subset X$ and any $r \in (0, 1)$, then there exist $x \in X$ with $\norm[x] = 1$ such that 
    \begin{align*}
    \inf_{y \in Y} \norm[x - y] \geq r. 
    \end{align*}
\end{flemma}
\begin{proof}\hfill \\
    Let $x_0 \in \overline{Y}^c$ and $R = \inf_{y \in Y}\norm[y - x_0] > 0$. Given any $\epsilon > 0$ we can pick (by definition of $\inf$) a $y_0 \in Y$ such that 
    \begin{align*}
    \norm[y_0 - x_0] < R + \epsilon. 
    \end{align*}
    Take $\epsilon < R \frac{1 - r}{r}$ and define $x \in X$ to be
    \begin{align*}
    x = \frac{y_0 - x_0}{\norm[y_0 - x_0]}. 
    \end{align*}
    Observe that $\norm[x] = 1$ and 
    \begin{align*}
    \inf_{y \in Y} \norm[x - y] 
    &= \frac{1}{\norm[x_0 - y_0]} \inf_{y \in Y} \norm[y_0 - x_0 - y \norm[x_0 - y_0] ] \\
    &= \frac{1}{\norm[x_0 - y_0]} \inf_{y \in Y} \norm[x_0 - y] \tag*{since $\alpha y - y_0 \in Y$ for any scalar $\alpha$} \\
    &\geq \frac{R}{R + \epsilon} \\
    &\geq \frac{R}{R + R \frac{1 - r}{r}} \\
    &= r
    \end{align*}
    as required. 
\end{proof}



Riez's lemma gives us a clear distinction between finite and infinite dimensional Banach spaces. 
\begin{fcor}
    The closed unit ball in a Banach Space $X$ is compact iff $X$ is finite dimensional. 
\end{fcor}
\begin{proof}
    Let $X$ be a Banach space and $\overline{B}$ be closed unit ball. 
    \begin{case}[ $\Longleftarrow$ ] If $X$ is finite dimensional, it is isometrically isomorphic to $\R^n$ for some $n \in \N$, where, by Heine-Borel theorem, the closed unit ball is compact. 
    \end{case}
    
    
    \begin{case}[$\implies$] We will prove the contrapositive. Suppose, $X$ is infinite dimensional. Let $x_0 \in \p \overline{B}$ be an element in the unit sphere. For each $n \in \N$, we will use Riez Lemma to obtain a unit vector $x_n$ such that 
        $$\inf_{y \in \mathrm{span}\set{x_0, \dots, x_{n - 1}}} \norm[x_n - y] \geq \frac{1}{2}. $$
        It is clear that $\set{x_n \wh n \in \N}$ is a sequence of element in $\overline{B}$ that has no convergent subsequence. Therefore $\overline{B}$ is not compact. 
    \end{case}
\end{proof}




\end{document}