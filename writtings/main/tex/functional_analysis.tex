\documentclass{article}
\usepackage{../thesis_style}

\title{Functional Analysis}
\date{}

\begin{document}
\maketitle 
\section{Fundamental constructs} 
In this section, we shall define the spaces and objects that are foundational to the subject. We will denote a (ordered, normed) field by $\mathbbm{F}$, it will usually be either $\R$ or $\C$. 

\begin{fdefinition}[Normed Linear Space] Given $V$ a $\mathbbm{F}$-vector space, a norm on $V$ is a function $\norm : V \times V \to \R_{\geq 0}$ is a function satisfying the following properties
\begin{description}
\item[Positive definite] $\forall x \in V$, $\norm[x] \geq 0 $ and $ \norm[x] = 0 \iff x = 0$. 
\item[Homogeneous] $\forall x \in V$, $\forall \alpha \in \mathbbm{F}$, $\norm[\alpha x] = \abs{\alpha} \norm[x]$. 
\item[Triangle inequality] $\forall x, y \in V$, $\norm[x + y] \leq \norm[x] + \norm[y]$. 
\end{description}
A \textbf{Normed Linear Space} is a vector space equipped with a norm,  $(V, \norm)$. 
\end{fdefinition}
We note that the norm functional define a metric on the vector space, $V$ by sending $(x, y) \mapsto d(x, y) = \norm[x-y]$. This provides $V$ with a metric space topology. We can thus speak of the notion completeness with respect to this metric. 

\begin{fdefinition}[Banach space] A \textbf{Banach space} $(V, \norm)$ is a normed linear space that is complete in the topology induced by the norm. 
\end{fdefinition}

\begin{fdefinition}[Inner product space] Let $V$ be an $\mathbbm{F}$-vector space. An inner product on $V$ is a function $\inprod : V \times V \to \mathbbm{F}$ satisfying: 
\begin{description}
\item[Positive definite] $\forall x \in V$, $\inprod[x, x] \geq 0$ and $\inprod[x, x] = 0 \iff x = 0$. 
\item[Congugate symmetry] $\forall x, y \in V$, $\inprod[x, y] = \overline{\inprod[y, x]}$. 
\item[Linearity in second argument] $\forall x, y, z \in V$, $\forall a, b \in \mathbbm{F}$, $\inprod[x, ay + bz] = a \inprod[x, y] + b \inprod[x, z]. $
\end{description}
An \textbf{inner product space} is a vector space equipped with an inner product, $(V, \inprod)$. If the field is $\R$ or $\C$, we can defined the induced norm on $V$ is given by 
\begin{align*}
\norm[x] = \sqrt{\inprod[x, x]}
\end{align*}
which in turns define a metric space topology on $V$. 
\end{fdefinition}

\begin{fdefinition}[Hilbert space] A \textbf{Hilbert space} $(\mathcal{H}, \inprod)$ is an inner product space that is complete in the topology induced by $\inprod$. 
\end{fdefinition}

\begin{fdefinition}[Locally convex space]
\end{fdefinition}

\begin{fdefinition}[Frechet space]
\end{fdefinition}



\begin{fdefinition}[Dual spaces] Let $V$ be a vector space. 
\end{fdefinition}


%%%%%%%%%%%%%%%%%%%%%%%%%%%%%%%%%%%%%%%%%%%%%%%%%%%
\section{Examples and useful results}


\begin{flemma}[Equivalent condition for completeness] A normed linear space $(V, \norm)$ is complete if and only if all absolutely convergent series converges. 
\end{flemma}

\begin{flemma}[Orthogonal projection for Hilbert space]
\end{flemma}

\begin{flemma}[Orthogonal decomposition for Hilbert space]
\end{flemma}


\begin{ftheorem}[Riez Representation theorem]
\end{ftheorem}





\begin{mdframed}
\begin{example}[$\ell_p$ spaces] For each $p \in [1, \infty)$, we can define a $p$-norm on the space of sequences in $\C$ (or $\R$). Let $x = \set{x_n \in \C \wh n \in \N}$ be a sequence, then its $p$-norm is given by 
\begin{align*}
\norm[ x] = \brac{\sum_{n \in \N} \abs{x_n}^p}^{1/p}. 
\end{align*}
For $p = \infty$, we can define 
\begin{align*}
\norm[x]_\infty = \sup_{n \in \N} x_n
\end{align*}
The space of all sequences that has finite $p$-norm together with $\norm_p$ forms a Banach space which we denote $\ell_p$ spaces. 
\end{example}
\end{mdframed}

\begin{mdframed}
\begin{example}[$L^p(X, \mu)$] Given a measure space $(X, \mu)$ and $p \in [1, \infty)$ we can define a $p$-norm on the space of measurable functions $f: X \to \C$ (or $\R$) by 
\begin{align*}
\norm[f]_p = \brac{\int \abs{f}^p \, d\mu }^{1/p}. 
\end{align*}
For $p = \infty$, 
\begin{align*}
\norm[f]_\infty = \sup_{x \in X} \abs{f(x)}. 
\end{align*}
Then the space of measurable functions with finite $p$-norm forms a Banach space denoted $L^p(X, \mu)$ or just $L^p(X)$.  We note that $L^p(\N) = \ell_p$. 
\end{example}
\end{mdframed} 



\section{Linear operators}
Given any topological vector spaces $V$ and $W$, we can define the space of continuous linear operators $\L(V, W)$ to be the set of $T: V \to W$ that is continuous and linear. 
\begin{flemma}[Bounded iff continuous] Let $V$, $W$ be normed linear spaces and $T \in \L(V, W)$. Then the following are equivalent. 
\begin{enumerate}
\item $T$ is continuous. 
\item $T$ is continuous at $0$. 
\item $T$ is bounded, i.e. there exist $C > 0$ such that $\forall v \in V$, $\norm[T(v)]_W \leq C \norm[v]_V$. 
\end{enumerate}
\end{flemma}
The above lemma shows that the class of continuous operator is the same as the class of bounded linear operators. We shall henceforth use those terms interchangeably. There are other more restrictive class of operator that one might chance to study. 

\begin{fdefinition}[Compact operator] \label{def: compact operator}Let $V$, $W$ be normed linear spaces. A \textbf{compact operator} $T: V \to W$ is a continuous linear operator that maps all bounded sets to precompact sets. \\

There are several equivalent formulation that characterise compact operator. 
\begin{enumerate}
\item The image of the unit ball in $V$ is precompact in $W$. 
\item The image of any bounded sequence, $\set{x_n}_n$ in $V$, contain a convergent subsequence $\set{Tx_{n_j}}_j$ in $W$.
\item The exist a neighbourhood of 0 in $V$ that gets map to a subset of a compact neighbourhood of 0 in $W$. 
\item If $W$ is Banach, then image of any bounded set in $V$ is totally bounded in $W$. 
\end{enumerate}
\end{fdefinition}


Another interesting class of linear operators that is slightly more general than the class of continuous (and therefore bounded) linear operators called closed operators. 
\begin{fdefinition}[Closed operators] Let $V$, $W$ be Banach spaces and $T: D(T) \to W$ be a linear operator, where the domain is given by $D(T) \subset V$ . Then $T$ is closed if for any sequence $\set{x_n \in D(T)}_n$ that converges to $x \in D(T)$, and $Tx_n \to y \in W$ as $n \to \infty$, we have $Tx = y$. 
\end{fdefinition}

\begin{fdefinition}[Operator norm on $\L(V, W)$] If $V$, $W$ are normed linear spaces. Then we can define a norm on $\L(V, W)$ by 
\begin{align*}
\norm[T] = \sup \set{ \norm[T(v)]_W \wh v \in V, \, \norm[v]_V = 1}
\end{align*}
\end{fdefinition}


\begin{ftheorem}[Completeness of $\L(V, W)$] If $W$ is a Banach space (i.e. it is complete), then $\L(V, W)$ is complete with the operator norm. 
\end{ftheorem}

\begin{fdefinition}[Adjoint] Let $V$, $W$ be vectors space and $T \in \L(V, W)$ be a linear operator. Then we can define the \textbf{adjoint}(or the transpose), $T'$ of $T$ to be the unique linear map $T' \in \L(V', W')$ such that 
\begin{align*}
\inprod[Tv, \omega] = \inprod[v, T'\omega]
\end{align*}
for any $v \in V$ and $\omega \in W'$, where $\inprod$ is the natural pairing between a vector space $V$ with its dual $V'$ given by $\inprod[v, \omega] = \omega(v), \, \forall v \in V, \omega \in V'$. 
\end{fdefinition}




A corner stone result in functional analysis is the Baire category theorem which give sufficient condition for a topological space to be a Baire space: a space where countable union of nowhere dense set is nowhere dense. 
\begin{ftheorem}[Baire Category Theorem] In a complete metric (or completely metrisable) space, countable union of nowhere dense set is nowhere dense. 
\end{ftheorem}

Baire category theorem is vital in proving the following theorems. 
\begin{ftheorem}[Uniform boundedness principle] Let $V$ be Banach spaces and $W$ a normed vector space and $\set{T_\alpha \in \L(V, W)}_{\alpha \in A}$ be a family of continuous linear operators indexed by $A$. Then, 
\begin{align*}
\forall v \in V, \, \sup_{\alpha \in A} \norm[T(v)]_V < \infty \implies \sup_{\alpha \in A} \norm[T]_{\L(V, W)} < \infty. 
\end{align*}
\end{ftheorem}

\begin{ftheorem}[Open mapping theorem] A surjective continuous linear map between Banach spaces is an open map. Explicitly, if $V, W$ are Banach spaces and $T \in \L(V, W)$ is onto, then any neighbourhood $N$ of $0$ is map by $T$ onto a neighbourhood $T(N)$ of $0$. 
\end{ftheorem}
An immediate corollary to the open mapping theorem is that, if $T: V \to W$ is bijective, then it is a homeomorphism as well as an isomorphism of vector space, i.e. a topological isomorphism. 

\begin{ftheorem}[Closed graph theorem] If the graph of a linear map $T: V \to W$ between Banach spaces is closed in $V \oplus W$, then $T$ is continuous. 
\end{ftheorem}
The above result can be extended with minor modification to the statements to Fr\'echet spaces. \\

We will now turn our attention to the interaction between linear operators and their adjoint, and therefore the spaces and their duals. Recall that, in Hilbert spaces, we can define the orthogonal complement of a closed linear subspace. We do not, however, have the notion of orthogonality natively in a Banach space. A substitute is given below. 
\begin{fdefinition}[Perpendicular subspace] Let $V$ be a Banach space and $W \subset V$ be a linear subspace. We define the perpendicular subspace $W^\perp$ of $W$ to be the set of linear functional on $V$ that annihilate all elements of $W$, i.e. 
\begin{align*}
W^\perp = \set{ \omega \in V' \wh \omega(v) = 0, \, \forall v \in W}.
\end{align*}
We note that $W^\perp$ is a subspace of $V'$, rather than $V$. 
\end{fdefinition}

We shall state another important result for linear operators. 
\begin{ftheorem}[Closed range theorem] Let $V$, $W$ be Banach spaces and $T: D(T) \to W$ be a closed linear operator whose domain $D(T) \subset V$ is dense in $V$.  Denote the image of $T$ as $\im(T)$. Then the following are equivalent
\begin{enumerate}
\item $\im(T)$ is closed in $W$. 
\item $\im(T')$ is closed in $V'$. 
\item $\im(T) = \ker(T')^\perp$.
\item $\im(T') = \ker(T)^\perp$. 
\end{enumerate}
\end{ftheorem}


One is often concern about the invertibility of a linear operator on a space to itself (endomorphism). The following provides a result to that effect. 
\begin{flemma} Let $V$ be a Banach space and $T \in \L(V) = \L(V, V)$ be a continuous linear operator. If $\norm[T] < 1$, then $1 - T$ is invertible.  

\begin{proof} The core idea for this result is to use the geometric series, i.e. we shall prove that the series $\sum_n T^n$ converges to $(1 - T)^{-1}$. Since we have that $\norm[T] < 1$, we know that the series converges absolutely and therefore converges (the space is complete and we have used that $\norm[ST] \leq \norm[S]\norm[T]$ property of the operator norm. We can therefore show that the partial sums satisfies
\begin{align*}
\norm[(1 - T) \sum_{n = 1}^k T^n] = \norm[1 - T^{k + 1} ] \leq 1 + \norm[T]^{k + 1} \to 0
\end{align*}
as $k \to \infty$ as required. 
\end{proof}

\end{flemma}


%##################
\subsection{Compact and Fredholm Operators}
In this section we shall restrict our attention to just maps between Banach spaces (unless specified otherwise). As per definition (\ref{def: compact operator}), a compact operator between Banach spaces is one where the image of all bounded sets are precompact. We shall denote the set of all compact (continuous) operators between $V$ and $W$ as $\K(V, W) \subset \L(V, W)$. Some elementary result pertaining to compact operator are given below. 


\begin{flemma} Let $V$, $W$ be Banach spaces. 
\begin{enumerate}
\item $\K(V, W)$ is a closed linear subspace in $\L(V, W)$ in the (operator-)norm topology, i.e. $\K$ is closed and closed under linear combination. 
\item If $T \in \L(V, W)$ and $T(V)$ is finite dimensional, then $T$ is compact. 
\item If $T \in \K(V, W)$ then $T' \in \K(W', V')$. 
\end{enumerate}
\end{flemma}

\begin{ftheorem} Let $V$, $W$, $Y$ be Banach spaces, $T \in \L(V, W)$ and $K \in \K(V, Y)$. If for all $u \in V$, the estimate, 
\begin{align*}
\norm[u]_V \leq C \brac{\norm[Tu]_W + \norm[Ku]_Y}
\end{align*}
for some $C > 0$, then $T(V)$ is closed. 

\begin{proof}
Let $\set{Tu_n \in T(V) \wh n \in \N, u_n \in V}$ be a convergent sequence in $T(V)$ with limit $w \in W$. We need to show that there exist $v \in V$ such that $T v = w$. Let $L = \ker T$. 

\begin{case}[ {$\forall n \in \N, \, d(u_n, L) \leq a < \infty$}] By definition of distance of a point to a set, for each $n$ there exist $x_n \in L$ such that $\norm[u_n - x_n] \leq 2a$. We can therefore define, for each $n$, $v_n = u_n - x_n$. Note that $\norm[v_n] \leq 2a$ and $\Lim Tv_n = \Lim T u_n + Tx_n = \Lim Tu_n + 0 = w$. Since the sequence $v_n$ is bounded and $K$ is compact, there exist a subsequence $\set{v_{n_j}}_{j \in \N}$ such that $Kv_{n_j} \to y_0 \in Y$. Then, applying the estimate on $v_{n_j} - v_{n_{j + k}}$, we get, as $j \to \infty$
\begin{align*}
\norm[v_{n_j} - v_{n_{j + k}}]_V \leq C \brac{\norm[Tv_{n_j} - Tv_{n_{j + k}}]_W + \norm[Kv_{n_j} - Kv_{n_{j + k}}]_Y} \to  \brac{\norm[w - w]_W + \norm[y_0 - y_0]_Y} = 0 
\end{align*}
which shows that $\set{v_{n_j}}_j$ is a Cauchy and therefore has a limit $v \in V$. Using continuity we get $w = \Lim Tv_{n} = \Lim[j] Tu_{n_j} = T \Lim[j] u_{n_j} = Tv$ as required. 
\end{case} \hfill \\

\begin{case}[ {$d(u_n, L) \to \infty $ as $n \to \infty$}] We can assume without loss of generality that $d(u_n, L) \geq 1, \forall n$. For each $n$, there exist $x_n \in L$ such that $1 \leq d(u_n, L) \leq \norm[v_n] \leq d(u_n, L) + 1$ where $v_n := u_n - x_n$. Define $w_n = v_n / \norm[v_n]$. Since $w_n$ is a bounded sequence (bounded by 1), there is a subsequence $Kw_{n_j}$ that converges, with limit $y_0 \in Y$. Furthermore, $T(w_n) = T(u_n - x_n) / \norm[u_n - x_n] \to 0$ as $n \to \infty$ since $\norm[u_n - x_n] \geq d(u_n, L) \to \infty$. Therefore, the estimate applied on $w_{n_j} - w_{n_{j + k}}$ gives
\begin{align*}
\norm[w_{n_j} - w_{n_{j + k}}]_V \leq C \brac{\norm[Tw_{n_j} - Tw_{n_{j + k}}]_W + \norm[Kw_{n_j} - Kw_{n_{j + k}}]_Y} \to  \brac{\norm[0 - 0]_W + \norm[y_0 - y_0]_Y} = 0 
\end{align*}
as $j \to \infty$, showing that $\set{w_{n_j}}_j$ is a Cauchy sequence and therefore have a limit $w \in V$. But, $Tw = \Lim[j] Tw_{n_j} = 0 \implies w \in L$, yet $d(w, L) \geq 1/2$ which is a contraction. 
\end{case}




\end{proof}
\end{ftheorem}





%%%%%%%%%%%%%%%%%%%%%%%%%%%%%%%%%%%%%%%%%%%%%%%%%%%
\section{Sobolev Spaces \cite[Chapter 4]{taylor_pde}}
\begin{fdefinition}[Sobolev Spaces] Let $p \in \R$ and $k \in \N$ be given. We define the $k^{\text{th}}$-order $L^p$-based Sobolev spaces on $\R^n$ as the Banach space
\begin{align*}
W^{p, k}(\R^n) = \set{u \in L^p(\R^n) \wh D^\alpha u \in L^p(\R^n), \, \, \abs{\alpha} \leq k }
\end{align*}
with the norm
\begin{align*}
\norm[u]_{W^{p, k}} = \norm[u]_{L^p} + \norm[D^k u]_{L^p}. 
\end{align*}
For $p = 2$, we have denote $H^k := W^{2, k}$ and note that result from Fourier analysis gives
\begin{align*}
H^k = \set{u \in L^2(\R^n) \wh \abrac{\xi}^k \hat{u} \in L^2(\R^n)}
\end{align*}
allowing us to extend the definition to each real order $s \in \R$, 
\begin{align*}
H^s = \set{u \in S'(\R^n) \wh \abrac{\xi}^s \hat{u} \in L^2(\R^n)}
\end{align*}
where $S'(\R^n)$ is the space of tempered distribution on $\R^n$. This forms a Hilbert space with inner product given by
\begin{align*}
\inprod[u, v]_{H^s} = \inprod[ \Lambda^su, \Lambda^sv]_{L^2}
\end{align*}
with $\Lambda^s: S'(\R^n) \to S'(\R^n)$ being the operator $\Lambda^su = \F^{-1}(\abrac{\xi}^s \hat{u})$. 
\end{fdefinition}



%%%%%%%%%%%%%%%%%%%%%%%%%%%%%%%%%%%%%%%%%%%%%%%%%%%
\section{Linear Elliptic equations \cite[Chapter 5]{taylor_pde}}
\begin{ftheorem}[Elliptic estimate] Let $\overline{\Omega}$ be a compact Riemanian manifold and $\Omega$ be its interior. Let $L = -\Delta + X$, where $\Delta$ is the Laplacian and $X$ and first order differential operator with smooth coefficient in $\overline{\Omega}$. Then, we have the estimate
\begin{align*}
\norm[u]_{H^{k +1}}^2 \leq C \brac{\norm[Lu]_{H^{k -1}}^2 + \norm[u]_{H^k}^2}
\end{align*}
for all $u \in H^{k + 1}(\Omega) \cap H^1_0(\Omega)$ and for some $C > 0$. 
\end{ftheorem}






%%%%%%%%%%%%%%%%%%%%%%%%%%%%%%%%%%%%%%%%%%%%%%%%%%%
\section{Miscellaneous}
\begin{flemma}[Riez's inequality]
Let $X$ be a normed linear space. Given a non-dense subspace (or closed proper subspace) $Y \subset X$ and any $r \in (0, 1)$, then there exist $x \in X$ with $\norm[x] = 1$ such that 
\begin{align*}
\inf_{y \in Y} \norm[x - y] \geq r. 
\end{align*}


\begin{proof}\hfill \\
Let $x_0 \in \overline{Y}^c$ and $R = \inf_{y \in Y}\norm[y - x_0] > 0$. Given any $\epsilon > 0$ we can pick (by definition of $\inf$) a $y_0 \in Y$ such that 
\begin{align*}
\norm[y_0 - x_0] < R + \epsilon. 
\end{align*}
Take $\epsilon < R \frac{1 - r}{r}$ and define $x \in X$ to be
\begin{align*}
x = \frac{y_0 - x_0}{\norm[y_0 - x_0]}. 
\end{align*}
Observe that $\norm[x] = 1$ and 
\begin{align*}
\inf_{y \in Y} \norm[x - y] 
&= \frac{1}{\norm[x_0 - y_0]} \inf_{y \in Y} \norm[y_0 - x_0 - y \norm[x_0 - y_0] ] \\
&= \frac{1}{\norm[x_0 - y_0]} \inf_{y \in Y} \norm[x_0 - y] \tag*{since $\alpha y - y_0 \in Y$ for any scalar $\alpha$} \\
&\geq \frac{R}{R + \epsilon} \\
&\geq \frac{R}{R + R \frac{1 - r}{r}} \\
&= r
\end{align*}
as required. 
\end{proof}
\end{flemma}

Riez's lemma gives us a clear distinction between finite and infinite dimensional Banach spaces. 
\begin{mdframed}
\begin{cor}
The closed unit ball in a Banach Space $X$ is compact iff $X$ is finite dimensional. 
\end{cor}
\begin{proof}
Let $X$ be a Banach space and $\overline{B}$ be closed unit ball. 
\begin{case}[ $\Longleftarrow$ ] If $X$ is finite dimensional, it is isometrically isomorphic to $\R^n$ for some $n \in \N$, where, by Heine-Borel theorem, the closed unit ball is compact. 
\end{case}


\begin{case}[$\implies$] We will prove the contrapositive. Suppose, $X$ is infinite dimensional. Let $x_0 \in \p \overline{B}$ be an element in the unit sphere. For each $n \in \N$, we will use Riez Lemma to obtain a unit vector $x_n$ such that 
$$\inf_{y \in \mathrm{span}\set{x_0, \dots, x_{n - 1}}} \norm[x_n - y] \geq \frac{1}{2}. $$
It is clear that $\set{x_n \wh n \in \N}$ is a sequence of element in $\overline{B}$ that has no convergent subsequence. Therefore $\overline{B}$ is not compact. 
\end{case}
\end{proof}
\end{mdframed}




\bibliographystyle{apalike}
\bibliography{../main.bib}
\end{document}

