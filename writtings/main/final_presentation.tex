\documentclass{beamer}
\usepackage{./presentation_style}
\mode<presentation>{
    \usetheme{Madrid}
%    \usetheme{Pittsburgh}
%    \usetheme{metropolis}
}





% To remove the footer line in all slides 
\setbeamertemplate{footline}

% To replace the footer line in all slides with a simple slide count 
\setbeamertemplate{footline}[page number]

% To remove the navigation symbols from the bottom of all slides 
\setbeamertemplate{navigation symbols}{} 



% ---------- Title Page ----------------
 % The short title appears at the bottom of every slide, the full title is only on the title page
\title{Microlocal Analysis \\ 
    \large with Applications to Non-Elliptic Fredholm Problems}

\author{Edmund Lau \\
Supervised by: Dr Jesse Gell-Redman} 

% Your institution as it will appear on the bottom of every slide, may be shorthand to save space
\institute[Unimelb] {
    The University of Melbourne \\ % institution for the title page
    \medskip
    \textit{elau1@student.unimelb.edu.au} % email address
}

\date{19 October 2018} 

\begin{document}
    
% ------------------------------------------ Title Frame 
\begin{frame}
\titlepage 
\end{frame}


% ------------------------------------------ Content Overview 
\begin{frame}{Overview}
\tableofcontents
\end{frame}



%######################################### Section: Introduction
\section{Introduction} 
\begin{frame}{Introduction}

\end{frame}




%######################################### Section: Fredholm operators
\section{Fredholm and Elliptic operators} 
\begin{frame}{Fredholm Operators}

\begin{theorem}[Rank-nullity]
    If $T : V \to W$ be a linear operator between finite dimensional vector space $V$ and $W$, then 
    \begin{align*}
    \mathrm{Ind}(T) := \dim \ker T - \dim \coker T = \dim W - \dim V. 
    \end{align*}
\end{theorem}

\begin{definition}[Fredholm operators]
    A continuous linear operator $T : X \to Y$ between Banach spaces $X$ and $Y$ is Fredholm, if 
    \begin{itemize}
        \item $T$ has closed range, 
        \item $\ker(T)$ is finite dimensional, 
        \item $\coker(T)$ is finite dimensional. 
    \end{itemize}
\end{definition}
\end{frame}


\begin{frame}{Fredholm operators}
\begin{align*}
T x = y \quad x \in X, \, y \in Y
\end{align*}
\begin{itemize}
    \item $\dim \coker(T) < \infty $ means that solutions exist if and only if 
    $$\omega_1(y) = \omega_2(y) = \dots \omega_n(y) = 0$$ 
    where $\set{\omega_i}_{i = 1}^n$ is any basis of $\coker(T)$. \todo{with isomorphism to the perpendicular space}
    
    \item $\dim \ker(T) < \infty$ means that solution is unique up to addition of element in a finite dimensional space. 
    
    \item $T$ is invertible modulo compact operators (limits of finite rank operators). 
\end{itemize}

\end{frame}


\begin{frame}{Fredholm Index}
\begin{definition}
    The index of a Fredholm operator $T$ is defined by
    \begin{align*}
    \ind(T) := \dim \ker(T) - \dim \coker(T). 
    \end{align*}
\end{definition}
\begin{itemize}
    \item $\ind : \mathrm{Fred}(X, Y) \to \Z$ is a continuous map. 
    \item Unlike kernel and cokernel themselves, $\ind$ is well-behaved. \todo{example on the circle.}
    \item \todo{something about Atiyah-Singer index which only works for elliptic diff on compact manifolds} 
\end{itemize}


\end{frame}



\begin{frame}{Fredholm differential operators}
\begin{theorem} \label{theorem: fredholm estimates}
    Let $V$, $W$, $Y$ be Banach spaces, $T \in \L(V, W)$ and $K \in \K(V, Y)$. If for all $u \in V$, the estimate 
    \begin{align*}
    \norm[u]_V \leq C \brac{\norm[Tu]_W + \norm[Ku]_Y}
    \end{align*}
    holds for some positive real constant $C \in \R_{> 0}$, then the image, $T(V)$ is closed, and $T$ has finite dimensional kernel. 
\end{theorem}

Suppose we can show that the differential operator $P$ is Fredholm as a map 
\begin{align*}
P : H^{s} \to H^{s - m}
\end{align*}

\begin{align*}
Pu = f \quad f \in H^s(M), u \in \sch'
\end{align*}

\end{frame}
\todo{
Significance of fredholm operator: \\
Finite dimension of restrictions and non-injectivity. \\
Connection to regularity to solutions of PDE. (definition of Sobolev spaces). \\
The "Fredholm Estimates". \\
}
%######################################### Section: Fredholm operators
\section{``Elliptic Operators are Fredholm"} 
\begin{frame}{``Ellitptic Operators are Fredholm"}
\begin{example}[Laplacian]
    $\Delta: H^{s} \to H^{s - 2}$
    \begin{align*}
    \norm[u]_{H^{s + 2}} \leq C \brac{\norm[\Delta u]_{H^s} + \norm[u]_{L^2}}. 
    \end{align*}
\end{example}

\begin{proposition}
    Let $A \in \Psi^{m}_{\infty}(\R^n)$ be elliptic and $u \in H^{N}(\R^n)$ for some $N \in \R$. Then, for any $s \in \R$
    \begin{align*}
    Au \in H^{s}(\R^n) \implies u \in H^{s + m}(\R^n)
    \end{align*}
    and $u$ satisfies the estimates: $\exists C > 0$
    \begin{align*}
    \norm[u]_{H^{s + m}} \leq C \brac{\norm[Au]_{H^s} + \norm[u]_{H^N}}. 
    \end{align*}
\end{proposition}


\todo{
    Elliptic regularity estimate. (compact and non-compact manifold case). 
}

\end{frame}

%######################################### Section: 
\section{Non-elliptic Fredholm problem} 
\subsection{Propagation of singularity} 
\begin{frame}{Propagation of singularities}
\begin{theorem}[Propagation of singularities] \label{theorem: propagation of singularity estimates}
    Suppose we have 
    \begin{itemize}
        \item $P \in \Psi^k_{cl}(\R^n)$ a properly supported operator,
        \item $\sigma_{k}(P) = p - iq$ for real polyhomogeneous symbols $p, q \in S^{k}_{ph}(\R^{2n}; \R^{n})$, 
        \item $A, B, B' \in \Psi^{0}_{cl}(\R^n)$ compactly supported and $q \geq 0 $ on $\WF(B')$, 
        \item for all $(x, \xi) \in \WF(A)$, there exists $\sigma \geq 0$ such that for all $t \in [-\sigma, 0]$
        \begin{align*}
        \exp(-t\sym[\xi]^{1-k}H_p)(x, \xi) \in \Ell(B)
        \end{align*}
    \end{itemize}
    then for all $s, N \in \R$ and $u \in C^\infty(\R^n)$, there exist $C > 0$ such that 
    \begin{align*}
    \norm[Au]_{H^s} \leq C\brac{\norm[Bu]_{H^s} + \norm[B'Pu]_{H^{s - k + 1}} + \norm[u]_{H^{-N}}}. 
    \end{align*}
\end{theorem}
\end{frame}


\begin{frame}{Characteristic set of wave operator: the ``light cone"}
\begin{center}
    \begin{tikzpicture}
    \draw[->] (0, 3) -- (6, 3);
    \node at (6.2, 2.8) {$x$}; 
    \draw[->, thick] (0, 0) -- (0, 6);
    \node at (-0.2, 6.2) {$t$};  
    
    \node at (3, 2.6 ) {$x_0$}; 
    \draw (3, 2.9) -- (3, 3.1); 
    
    
    \draw [->] (1, 1) -- (5, 5); \node at (5.5, 5.5) {$\set{x - t = x_0}$}; 
    \draw [<-] (1, 5) -- (5, 1); \node at (1.5, 5.5) {$\set{x + t = x_0}$}; 
    \end{tikzpicture}
\end{center}
\end{frame}

\begin{frame}{Complex absoption}

\begin{center}
    \begin{tikzpicture}[scale=0.8]
    \draw [] (0, 0) rectangle (8, 8);
    \draw[pattern=north west lines, pattern color=blue, opacity=0.5] (0, 0) rectangle (8, 2.5);
    \draw[pattern=north west lines, pattern color=blue, opacity=0.5] (0, 5.5) rectangle (8, 8);
    
    \draw[pattern=north east lines, pattern color=red, opacity=0.5] (0, 2.25) rectangle (8, 5.75); 
    \draw [thick] (0, 2.25) -- (8, 2.25); 
    \draw [thick] (0, 5.75) -- (8, 5.75); 
    
    \draw  [dashed] (-2.5, 2) -- (8.1, 2);
    \draw  [dashed] (-2.5, 6) -- (8.1, 6);
    \draw  (-0.5, 5.5) -- (8.1, 5.5);
    \draw  (-0.5, 2.5) -- (8.1, 2.5);
    \node at (8.6, 2) {$\delta$}; 
    \node at (8.6, 6) {$1 - \delta$};
    \node at (8.7, 2.5) {$\delta + \delta'$}; 
    \node at (9, 5.5) {$1 - \delta - \delta'$};
    
    
    \node at (8.3, -0.1) {$x$};
    \node at (-0.1, 8.3) {$t$}; 
    \node at (-0.2, -0.2) {$0$};
    \node [fill=white]  at (4, 7) {$U_1 :=  \set{\chi(t) \neq 0}$ };
    \node [fill=white] at (4, 1) {$U_1$};
    \node [fill=white] at (4, 4) {$U_2 = $ a neighbourhood of $\set{\chi(t) = 0}$};
    
    
    \draw [->, thick] (-0.5, 0) -- (-3, 0); 
    \draw [->, thick] (-0.5, 0) -- (-0.5, 8);
    \draw [-, thick] (-2.5, 0.1) -- (-2.5, -0.1);
    \node at (-3.3, -0.3) {$\chi(t)$};
    \node at (-2.5, -0.3) {1}; 
    
    \draw (-2.5, 0) -- (-2.5, 2);
    \draw[rounded corners=8pt] (-2.5, 2) -- (-2.5, 2.24) -- (-1, 2.25) -- (-0.5, 2.5);
    \node at (-2.5, 8) (right start) {};
    
    \draw (-2.5, 8) -- (-2.5, 8 - 2);
    \draw[rounded corners=8pt] (-2.5, 8 - 2) -- (-2.5, 8 - 2.24) -- (-1, 8 - 2.25) -- (-0.5, 8 - 2.5);
    \end{tikzpicture}
%    \captionof{figure}{\small The square above represents the fundamental domain of the $n + 1$-torus with the periodic time dimension , $t \in [0, 1] / \sim $ runing vertically and the space dimensions $x \in ([0, 1]/\sim )^n$ collapsed into one-dimensional running horizotally.  The graph on the left is the graph of the smooth cut-off function that is supported in $U_1$, the blue region. The complement of the blue region, $U_1^c$ is the projection (from the cotangent bundle $T^*M$) of the characteristic set of $\Box - iQ$ onto the manifold $M$. The red region $U_2$ is an open neighbourhood of $U_1^c$. }
\end{center}

\end{frame}




% ------------------------------------------ Sandbox frame
\begin{frame}{Sandbox Frame}


Fourier analysis relates the (global) regularity of functions to their fourier transform. e.g. in the 'superposition of wave' picture, only waves with high frequency can approximate jump discontinuity, linking continuity with the decay of fourier coefficients. 
Microlocal analysis also keeps track of the direction of decay. 


\begin{theorem}[Atiyah-Singer index theorem]
    Given
    \begin{itemize}
        \item $X$ a smooth comapct manifold, 
        \item $E, F$ smooth vector bundles over $X$, 
        \item $P : \Gamma(E) \to \Gamma(F)$ be an elliptic differential operators between the space of sections of $E$ and $F$. 
    \end{itemize}
    Then, $P$ is Fredholm and its Fredholm index is related to its topological index. 
\end{theorem}
\end{frame}
\end{document}